% ============================================================
% Module 00, Unit 02 — Dot Products & Cross Products
% Exercises
% Threat Surfaces: Multivariable Calculus for AI Security
% fischer³ Education
% ============================================================

\newcommand{\coursemodule}{Module 00, Unit 02}
\newcommand{\coursetitle}{Dot Products \& Cross Products}
\newcommand{\coursedate}{February 2026}

\documentclass[11pt, letterpaper]{article}
% ============================================================
% latex-preamble.tex — Threat Surfaces: Multivariable Calculus for AI Security
% fischer³ Education
%
% PURPOSE:
%   Shared preamble included by all exercise and solution files.
%   Defines packages, page layout, theorem environments, and
%   course-specific notation macros.
%
% USAGE:
%   In any exercise or solution .tex file, place the following
%   at the top of the document before \begin{document}:
%
%     % ============================================================
% latex-preamble.tex — Threat Surfaces: Multivariable Calculus for AI Security
% fischer³ Education
%
% PURPOSE:
%   Shared preamble included by all exercise and solution files.
%   Defines packages, page layout, theorem environments, and
%   course-specific notation macros.
%
% USAGE:
%   In any exercise or solution .tex file, place the following
%   at the top of the document before \begin{document}:
%
%     % ============================================================
% latex-preamble.tex — Threat Surfaces: Multivariable Calculus for AI Security
% fischer³ Education
%
% PURPOSE:
%   Shared preamble included by all exercise and solution files.
%   Defines packages, page layout, theorem environments, and
%   course-specific notation macros.
%
% USAGE:
%   In any exercise or solution .tex file, place the following
%   at the top of the document before \begin{document}:
%
%     \input{../../assets/latex-preamble.tex}
%
%   Adjust the relative path as needed for the file's location
%   within the repository.
% ============================================================


% ------------------------------------------------------------
% PACKAGES
% ------------------------------------------------------------

% Page layout
\usepackage[
  margin=1.25in,
  top=1.0in,
  bottom=1.0in,
  headheight=14pt
]{geometry}

% Typography & encoding
\usepackage[T1]{fontenc}
\usepackage[utf8]{inputenc}
\usepackage{microtype}                % Improved justification
\usepackage{lmodern}                  % Scalable CM fonts

% Mathematics
\usepackage{amsmath}                  % Core math environments
\usepackage{amssymb}                  % Extended symbol set
\usepackage{amsthm}                   % Theorem environments
\usepackage{mathtools}                % Extensions to amsmath
\usepackage{bm}                       % Bold math (\bm)
\usepackage{cancel}                   % Strikethrough in math

% Graphics & color
\usepackage{graphicx}
\usepackage{xcolor}

% Course color palette
\definecolor{ts-blue}{RGB}{30, 100, 180}      % Primary — headings, links
\definecolor{ts-amber}{RGB}{200, 120, 20}     % Accent — warnings, highlights
\definecolor{ts-green}{RGB}{30, 140, 80}      % Statistical bridges
\definecolor{ts-gray}{RGB}{100, 100, 110}     % Secondary text
\definecolor{ts-light}{RGB}{245, 247, 250}    % Background tints

% Hyperlinks
\usepackage{hyperref}
\hypersetup{
  colorlinks=true,
  linkcolor=ts-blue,
  citecolor=ts-blue,
  urlcolor=ts-blue,
  pdfauthor={fischer\textsuperscript{3} Education},
  pdfcreator={LaTeX}
}

% Headers & footers
\usepackage{fancyhdr}
\pagestyle{fancy}
\fancyhf{}
\renewcommand{\headrulewidth}{0.4pt}
\fancyhead[L]{\small\color{ts-gray}\textit{Threat Surfaces}}
\fancyhead[R]{\small\color{ts-gray}\textit{\coursemodule}}   % Set \coursemodule in each file
\fancyfoot[C]{\small\color{ts-gray}\thepage}

% Tables
\usepackage{booktabs}
\usepackage{array}

% Lists
\usepackage{enumitem}
\setlist[enumerate,1]{label=\textbf{\arabic*.}}
\setlist[enumerate,2]{label=(\alph*)}

% Boxes & frames
\usepackage{tcolorbox}
\tcbuselibrary{breakable, skins, theorems}

% Code listings (for Python lab cross-references)
\usepackage{listings}
\lstset{
  basicstyle=\ttfamily\small,
  keywordstyle=\color{ts-blue},
  commentstyle=\color{ts-gray},
  stringstyle=\color{ts-green},
  numbers=left,
  numberstyle=\tiny\color{ts-gray},
  frame=single,
  breaklines=true,
  backgroundcolor=\color{ts-light}
}


% ------------------------------------------------------------
% THEOREM ENVIRONMENTS
% ------------------------------------------------------------

\theoremstyle{definition}
\newtheorem{definition}{Definition}[section]
\newtheorem{example}{Example}[section]
\newtheorem{exercise}{Exercise}[section]

\theoremstyle{plain}
\newtheorem{theorem}{Theorem}[section]
\newtheorem{lemma}[theorem]{Lemma}
\newtheorem{proposition}[theorem]{Proposition}
\newtheorem{corollary}[theorem]{Corollary}

\theoremstyle{remark}
\newtheorem*{remark}{Remark}
\newtheorem*{note}{Note}


% ------------------------------------------------------------
% TCOLORBOX ENVIRONMENTS
% ------------------------------------------------------------

% Problem statement box
\newtcolorbox{problembox}[1][]{
  enhanced,
  breakable,
  colback=ts-light,
  colframe=ts-blue,
  title={Problem},
  fonttitle=\bfseries,
  #1
}

% Statistical bridge box (green accent)
\newtcolorbox{bridgebox}[1][]{
  enhanced,
  breakable,
  colback=ts-light,
  colframe=ts-green,
  title={Statistical Bridge},
  fonttitle=\bfseries\color{ts-green},
  #1
}

% Warning / watch out box (amber accent)
\newtcolorbox{warningbox}[1][]{
  enhanced,
  breakable,
  colback=ts-light,
  colframe=ts-amber,
  title={Watch Out},
  fonttitle=\bfseries\color{ts-amber},
  #1
}

% Hint box (for exercise files — only appears in solution sets)
\newtcolorbox{hintbox}[1][]{
  enhanced,
  breakable,
  colback=ts-light,
  colframe=ts-gray,
  title={Hint},
  fonttitle=\bfseries\color{ts-gray},
  #1
}


% ------------------------------------------------------------
% NOTATION MACROS
% ------------------------------------------------------------

% --- Number systems ---
\newcommand{\R}{\mathbb{R}}             % Real numbers
\newcommand{\N}{\mathbb{N}}             % Natural numbers
\newcommand{\Z}{\mathbb{Z}}             % Integers
\newcommand{\Q}{\mathbb{Q}}             % Rationals
\newcommand{\C}{\mathbb{C}}             % Complex numbers

% --- Vectors & matrices ---
% Convention: bold lowercase for vectors, bold uppercase for matrices
\newcommand{\va}{\bm{a}}
\newcommand{\vb}{\bm{b}}
\newcommand{\vc}{\bm{c}}
\newcommand{\vd}{\bm{d}}
\newcommand{\ve}{\bm{e}}
\newcommand{\vf}{\bm{f}}
\newcommand{\vg}{\bm{g}}
\newcommand{\vh}{\bm{h}}
\newcommand{\vi}{\bm{i}}               % Unit vector i
\newcommand{\vj}{\bm{j}}               % Unit vector j
\newcommand{\vk}{\bm{k}}               % Unit vector k
\newcommand{\vn}{\bm{n}}               % Normal vector
\newcommand{\vr}{\bm{r}}               % Position vector
\newcommand{\vu}{\bm{u}}               % Unit direction vector
\newcommand{\vv}{\bm{v}}
\newcommand{\vw}{\bm{w}}
\newcommand{\vx}{\bm{x}}               % Input vector
\newcommand{\vy}{\bm{y}}               % Output / observation vector
\newcommand{\vz}{\bm{z}}               % Latent vector
\newcommand{\vzero}{\bm{0}}            % Zero vector

% Statistical vectors
\newcommand{\vmu}{\bm{\mu}}            % Mean vector
\newcommand{\vbeta}{\bm{\beta}}        % Regression coefficients
\newcommand{\vtheta}{\bm{\theta}}      % Parameter vector
\newcommand{\veps}{\bm{\varepsilon}}   % Error vector

% Matrices
\newcommand{\mA}{\bm{A}}
\newcommand{\mB}{\bm{B}}
\newcommand{\mC}{\bm{C}}
\newcommand{\mH}{\bm{H}}               % Hessian (also \Hess below)
\newcommand{\mI}{\bm{I}}               % Identity matrix
\newcommand{\mJ}{\bm{J}}               % Jacobian (also \Jac below)
\newcommand{\mK}{\bm{K}}               % Kernel matrix
\newcommand{\mL}{\bm{L}}               % Lower triangular / Cholesky
\newcommand{\mQ}{\bm{Q}}               % Orthogonal matrix
\newcommand{\mR}{\bm{R}}               % Upper triangular
\newcommand{\mS}{\bm{S}}               % Sample covariance
\newcommand{\mSigma}{\bm{\Sigma}}      % Covariance matrix
\newcommand{\mX}{\bm{X}}               % Design matrix
\newcommand{\mY}{\bm{Y}}

% --- Calculus operators ---
\newcommand{\grad}{\nabla}                              % Gradient ∇
\newcommand{\Grad}[1]{\nabla #1}                        % Gradient of #1
\newcommand{\GradP}[2]{\nabla_{#1} #2}                 % Gradient of #2 w.r.t. #1
\newcommand{\Div}[1]{\nabla \cdot #1}                  % Divergence
\newcommand{\Curl}[1]{\nabla \times #1}                % Curl
\newcommand{\Lap}[1]{\nabla^2 #1}                      % Laplacian
\newcommand{\Hess}[1]{\bm{H}_{#1}}                     % Hessian of #1
\newcommand{\Jac}[1]{\bm{J}_{#1}}                      % Jacobian of #1

% Partial derivative shorthand
\newcommand{\pd}[2]{\frac{\partial #1}{\partial #2}}           % ∂f/∂x
\newcommand{\pdd}[2]{\frac{\partial^2 #1}{\partial #2^2}}      % ∂²f/∂x²
\newcommand{\pdm}[3]{\frac{\partial^2 #1}{\partial #2\,\partial #3}}  % ∂²f/∂x∂y

% Ordinary derivative shorthand
\newcommand{\od}[2]{\frac{d #1}{d #2}}                         % df/dx
\newcommand{\odd}[2]{\frac{d^2 #1}{d #2^2}}                    % d²f/dx²

% Integrals
\newcommand{\dx}{\,dx}
\newcommand{\dy}{\,dy}
\newcommand{\dz}{\,dz}
\newcommand{\dt}{\,dt}
\newcommand{\dA}{\,dA}
\newcommand{\dV}{\,dV}
\newcommand{\ds}{\,ds}
\newcommand{\dvx}{\,d\vx}              % Vector integral element

% --- Statistical notation ---
\newcommand{\E}[1]{\mathbb{E}\!\left[#1\right]}          % Expectation
\newcommand{\Var}[1]{\operatorname{Var}\!\left(#1\right)} % Variance
\newcommand{\Cov}[2]{\operatorname{Cov}\!\left(#1,\,#2\right)} % Covariance
\newcommand{\Cor}[2]{\operatorname{Cor}\!\left(#1,\,#2\right)} % Correlation

\newcommand{\given}{\,\vert\,}                           % Conditional bar
\newcommand{\iid}{\overset{\text{iid}}{\sim}}           % iid ~
\newcommand{\ind}{\overset{\text{ind}}{\sim}}           % ind ~

% Distributions
\newcommand{\Normal}{\mathcal{N}}                        % Normal distribution
\newcommand{\MVN}[2]{\mathcal{N}\!\left(#1,\,#2\right)} % Multivariate normal
\newcommand{\Bern}{\operatorname{Bernoulli}}
\newcommand{\Bin}{\operatorname{Binomial}}
\newcommand{\Pois}{\operatorname{Poisson}}
\newcommand{\Unif}{\operatorname{Uniform}}
\newcommand{\Gam}{\operatorname{Gamma}}
\newcommand{\Beta}{\operatorname{Beta}}

% Likelihood & inference
\newcommand{\Lik}{\mathcal{L}}                          % Likelihood
\newcommand{\loglik}{\ell}                              % Log-likelihood
\newcommand{\Score}[1]{s(#1)}                           % Score function
\newcommand{\FI}{\mathcal{I}}                           % Fisher information
\newcommand{\MLE}[1]{\hat{#1}_{\text{MLE}}}            % MLE estimator
\newcommand{\MAP}[1]{\hat{#1}_{\text{MAP}}}            % MAP estimator
\newcommand{\posterior}{p(\vtheta \given \vx)}          % Posterior shorthand
\newcommand{\prior}{p(\vtheta)}                         % Prior shorthand
\newcommand{\likelihood}{p(\vx \given \vtheta)}        % Likelihood shorthand
\newcommand{\KL}[2]{D_{\text{KL}}\!\left(#1\,\|\,#2\right)} % KL divergence

% --- Linear algebra ---
\newcommand{\tr}[1]{\operatorname{tr}\!\left(#1\right)}        % Trace
\newcommand{\rank}[1]{\operatorname{rank}\!\left(#1\right)}    % Rank
\newcommand{\diag}[1]{\operatorname{diag}\!\left(#1\right)}    % Diagonal matrix
\newcommand{\spn}[1]{\operatorname{span}\!\left\{#1\right\}}   % Span
\newcommand{\norm}[1]{\left\|#1\right\|}                       % Norm
\newcommand{\abs}[1]{\left|#1\right|}                          % Absolute value
\newcommand{\inner}[2]{\langle #1,\, #2 \rangle}               % Inner product
\newcommand{\T}{^{\top}}                                        % Transpose superscript

% Positive (semi-)definiteness shorthand
\newcommand{\psd}{\succeq 0}    % Positive semi-definite
\newcommand{\pd}{\succ 0}       % Positive definite (Note: redefines \pd — see below)
% Note: \pd is overloaded: partial derivative above, positive definite here.
% In ambiguous contexts, use \succ 0 directly for positive definite.

% --- Miscellaneous ---
\newcommand{\defeq}{\coloneqq}                          % Definition equality :=
\newcommand{\st}{\text{ s.t. }}                         % "such that"
\newcommand{\approxeq}{\approx}                         % Approximate equality
\newcommand{\BigO}[1]{\mathcal{O}\!\left(#1\right)}    % Big-O notation
\newcommand{\sigmoid}[1]{\sigma\!\left(#1\right)}      % Sigmoid function
\newcommand{\softmax}[1]{\operatorname{softmax}\!\left(#1\right)}


% ------------------------------------------------------------
% DOCUMENT METADATA COMMANDS
% (Set these in each individual exercise file)
% ------------------------------------------------------------
%
%   \newcommand{\coursemodule}{Module 02, Unit 03}
%   \newcommand{\coursetitle}{The Multivariable Chain Rule}
%   \newcommand{\coursedate}{February 2026}
%
% These are used in the header/footer and title block.
% Define them BEFORE \input{...latex-preamble.tex} or immediately after.
%
% Default definitions (overridden per-file):
\providecommand{\coursemodule}{[Module TBD]}
\providecommand{\coursetitle}{[Unit Title TBD]}
\providecommand{\coursedate}{\today}


% ------------------------------------------------------------
% TITLE BLOCK MACRO
% ------------------------------------------------------------
%
% Usage in exercise files:
%   \maketitleblock
%
\newcommand{\maketitleblock}{%
  \begin{center}
    {\LARGE\bfseries\color{ts-blue} Threat Surfaces}\\[4pt]
    {\large Multivariable Calculus for AI Security}\\[2pt]
    {\small\color{ts-gray} fischer\textsuperscript{3} Education}\\[12pt]
    \rule{\linewidth}{0.4pt}\\[6pt]
    {\large\bfseries \coursemodule: \coursetitle}\\[4pt]
    {\small\color{ts-gray} Exercises \textbar\ \coursedate}\\[4pt]
    \rule{\linewidth}{0.4pt}
  \end{center}
  \vspace{12pt}
}


% ============================================================
% END OF PREAMBLE
% ============================================================

%
%   Adjust the relative path as needed for the file's location
%   within the repository.
% ============================================================


% ------------------------------------------------------------
% PACKAGES
% ------------------------------------------------------------

% Page layout
\usepackage[
  margin=1.25in,
  top=1.0in,
  bottom=1.0in,
  headheight=14pt
]{geometry}

% Typography & encoding
\usepackage[T1]{fontenc}
\usepackage[utf8]{inputenc}
\usepackage{microtype}                % Improved justification
\usepackage{lmodern}                  % Scalable CM fonts

% Mathematics
\usepackage{amsmath}                  % Core math environments
\usepackage{amssymb}                  % Extended symbol set
\usepackage{amsthm}                   % Theorem environments
\usepackage{mathtools}                % Extensions to amsmath
\usepackage{bm}                       % Bold math (\bm)
\usepackage{cancel}                   % Strikethrough in math

% Graphics & color
\usepackage{graphicx}
\usepackage{xcolor}

% Course color palette
\definecolor{ts-blue}{RGB}{30, 100, 180}      % Primary — headings, links
\definecolor{ts-amber}{RGB}{200, 120, 20}     % Accent — warnings, highlights
\definecolor{ts-green}{RGB}{30, 140, 80}      % Statistical bridges
\definecolor{ts-gray}{RGB}{100, 100, 110}     % Secondary text
\definecolor{ts-light}{RGB}{245, 247, 250}    % Background tints

% Hyperlinks
\usepackage{hyperref}
\hypersetup{
  colorlinks=true,
  linkcolor=ts-blue,
  citecolor=ts-blue,
  urlcolor=ts-blue,
  pdfauthor={fischer\textsuperscript{3} Education},
  pdfcreator={LaTeX}
}

% Headers & footers
\usepackage{fancyhdr}
\pagestyle{fancy}
\fancyhf{}
\renewcommand{\headrulewidth}{0.4pt}
\fancyhead[L]{\small\color{ts-gray}\textit{Threat Surfaces}}
\fancyhead[R]{\small\color{ts-gray}\textit{\coursemodule}}   % Set \coursemodule in each file
\fancyfoot[C]{\small\color{ts-gray}\thepage}

% Tables
\usepackage{booktabs}
\usepackage{array}

% Lists
\usepackage{enumitem}
\setlist[enumerate,1]{label=\textbf{\arabic*.}}
\setlist[enumerate,2]{label=(\alph*)}

% Boxes & frames
\usepackage{tcolorbox}
\tcbuselibrary{breakable, skins, theorems}

% Code listings (for Python lab cross-references)
\usepackage{listings}
\lstset{
  basicstyle=\ttfamily\small,
  keywordstyle=\color{ts-blue},
  commentstyle=\color{ts-gray},
  stringstyle=\color{ts-green},
  numbers=left,
  numberstyle=\tiny\color{ts-gray},
  frame=single,
  breaklines=true,
  backgroundcolor=\color{ts-light}
}


% ------------------------------------------------------------
% THEOREM ENVIRONMENTS
% ------------------------------------------------------------

\theoremstyle{definition}
\newtheorem{definition}{Definition}[section]
\newtheorem{example}{Example}[section]
\newtheorem{exercise}{Exercise}[section]

\theoremstyle{plain}
\newtheorem{theorem}{Theorem}[section]
\newtheorem{lemma}[theorem]{Lemma}
\newtheorem{proposition}[theorem]{Proposition}
\newtheorem{corollary}[theorem]{Corollary}

\theoremstyle{remark}
\newtheorem*{remark}{Remark}
\newtheorem*{note}{Note}


% ------------------------------------------------------------
% TCOLORBOX ENVIRONMENTS
% ------------------------------------------------------------

% Problem statement box
\newtcolorbox{problembox}[1][]{
  enhanced,
  breakable,
  colback=ts-light,
  colframe=ts-blue,
  title={Problem},
  fonttitle=\bfseries,
  #1
}

% Statistical bridge box (green accent)
\newtcolorbox{bridgebox}[1][]{
  enhanced,
  breakable,
  colback=ts-light,
  colframe=ts-green,
  title={Statistical Bridge},
  fonttitle=\bfseries\color{ts-green},
  #1
}

% Warning / watch out box (amber accent)
\newtcolorbox{warningbox}[1][]{
  enhanced,
  breakable,
  colback=ts-light,
  colframe=ts-amber,
  title={Watch Out},
  fonttitle=\bfseries\color{ts-amber},
  #1
}

% Hint box (for exercise files — only appears in solution sets)
\newtcolorbox{hintbox}[1][]{
  enhanced,
  breakable,
  colback=ts-light,
  colframe=ts-gray,
  title={Hint},
  fonttitle=\bfseries\color{ts-gray},
  #1
}


% ------------------------------------------------------------
% NOTATION MACROS
% ------------------------------------------------------------

% --- Number systems ---
\newcommand{\R}{\mathbb{R}}             % Real numbers
\newcommand{\N}{\mathbb{N}}             % Natural numbers
\newcommand{\Z}{\mathbb{Z}}             % Integers
\newcommand{\Q}{\mathbb{Q}}             % Rationals
\newcommand{\C}{\mathbb{C}}             % Complex numbers

% --- Vectors & matrices ---
% Convention: bold lowercase for vectors, bold uppercase for matrices
\newcommand{\va}{\bm{a}}
\newcommand{\vb}{\bm{b}}
\newcommand{\vc}{\bm{c}}
\newcommand{\vd}{\bm{d}}
\newcommand{\ve}{\bm{e}}
\newcommand{\vf}{\bm{f}}
\newcommand{\vg}{\bm{g}}
\newcommand{\vh}{\bm{h}}
\newcommand{\vi}{\bm{i}}               % Unit vector i
\newcommand{\vj}{\bm{j}}               % Unit vector j
\newcommand{\vk}{\bm{k}}               % Unit vector k
\newcommand{\vn}{\bm{n}}               % Normal vector
\newcommand{\vr}{\bm{r}}               % Position vector
\newcommand{\vu}{\bm{u}}               % Unit direction vector
\newcommand{\vv}{\bm{v}}
\newcommand{\vw}{\bm{w}}
\newcommand{\vx}{\bm{x}}               % Input vector
\newcommand{\vy}{\bm{y}}               % Output / observation vector
\newcommand{\vz}{\bm{z}}               % Latent vector
\newcommand{\vzero}{\bm{0}}            % Zero vector

% Statistical vectors
\newcommand{\vmu}{\bm{\mu}}            % Mean vector
\newcommand{\vbeta}{\bm{\beta}}        % Regression coefficients
\newcommand{\vtheta}{\bm{\theta}}      % Parameter vector
\newcommand{\veps}{\bm{\varepsilon}}   % Error vector

% Matrices
\newcommand{\mA}{\bm{A}}
\newcommand{\mB}{\bm{B}}
\newcommand{\mC}{\bm{C}}
\newcommand{\mH}{\bm{H}}               % Hessian (also \Hess below)
\newcommand{\mI}{\bm{I}}               % Identity matrix
\newcommand{\mJ}{\bm{J}}               % Jacobian (also \Jac below)
\newcommand{\mK}{\bm{K}}               % Kernel matrix
\newcommand{\mL}{\bm{L}}               % Lower triangular / Cholesky
\newcommand{\mQ}{\bm{Q}}               % Orthogonal matrix
\newcommand{\mR}{\bm{R}}               % Upper triangular
\newcommand{\mS}{\bm{S}}               % Sample covariance
\newcommand{\mSigma}{\bm{\Sigma}}      % Covariance matrix
\newcommand{\mX}{\bm{X}}               % Design matrix
\newcommand{\mY}{\bm{Y}}

% --- Calculus operators ---
\newcommand{\grad}{\nabla}                              % Gradient ∇
\newcommand{\Grad}[1]{\nabla #1}                        % Gradient of #1
\newcommand{\GradP}[2]{\nabla_{#1} #2}                 % Gradient of #2 w.r.t. #1
\newcommand{\Div}[1]{\nabla \cdot #1}                  % Divergence
\newcommand{\Curl}[1]{\nabla \times #1}                % Curl
\newcommand{\Lap}[1]{\nabla^2 #1}                      % Laplacian
\newcommand{\Hess}[1]{\bm{H}_{#1}}                     % Hessian of #1
\newcommand{\Jac}[1]{\bm{J}_{#1}}                      % Jacobian of #1

% Partial derivative shorthand
\newcommand{\pd}[2]{\frac{\partial #1}{\partial #2}}           % ∂f/∂x
\newcommand{\pdd}[2]{\frac{\partial^2 #1}{\partial #2^2}}      % ∂²f/∂x²
\newcommand{\pdm}[3]{\frac{\partial^2 #1}{\partial #2\,\partial #3}}  % ∂²f/∂x∂y

% Ordinary derivative shorthand
\newcommand{\od}[2]{\frac{d #1}{d #2}}                         % df/dx
\newcommand{\odd}[2]{\frac{d^2 #1}{d #2^2}}                    % d²f/dx²

% Integrals
\newcommand{\dx}{\,dx}
\newcommand{\dy}{\,dy}
\newcommand{\dz}{\,dz}
\newcommand{\dt}{\,dt}
\newcommand{\dA}{\,dA}
\newcommand{\dV}{\,dV}
\newcommand{\ds}{\,ds}
\newcommand{\dvx}{\,d\vx}              % Vector integral element

% --- Statistical notation ---
\newcommand{\E}[1]{\mathbb{E}\!\left[#1\right]}          % Expectation
\newcommand{\Var}[1]{\operatorname{Var}\!\left(#1\right)} % Variance
\newcommand{\Cov}[2]{\operatorname{Cov}\!\left(#1,\,#2\right)} % Covariance
\newcommand{\Cor}[2]{\operatorname{Cor}\!\left(#1,\,#2\right)} % Correlation

\newcommand{\given}{\,\vert\,}                           % Conditional bar
\newcommand{\iid}{\overset{\text{iid}}{\sim}}           % iid ~
\newcommand{\ind}{\overset{\text{ind}}{\sim}}           % ind ~

% Distributions
\newcommand{\Normal}{\mathcal{N}}                        % Normal distribution
\newcommand{\MVN}[2]{\mathcal{N}\!\left(#1,\,#2\right)} % Multivariate normal
\newcommand{\Bern}{\operatorname{Bernoulli}}
\newcommand{\Bin}{\operatorname{Binomial}}
\newcommand{\Pois}{\operatorname{Poisson}}
\newcommand{\Unif}{\operatorname{Uniform}}
\newcommand{\Gam}{\operatorname{Gamma}}
\newcommand{\Beta}{\operatorname{Beta}}

% Likelihood & inference
\newcommand{\Lik}{\mathcal{L}}                          % Likelihood
\newcommand{\loglik}{\ell}                              % Log-likelihood
\newcommand{\Score}[1]{s(#1)}                           % Score function
\newcommand{\FI}{\mathcal{I}}                           % Fisher information
\newcommand{\MLE}[1]{\hat{#1}_{\text{MLE}}}            % MLE estimator
\newcommand{\MAP}[1]{\hat{#1}_{\text{MAP}}}            % MAP estimator
\newcommand{\posterior}{p(\vtheta \given \vx)}          % Posterior shorthand
\newcommand{\prior}{p(\vtheta)}                         % Prior shorthand
\newcommand{\likelihood}{p(\vx \given \vtheta)}        % Likelihood shorthand
\newcommand{\KL}[2]{D_{\text{KL}}\!\left(#1\,\|\,#2\right)} % KL divergence

% --- Linear algebra ---
\newcommand{\tr}[1]{\operatorname{tr}\!\left(#1\right)}        % Trace
\newcommand{\rank}[1]{\operatorname{rank}\!\left(#1\right)}    % Rank
\newcommand{\diag}[1]{\operatorname{diag}\!\left(#1\right)}    % Diagonal matrix
\newcommand{\spn}[1]{\operatorname{span}\!\left\{#1\right\}}   % Span
\newcommand{\norm}[1]{\left\|#1\right\|}                       % Norm
\newcommand{\abs}[1]{\left|#1\right|}                          % Absolute value
\newcommand{\inner}[2]{\langle #1,\, #2 \rangle}               % Inner product
\newcommand{\T}{^{\top}}                                        % Transpose superscript

% Positive (semi-)definiteness shorthand
\newcommand{\psd}{\succeq 0}    % Positive semi-definite
\newcommand{\pd}{\succ 0}       % Positive definite (Note: redefines \pd — see below)
% Note: \pd is overloaded: partial derivative above, positive definite here.
% In ambiguous contexts, use \succ 0 directly for positive definite.

% --- Miscellaneous ---
\newcommand{\defeq}{\coloneqq}                          % Definition equality :=
\newcommand{\st}{\text{ s.t. }}                         % "such that"
\newcommand{\approxeq}{\approx}                         % Approximate equality
\newcommand{\BigO}[1]{\mathcal{O}\!\left(#1\right)}    % Big-O notation
\newcommand{\sigmoid}[1]{\sigma\!\left(#1\right)}      % Sigmoid function
\newcommand{\softmax}[1]{\operatorname{softmax}\!\left(#1\right)}


% ------------------------------------------------------------
% DOCUMENT METADATA COMMANDS
% (Set these in each individual exercise file)
% ------------------------------------------------------------
%
%   \newcommand{\coursemodule}{Module 02, Unit 03}
%   \newcommand{\coursetitle}{The Multivariable Chain Rule}
%   \newcommand{\coursedate}{February 2026}
%
% These are used in the header/footer and title block.
% Define them BEFORE % ============================================================
% latex-preamble.tex — Threat Surfaces: Multivariable Calculus for AI Security
% fischer³ Education
%
% PURPOSE:
%   Shared preamble included by all exercise and solution files.
%   Defines packages, page layout, theorem environments, and
%   course-specific notation macros.
%
% USAGE:
%   In any exercise or solution .tex file, place the following
%   at the top of the document before \begin{document}:
%
%     \input{../../assets/latex-preamble.tex}
%
%   Adjust the relative path as needed for the file's location
%   within the repository.
% ============================================================


% ------------------------------------------------------------
% PACKAGES
% ------------------------------------------------------------

% Page layout
\usepackage[
  margin=1.25in,
  top=1.0in,
  bottom=1.0in,
  headheight=14pt
]{geometry}

% Typography & encoding
\usepackage[T1]{fontenc}
\usepackage[utf8]{inputenc}
\usepackage{microtype}                % Improved justification
\usepackage{lmodern}                  % Scalable CM fonts

% Mathematics
\usepackage{amsmath}                  % Core math environments
\usepackage{amssymb}                  % Extended symbol set
\usepackage{amsthm}                   % Theorem environments
\usepackage{mathtools}                % Extensions to amsmath
\usepackage{bm}                       % Bold math (\bm)
\usepackage{cancel}                   % Strikethrough in math

% Graphics & color
\usepackage{graphicx}
\usepackage{xcolor}

% Course color palette
\definecolor{ts-blue}{RGB}{30, 100, 180}      % Primary — headings, links
\definecolor{ts-amber}{RGB}{200, 120, 20}     % Accent — warnings, highlights
\definecolor{ts-green}{RGB}{30, 140, 80}      % Statistical bridges
\definecolor{ts-gray}{RGB}{100, 100, 110}     % Secondary text
\definecolor{ts-light}{RGB}{245, 247, 250}    % Background tints

% Hyperlinks
\usepackage{hyperref}
\hypersetup{
  colorlinks=true,
  linkcolor=ts-blue,
  citecolor=ts-blue,
  urlcolor=ts-blue,
  pdfauthor={fischer\textsuperscript{3} Education},
  pdfcreator={LaTeX}
}

% Headers & footers
\usepackage{fancyhdr}
\pagestyle{fancy}
\fancyhf{}
\renewcommand{\headrulewidth}{0.4pt}
\fancyhead[L]{\small\color{ts-gray}\textit{Threat Surfaces}}
\fancyhead[R]{\small\color{ts-gray}\textit{\coursemodule}}   % Set \coursemodule in each file
\fancyfoot[C]{\small\color{ts-gray}\thepage}

% Tables
\usepackage{booktabs}
\usepackage{array}

% Lists
\usepackage{enumitem}
\setlist[enumerate,1]{label=\textbf{\arabic*.}}
\setlist[enumerate,2]{label=(\alph*)}

% Boxes & frames
\usepackage{tcolorbox}
\tcbuselibrary{breakable, skins, theorems}

% Code listings (for Python lab cross-references)
\usepackage{listings}
\lstset{
  basicstyle=\ttfamily\small,
  keywordstyle=\color{ts-blue},
  commentstyle=\color{ts-gray},
  stringstyle=\color{ts-green},
  numbers=left,
  numberstyle=\tiny\color{ts-gray},
  frame=single,
  breaklines=true,
  backgroundcolor=\color{ts-light}
}


% ------------------------------------------------------------
% THEOREM ENVIRONMENTS
% ------------------------------------------------------------

\theoremstyle{definition}
\newtheorem{definition}{Definition}[section]
\newtheorem{example}{Example}[section]
\newtheorem{exercise}{Exercise}[section]

\theoremstyle{plain}
\newtheorem{theorem}{Theorem}[section]
\newtheorem{lemma}[theorem]{Lemma}
\newtheorem{proposition}[theorem]{Proposition}
\newtheorem{corollary}[theorem]{Corollary}

\theoremstyle{remark}
\newtheorem*{remark}{Remark}
\newtheorem*{note}{Note}


% ------------------------------------------------------------
% TCOLORBOX ENVIRONMENTS
% ------------------------------------------------------------

% Problem statement box
\newtcolorbox{problembox}[1][]{
  enhanced,
  breakable,
  colback=ts-light,
  colframe=ts-blue,
  title={Problem},
  fonttitle=\bfseries,
  #1
}

% Statistical bridge box (green accent)
\newtcolorbox{bridgebox}[1][]{
  enhanced,
  breakable,
  colback=ts-light,
  colframe=ts-green,
  title={Statistical Bridge},
  fonttitle=\bfseries\color{ts-green},
  #1
}

% Warning / watch out box (amber accent)
\newtcolorbox{warningbox}[1][]{
  enhanced,
  breakable,
  colback=ts-light,
  colframe=ts-amber,
  title={Watch Out},
  fonttitle=\bfseries\color{ts-amber},
  #1
}

% Hint box (for exercise files — only appears in solution sets)
\newtcolorbox{hintbox}[1][]{
  enhanced,
  breakable,
  colback=ts-light,
  colframe=ts-gray,
  title={Hint},
  fonttitle=\bfseries\color{ts-gray},
  #1
}


% ------------------------------------------------------------
% NOTATION MACROS
% ------------------------------------------------------------

% --- Number systems ---
\newcommand{\R}{\mathbb{R}}             % Real numbers
\newcommand{\N}{\mathbb{N}}             % Natural numbers
\newcommand{\Z}{\mathbb{Z}}             % Integers
\newcommand{\Q}{\mathbb{Q}}             % Rationals
\newcommand{\C}{\mathbb{C}}             % Complex numbers

% --- Vectors & matrices ---
% Convention: bold lowercase for vectors, bold uppercase for matrices
\newcommand{\va}{\bm{a}}
\newcommand{\vb}{\bm{b}}
\newcommand{\vc}{\bm{c}}
\newcommand{\vd}{\bm{d}}
\newcommand{\ve}{\bm{e}}
\newcommand{\vf}{\bm{f}}
\newcommand{\vg}{\bm{g}}
\newcommand{\vh}{\bm{h}}
\newcommand{\vi}{\bm{i}}               % Unit vector i
\newcommand{\vj}{\bm{j}}               % Unit vector j
\newcommand{\vk}{\bm{k}}               % Unit vector k
\newcommand{\vn}{\bm{n}}               % Normal vector
\newcommand{\vr}{\bm{r}}               % Position vector
\newcommand{\vu}{\bm{u}}               % Unit direction vector
\newcommand{\vv}{\bm{v}}
\newcommand{\vw}{\bm{w}}
\newcommand{\vx}{\bm{x}}               % Input vector
\newcommand{\vy}{\bm{y}}               % Output / observation vector
\newcommand{\vz}{\bm{z}}               % Latent vector
\newcommand{\vzero}{\bm{0}}            % Zero vector

% Statistical vectors
\newcommand{\vmu}{\bm{\mu}}            % Mean vector
\newcommand{\vbeta}{\bm{\beta}}        % Regression coefficients
\newcommand{\vtheta}{\bm{\theta}}      % Parameter vector
\newcommand{\veps}{\bm{\varepsilon}}   % Error vector

% Matrices
\newcommand{\mA}{\bm{A}}
\newcommand{\mB}{\bm{B}}
\newcommand{\mC}{\bm{C}}
\newcommand{\mH}{\bm{H}}               % Hessian (also \Hess below)
\newcommand{\mI}{\bm{I}}               % Identity matrix
\newcommand{\mJ}{\bm{J}}               % Jacobian (also \Jac below)
\newcommand{\mK}{\bm{K}}               % Kernel matrix
\newcommand{\mL}{\bm{L}}               % Lower triangular / Cholesky
\newcommand{\mQ}{\bm{Q}}               % Orthogonal matrix
\newcommand{\mR}{\bm{R}}               % Upper triangular
\newcommand{\mS}{\bm{S}}               % Sample covariance
\newcommand{\mSigma}{\bm{\Sigma}}      % Covariance matrix
\newcommand{\mX}{\bm{X}}               % Design matrix
\newcommand{\mY}{\bm{Y}}

% --- Calculus operators ---
\newcommand{\grad}{\nabla}                              % Gradient ∇
\newcommand{\Grad}[1]{\nabla #1}                        % Gradient of #1
\newcommand{\GradP}[2]{\nabla_{#1} #2}                 % Gradient of #2 w.r.t. #1
\newcommand{\Div}[1]{\nabla \cdot #1}                  % Divergence
\newcommand{\Curl}[1]{\nabla \times #1}                % Curl
\newcommand{\Lap}[1]{\nabla^2 #1}                      % Laplacian
\newcommand{\Hess}[1]{\bm{H}_{#1}}                     % Hessian of #1
\newcommand{\Jac}[1]{\bm{J}_{#1}}                      % Jacobian of #1

% Partial derivative shorthand
\newcommand{\pd}[2]{\frac{\partial #1}{\partial #2}}           % ∂f/∂x
\newcommand{\pdd}[2]{\frac{\partial^2 #1}{\partial #2^2}}      % ∂²f/∂x²
\newcommand{\pdm}[3]{\frac{\partial^2 #1}{\partial #2\,\partial #3}}  % ∂²f/∂x∂y

% Ordinary derivative shorthand
\newcommand{\od}[2]{\frac{d #1}{d #2}}                         % df/dx
\newcommand{\odd}[2]{\frac{d^2 #1}{d #2^2}}                    % d²f/dx²

% Integrals
\newcommand{\dx}{\,dx}
\newcommand{\dy}{\,dy}
\newcommand{\dz}{\,dz}
\newcommand{\dt}{\,dt}
\newcommand{\dA}{\,dA}
\newcommand{\dV}{\,dV}
\newcommand{\ds}{\,ds}
\newcommand{\dvx}{\,d\vx}              % Vector integral element

% --- Statistical notation ---
\newcommand{\E}[1]{\mathbb{E}\!\left[#1\right]}          % Expectation
\newcommand{\Var}[1]{\operatorname{Var}\!\left(#1\right)} % Variance
\newcommand{\Cov}[2]{\operatorname{Cov}\!\left(#1,\,#2\right)} % Covariance
\newcommand{\Cor}[2]{\operatorname{Cor}\!\left(#1,\,#2\right)} % Correlation

\newcommand{\given}{\,\vert\,}                           % Conditional bar
\newcommand{\iid}{\overset{\text{iid}}{\sim}}           % iid ~
\newcommand{\ind}{\overset{\text{ind}}{\sim}}           % ind ~

% Distributions
\newcommand{\Normal}{\mathcal{N}}                        % Normal distribution
\newcommand{\MVN}[2]{\mathcal{N}\!\left(#1,\,#2\right)} % Multivariate normal
\newcommand{\Bern}{\operatorname{Bernoulli}}
\newcommand{\Bin}{\operatorname{Binomial}}
\newcommand{\Pois}{\operatorname{Poisson}}
\newcommand{\Unif}{\operatorname{Uniform}}
\newcommand{\Gam}{\operatorname{Gamma}}
\newcommand{\Beta}{\operatorname{Beta}}

% Likelihood & inference
\newcommand{\Lik}{\mathcal{L}}                          % Likelihood
\newcommand{\loglik}{\ell}                              % Log-likelihood
\newcommand{\Score}[1]{s(#1)}                           % Score function
\newcommand{\FI}{\mathcal{I}}                           % Fisher information
\newcommand{\MLE}[1]{\hat{#1}_{\text{MLE}}}            % MLE estimator
\newcommand{\MAP}[1]{\hat{#1}_{\text{MAP}}}            % MAP estimator
\newcommand{\posterior}{p(\vtheta \given \vx)}          % Posterior shorthand
\newcommand{\prior}{p(\vtheta)}                         % Prior shorthand
\newcommand{\likelihood}{p(\vx \given \vtheta)}        % Likelihood shorthand
\newcommand{\KL}[2]{D_{\text{KL}}\!\left(#1\,\|\,#2\right)} % KL divergence

% --- Linear algebra ---
\newcommand{\tr}[1]{\operatorname{tr}\!\left(#1\right)}        % Trace
\newcommand{\rank}[1]{\operatorname{rank}\!\left(#1\right)}    % Rank
\newcommand{\diag}[1]{\operatorname{diag}\!\left(#1\right)}    % Diagonal matrix
\newcommand{\spn}[1]{\operatorname{span}\!\left\{#1\right\}}   % Span
\newcommand{\norm}[1]{\left\|#1\right\|}                       % Norm
\newcommand{\abs}[1]{\left|#1\right|}                          % Absolute value
\newcommand{\inner}[2]{\langle #1,\, #2 \rangle}               % Inner product
\newcommand{\T}{^{\top}}                                        % Transpose superscript

% Positive (semi-)definiteness shorthand
\newcommand{\psd}{\succeq 0}    % Positive semi-definite
\newcommand{\pd}{\succ 0}       % Positive definite (Note: redefines \pd — see below)
% Note: \pd is overloaded: partial derivative above, positive definite here.
% In ambiguous contexts, use \succ 0 directly for positive definite.

% --- Miscellaneous ---
\newcommand{\defeq}{\coloneqq}                          % Definition equality :=
\newcommand{\st}{\text{ s.t. }}                         % "such that"
\newcommand{\approxeq}{\approx}                         % Approximate equality
\newcommand{\BigO}[1]{\mathcal{O}\!\left(#1\right)}    % Big-O notation
\newcommand{\sigmoid}[1]{\sigma\!\left(#1\right)}      % Sigmoid function
\newcommand{\softmax}[1]{\operatorname{softmax}\!\left(#1\right)}


% ------------------------------------------------------------
% DOCUMENT METADATA COMMANDS
% (Set these in each individual exercise file)
% ------------------------------------------------------------
%
%   \newcommand{\coursemodule}{Module 02, Unit 03}
%   \newcommand{\coursetitle}{The Multivariable Chain Rule}
%   \newcommand{\coursedate}{February 2026}
%
% These are used in the header/footer and title block.
% Define them BEFORE \input{...latex-preamble.tex} or immediately after.
%
% Default definitions (overridden per-file):
\providecommand{\coursemodule}{[Module TBD]}
\providecommand{\coursetitle}{[Unit Title TBD]}
\providecommand{\coursedate}{\today}


% ------------------------------------------------------------
% TITLE BLOCK MACRO
% ------------------------------------------------------------
%
% Usage in exercise files:
%   \maketitleblock
%
\newcommand{\maketitleblock}{%
  \begin{center}
    {\LARGE\bfseries\color{ts-blue} Threat Surfaces}\\[4pt]
    {\large Multivariable Calculus for AI Security}\\[2pt]
    {\small\color{ts-gray} fischer\textsuperscript{3} Education}\\[12pt]
    \rule{\linewidth}{0.4pt}\\[6pt]
    {\large\bfseries \coursemodule: \coursetitle}\\[4pt]
    {\small\color{ts-gray} Exercises \textbar\ \coursedate}\\[4pt]
    \rule{\linewidth}{0.4pt}
  \end{center}
  \vspace{12pt}
}


% ============================================================
% END OF PREAMBLE
% ============================================================
 or immediately after.
%
% Default definitions (overridden per-file):
\providecommand{\coursemodule}{[Module TBD]}
\providecommand{\coursetitle}{[Unit Title TBD]}
\providecommand{\coursedate}{\today}


% ------------------------------------------------------------
% TITLE BLOCK MACRO
% ------------------------------------------------------------
%
% Usage in exercise files:
%   \maketitleblock
%
\newcommand{\maketitleblock}{%
  \begin{center}
    {\LARGE\bfseries\color{ts-blue} Threat Surfaces}\\[4pt]
    {\large Multivariable Calculus for AI Security}\\[2pt]
    {\small\color{ts-gray} fischer\textsuperscript{3} Education}\\[12pt]
    \rule{\linewidth}{0.4pt}\\[6pt]
    {\large\bfseries \coursemodule: \coursetitle}\\[4pt]
    {\small\color{ts-gray} Exercises \textbar\ \coursedate}\\[4pt]
    \rule{\linewidth}{0.4pt}
  \end{center}
  \vspace{12pt}
}


% ============================================================
% END OF PREAMBLE
% ============================================================

%
%   Adjust the relative path as needed for the file's location
%   within the repository.
% ============================================================


% ------------------------------------------------------------
% PACKAGES
% ------------------------------------------------------------

% Page layout
\usepackage[
  margin=1.25in,
  top=1.0in,
  bottom=1.0in,
  headheight=14pt
]{geometry}

% Typography & encoding
\usepackage[T1]{fontenc}
\usepackage[utf8]{inputenc}
\usepackage{microtype}                % Improved justification
\usepackage{lmodern}                  % Scalable CM fonts

% Mathematics
\usepackage{amsmath}                  % Core math environments
\usepackage{amssymb}                  % Extended symbol set
\usepackage{amsthm}                   % Theorem environments
\usepackage{mathtools}                % Extensions to amsmath
\usepackage{bm}                       % Bold math (\bm)
\usepackage{cancel}                   % Strikethrough in math

% Graphics & color
\usepackage{graphicx}
\usepackage{xcolor}

% Course color palette
\definecolor{ts-blue}{RGB}{30, 100, 180}      % Primary — headings, links
\definecolor{ts-amber}{RGB}{200, 120, 20}     % Accent — warnings, highlights
\definecolor{ts-green}{RGB}{30, 140, 80}      % Statistical bridges
\definecolor{ts-gray}{RGB}{100, 100, 110}     % Secondary text
\definecolor{ts-light}{RGB}{245, 247, 250}    % Background tints

% Hyperlinks
\usepackage{hyperref}
\hypersetup{
  colorlinks=true,
  linkcolor=ts-blue,
  citecolor=ts-blue,
  urlcolor=ts-blue,
  pdfauthor={fischer\textsuperscript{3} Education},
  pdfcreator={LaTeX}
}

% Headers & footers
\usepackage{fancyhdr}
\pagestyle{fancy}
\fancyhf{}
\renewcommand{\headrulewidth}{0.4pt}
\fancyhead[L]{\small\color{ts-gray}\textit{Threat Surfaces}}
\fancyhead[R]{\small\color{ts-gray}\textit{\coursemodule}}   % Set \coursemodule in each file
\fancyfoot[C]{\small\color{ts-gray}\thepage}

% Tables
\usepackage{booktabs}
\usepackage{array}

% Lists
\usepackage{enumitem}
\setlist[enumerate,1]{label=\textbf{\arabic*.}}
\setlist[enumerate,2]{label=(\alph*)}

% Boxes & frames
\usepackage{tcolorbox}
\tcbuselibrary{breakable, skins, theorems}

% Code listings (for Python lab cross-references)
\usepackage{listings}
\lstset{
  basicstyle=\ttfamily\small,
  keywordstyle=\color{ts-blue},
  commentstyle=\color{ts-gray},
  stringstyle=\color{ts-green},
  numbers=left,
  numberstyle=\tiny\color{ts-gray},
  frame=single,
  breaklines=true,
  backgroundcolor=\color{ts-light}
}


% ------------------------------------------------------------
% THEOREM ENVIRONMENTS
% ------------------------------------------------------------

\theoremstyle{definition}
\newtheorem{definition}{Definition}[section]
\newtheorem{example}{Example}[section]
\newtheorem{exercise}{Exercise}[section]

\theoremstyle{plain}
\newtheorem{theorem}{Theorem}[section]
\newtheorem{lemma}[theorem]{Lemma}
\newtheorem{proposition}[theorem]{Proposition}
\newtheorem{corollary}[theorem]{Corollary}

\theoremstyle{remark}
\newtheorem*{remark}{Remark}
\newtheorem*{note}{Note}


% ------------------------------------------------------------
% TCOLORBOX ENVIRONMENTS
% ------------------------------------------------------------

% Problem statement box
\newtcolorbox{problembox}[1][]{
  enhanced,
  breakable,
  colback=ts-light,
  colframe=ts-blue,
  title={Problem},
  fonttitle=\bfseries,
  #1
}

% Statistical bridge box (green accent)
\newtcolorbox{bridgebox}[1][]{
  enhanced,
  breakable,
  colback=ts-light,
  colframe=ts-green,
  title={Statistical Bridge},
  fonttitle=\bfseries\color{ts-green},
  #1
}

% Warning / watch out box (amber accent)
\newtcolorbox{warningbox}[1][]{
  enhanced,
  breakable,
  colback=ts-light,
  colframe=ts-amber,
  title={Watch Out},
  fonttitle=\bfseries\color{ts-amber},
  #1
}

% Hint box (for exercise files — only appears in solution sets)
\newtcolorbox{hintbox}[1][]{
  enhanced,
  breakable,
  colback=ts-light,
  colframe=ts-gray,
  title={Hint},
  fonttitle=\bfseries\color{ts-gray},
  #1
}


% ------------------------------------------------------------
% NOTATION MACROS
% ------------------------------------------------------------

% --- Number systems ---
\newcommand{\R}{\mathbb{R}}             % Real numbers
\newcommand{\N}{\mathbb{N}}             % Natural numbers
\newcommand{\Z}{\mathbb{Z}}             % Integers
\newcommand{\Q}{\mathbb{Q}}             % Rationals
\newcommand{\C}{\mathbb{C}}             % Complex numbers

% --- Vectors & matrices ---
% Convention: bold lowercase for vectors, bold uppercase for matrices
\newcommand{\va}{\bm{a}}
\newcommand{\vb}{\bm{b}}
\newcommand{\vc}{\bm{c}}
\newcommand{\vd}{\bm{d}}
\newcommand{\ve}{\bm{e}}
\newcommand{\vf}{\bm{f}}
\newcommand{\vg}{\bm{g}}
\newcommand{\vh}{\bm{h}}
\newcommand{\vi}{\bm{i}}               % Unit vector i
\newcommand{\vj}{\bm{j}}               % Unit vector j
\newcommand{\vk}{\bm{k}}               % Unit vector k
\newcommand{\vn}{\bm{n}}               % Normal vector
\newcommand{\vr}{\bm{r}}               % Position vector
\newcommand{\vu}{\bm{u}}               % Unit direction vector
\newcommand{\vv}{\bm{v}}
\newcommand{\vw}{\bm{w}}
\newcommand{\vx}{\bm{x}}               % Input vector
\newcommand{\vy}{\bm{y}}               % Output / observation vector
\newcommand{\vz}{\bm{z}}               % Latent vector
\newcommand{\vzero}{\bm{0}}            % Zero vector

% Statistical vectors
\newcommand{\vmu}{\bm{\mu}}            % Mean vector
\newcommand{\vbeta}{\bm{\beta}}        % Regression coefficients
\newcommand{\vtheta}{\bm{\theta}}      % Parameter vector
\newcommand{\veps}{\bm{\varepsilon}}   % Error vector

% Matrices
\newcommand{\mA}{\bm{A}}
\newcommand{\mB}{\bm{B}}
\newcommand{\mC}{\bm{C}}
\newcommand{\mH}{\bm{H}}               % Hessian (also \Hess below)
\newcommand{\mI}{\bm{I}}               % Identity matrix
\newcommand{\mJ}{\bm{J}}               % Jacobian (also \Jac below)
\newcommand{\mK}{\bm{K}}               % Kernel matrix
\newcommand{\mL}{\bm{L}}               % Lower triangular / Cholesky
\newcommand{\mQ}{\bm{Q}}               % Orthogonal matrix
\newcommand{\mR}{\bm{R}}               % Upper triangular
\newcommand{\mS}{\bm{S}}               % Sample covariance
\newcommand{\mSigma}{\bm{\Sigma}}      % Covariance matrix
\newcommand{\mX}{\bm{X}}               % Design matrix
\newcommand{\mY}{\bm{Y}}

% --- Calculus operators ---
\newcommand{\grad}{\nabla}                              % Gradient ∇
\newcommand{\Grad}[1]{\nabla #1}                        % Gradient of #1
\newcommand{\GradP}[2]{\nabla_{#1} #2}                 % Gradient of #2 w.r.t. #1
\newcommand{\Div}[1]{\nabla \cdot #1}                  % Divergence
\newcommand{\Curl}[1]{\nabla \times #1}                % Curl
\newcommand{\Lap}[1]{\nabla^2 #1}                      % Laplacian
\newcommand{\Hess}[1]{\bm{H}_{#1}}                     % Hessian of #1
\newcommand{\Jac}[1]{\bm{J}_{#1}}                      % Jacobian of #1

% Partial derivative shorthand
\newcommand{\pd}[2]{\frac{\partial #1}{\partial #2}}           % ∂f/∂x
\newcommand{\pdd}[2]{\frac{\partial^2 #1}{\partial #2^2}}      % ∂²f/∂x²
\newcommand{\pdm}[3]{\frac{\partial^2 #1}{\partial #2\,\partial #3}}  % ∂²f/∂x∂y

% Ordinary derivative shorthand
\newcommand{\od}[2]{\frac{d #1}{d #2}}                         % df/dx
\newcommand{\odd}[2]{\frac{d^2 #1}{d #2^2}}                    % d²f/dx²

% Integrals
\newcommand{\dx}{\,dx}
\newcommand{\dy}{\,dy}
\newcommand{\dz}{\,dz}
\newcommand{\dt}{\,dt}
\newcommand{\dA}{\,dA}
\newcommand{\dV}{\,dV}
\newcommand{\ds}{\,ds}
\newcommand{\dvx}{\,d\vx}              % Vector integral element

% --- Statistical notation ---
\newcommand{\E}[1]{\mathbb{E}\!\left[#1\right]}          % Expectation
\newcommand{\Var}[1]{\operatorname{Var}\!\left(#1\right)} % Variance
\newcommand{\Cov}[2]{\operatorname{Cov}\!\left(#1,\,#2\right)} % Covariance
\newcommand{\Cor}[2]{\operatorname{Cor}\!\left(#1,\,#2\right)} % Correlation

\newcommand{\given}{\,\vert\,}                           % Conditional bar
\newcommand{\iid}{\overset{\text{iid}}{\sim}}           % iid ~
\newcommand{\ind}{\overset{\text{ind}}{\sim}}           % ind ~

% Distributions
\newcommand{\Normal}{\mathcal{N}}                        % Normal distribution
\newcommand{\MVN}[2]{\mathcal{N}\!\left(#1,\,#2\right)} % Multivariate normal
\newcommand{\Bern}{\operatorname{Bernoulli}}
\newcommand{\Bin}{\operatorname{Binomial}}
\newcommand{\Pois}{\operatorname{Poisson}}
\newcommand{\Unif}{\operatorname{Uniform}}
\newcommand{\Gam}{\operatorname{Gamma}}
\newcommand{\Beta}{\operatorname{Beta}}

% Likelihood & inference
\newcommand{\Lik}{\mathcal{L}}                          % Likelihood
\newcommand{\loglik}{\ell}                              % Log-likelihood
\newcommand{\Score}[1]{s(#1)}                           % Score function
\newcommand{\FI}{\mathcal{I}}                           % Fisher information
\newcommand{\MLE}[1]{\hat{#1}_{\text{MLE}}}            % MLE estimator
\newcommand{\MAP}[1]{\hat{#1}_{\text{MAP}}}            % MAP estimator
\newcommand{\posterior}{p(\vtheta \given \vx)}          % Posterior shorthand
\newcommand{\prior}{p(\vtheta)}                         % Prior shorthand
\newcommand{\likelihood}{p(\vx \given \vtheta)}        % Likelihood shorthand
\newcommand{\KL}[2]{D_{\text{KL}}\!\left(#1\,\|\,#2\right)} % KL divergence

% --- Linear algebra ---
\newcommand{\tr}[1]{\operatorname{tr}\!\left(#1\right)}        % Trace
\newcommand{\rank}[1]{\operatorname{rank}\!\left(#1\right)}    % Rank
\newcommand{\diag}[1]{\operatorname{diag}\!\left(#1\right)}    % Diagonal matrix
\newcommand{\spn}[1]{\operatorname{span}\!\left\{#1\right\}}   % Span
\newcommand{\norm}[1]{\left\|#1\right\|}                       % Norm
\newcommand{\abs}[1]{\left|#1\right|}                          % Absolute value
\newcommand{\inner}[2]{\langle #1,\, #2 \rangle}               % Inner product
\newcommand{\T}{^{\top}}                                        % Transpose superscript

% Positive (semi-)definiteness shorthand
\newcommand{\psd}{\succeq 0}    % Positive semi-definite
\newcommand{\pd}{\succ 0}       % Positive definite (Note: redefines \pd — see below)
% Note: \pd is overloaded: partial derivative above, positive definite here.
% In ambiguous contexts, use \succ 0 directly for positive definite.

% --- Miscellaneous ---
\newcommand{\defeq}{\coloneqq}                          % Definition equality :=
\newcommand{\st}{\text{ s.t. }}                         % "such that"
\newcommand{\approxeq}{\approx}                         % Approximate equality
\newcommand{\BigO}[1]{\mathcal{O}\!\left(#1\right)}    % Big-O notation
\newcommand{\sigmoid}[1]{\sigma\!\left(#1\right)}      % Sigmoid function
\newcommand{\softmax}[1]{\operatorname{softmax}\!\left(#1\right)}


% ------------------------------------------------------------
% DOCUMENT METADATA COMMANDS
% (Set these in each individual exercise file)
% ------------------------------------------------------------
%
%   \newcommand{\coursemodule}{Module 02, Unit 03}
%   \newcommand{\coursetitle}{The Multivariable Chain Rule}
%   \newcommand{\coursedate}{February 2026}
%
% These are used in the header/footer and title block.
% Define them BEFORE % ============================================================
% latex-preamble.tex — Threat Surfaces: Multivariable Calculus for AI Security
% fischer³ Education
%
% PURPOSE:
%   Shared preamble included by all exercise and solution files.
%   Defines packages, page layout, theorem environments, and
%   course-specific notation macros.
%
% USAGE:
%   In any exercise or solution .tex file, place the following
%   at the top of the document before \begin{document}:
%
%     % ============================================================
% latex-preamble.tex — Threat Surfaces: Multivariable Calculus for AI Security
% fischer³ Education
%
% PURPOSE:
%   Shared preamble included by all exercise and solution files.
%   Defines packages, page layout, theorem environments, and
%   course-specific notation macros.
%
% USAGE:
%   In any exercise or solution .tex file, place the following
%   at the top of the document before \begin{document}:
%
%     \input{../../assets/latex-preamble.tex}
%
%   Adjust the relative path as needed for the file's location
%   within the repository.
% ============================================================


% ------------------------------------------------------------
% PACKAGES
% ------------------------------------------------------------

% Page layout
\usepackage[
  margin=1.25in,
  top=1.0in,
  bottom=1.0in,
  headheight=14pt
]{geometry}

% Typography & encoding
\usepackage[T1]{fontenc}
\usepackage[utf8]{inputenc}
\usepackage{microtype}                % Improved justification
\usepackage{lmodern}                  % Scalable CM fonts

% Mathematics
\usepackage{amsmath}                  % Core math environments
\usepackage{amssymb}                  % Extended symbol set
\usepackage{amsthm}                   % Theorem environments
\usepackage{mathtools}                % Extensions to amsmath
\usepackage{bm}                       % Bold math (\bm)
\usepackage{cancel}                   % Strikethrough in math

% Graphics & color
\usepackage{graphicx}
\usepackage{xcolor}

% Course color palette
\definecolor{ts-blue}{RGB}{30, 100, 180}      % Primary — headings, links
\definecolor{ts-amber}{RGB}{200, 120, 20}     % Accent — warnings, highlights
\definecolor{ts-green}{RGB}{30, 140, 80}      % Statistical bridges
\definecolor{ts-gray}{RGB}{100, 100, 110}     % Secondary text
\definecolor{ts-light}{RGB}{245, 247, 250}    % Background tints

% Hyperlinks
\usepackage{hyperref}
\hypersetup{
  colorlinks=true,
  linkcolor=ts-blue,
  citecolor=ts-blue,
  urlcolor=ts-blue,
  pdfauthor={fischer\textsuperscript{3} Education},
  pdfcreator={LaTeX}
}

% Headers & footers
\usepackage{fancyhdr}
\pagestyle{fancy}
\fancyhf{}
\renewcommand{\headrulewidth}{0.4pt}
\fancyhead[L]{\small\color{ts-gray}\textit{Threat Surfaces}}
\fancyhead[R]{\small\color{ts-gray}\textit{\coursemodule}}   % Set \coursemodule in each file
\fancyfoot[C]{\small\color{ts-gray}\thepage}

% Tables
\usepackage{booktabs}
\usepackage{array}

% Lists
\usepackage{enumitem}
\setlist[enumerate,1]{label=\textbf{\arabic*.}}
\setlist[enumerate,2]{label=(\alph*)}

% Boxes & frames
\usepackage{tcolorbox}
\tcbuselibrary{breakable, skins, theorems}

% Code listings (for Python lab cross-references)
\usepackage{listings}
\lstset{
  basicstyle=\ttfamily\small,
  keywordstyle=\color{ts-blue},
  commentstyle=\color{ts-gray},
  stringstyle=\color{ts-green},
  numbers=left,
  numberstyle=\tiny\color{ts-gray},
  frame=single,
  breaklines=true,
  backgroundcolor=\color{ts-light}
}


% ------------------------------------------------------------
% THEOREM ENVIRONMENTS
% ------------------------------------------------------------

\theoremstyle{definition}
\newtheorem{definition}{Definition}[section]
\newtheorem{example}{Example}[section]
\newtheorem{exercise}{Exercise}[section]

\theoremstyle{plain}
\newtheorem{theorem}{Theorem}[section]
\newtheorem{lemma}[theorem]{Lemma}
\newtheorem{proposition}[theorem]{Proposition}
\newtheorem{corollary}[theorem]{Corollary}

\theoremstyle{remark}
\newtheorem*{remark}{Remark}
\newtheorem*{note}{Note}


% ------------------------------------------------------------
% TCOLORBOX ENVIRONMENTS
% ------------------------------------------------------------

% Problem statement box
\newtcolorbox{problembox}[1][]{
  enhanced,
  breakable,
  colback=ts-light,
  colframe=ts-blue,
  title={Problem},
  fonttitle=\bfseries,
  #1
}

% Statistical bridge box (green accent)
\newtcolorbox{bridgebox}[1][]{
  enhanced,
  breakable,
  colback=ts-light,
  colframe=ts-green,
  title={Statistical Bridge},
  fonttitle=\bfseries\color{ts-green},
  #1
}

% Warning / watch out box (amber accent)
\newtcolorbox{warningbox}[1][]{
  enhanced,
  breakable,
  colback=ts-light,
  colframe=ts-amber,
  title={Watch Out},
  fonttitle=\bfseries\color{ts-amber},
  #1
}

% Hint box (for exercise files — only appears in solution sets)
\newtcolorbox{hintbox}[1][]{
  enhanced,
  breakable,
  colback=ts-light,
  colframe=ts-gray,
  title={Hint},
  fonttitle=\bfseries\color{ts-gray},
  #1
}


% ------------------------------------------------------------
% NOTATION MACROS
% ------------------------------------------------------------

% --- Number systems ---
\newcommand{\R}{\mathbb{R}}             % Real numbers
\newcommand{\N}{\mathbb{N}}             % Natural numbers
\newcommand{\Z}{\mathbb{Z}}             % Integers
\newcommand{\Q}{\mathbb{Q}}             % Rationals
\newcommand{\C}{\mathbb{C}}             % Complex numbers

% --- Vectors & matrices ---
% Convention: bold lowercase for vectors, bold uppercase for matrices
\newcommand{\va}{\bm{a}}
\newcommand{\vb}{\bm{b}}
\newcommand{\vc}{\bm{c}}
\newcommand{\vd}{\bm{d}}
\newcommand{\ve}{\bm{e}}
\newcommand{\vf}{\bm{f}}
\newcommand{\vg}{\bm{g}}
\newcommand{\vh}{\bm{h}}
\newcommand{\vi}{\bm{i}}               % Unit vector i
\newcommand{\vj}{\bm{j}}               % Unit vector j
\newcommand{\vk}{\bm{k}}               % Unit vector k
\newcommand{\vn}{\bm{n}}               % Normal vector
\newcommand{\vr}{\bm{r}}               % Position vector
\newcommand{\vu}{\bm{u}}               % Unit direction vector
\newcommand{\vv}{\bm{v}}
\newcommand{\vw}{\bm{w}}
\newcommand{\vx}{\bm{x}}               % Input vector
\newcommand{\vy}{\bm{y}}               % Output / observation vector
\newcommand{\vz}{\bm{z}}               % Latent vector
\newcommand{\vzero}{\bm{0}}            % Zero vector

% Statistical vectors
\newcommand{\vmu}{\bm{\mu}}            % Mean vector
\newcommand{\vbeta}{\bm{\beta}}        % Regression coefficients
\newcommand{\vtheta}{\bm{\theta}}      % Parameter vector
\newcommand{\veps}{\bm{\varepsilon}}   % Error vector

% Matrices
\newcommand{\mA}{\bm{A}}
\newcommand{\mB}{\bm{B}}
\newcommand{\mC}{\bm{C}}
\newcommand{\mH}{\bm{H}}               % Hessian (also \Hess below)
\newcommand{\mI}{\bm{I}}               % Identity matrix
\newcommand{\mJ}{\bm{J}}               % Jacobian (also \Jac below)
\newcommand{\mK}{\bm{K}}               % Kernel matrix
\newcommand{\mL}{\bm{L}}               % Lower triangular / Cholesky
\newcommand{\mQ}{\bm{Q}}               % Orthogonal matrix
\newcommand{\mR}{\bm{R}}               % Upper triangular
\newcommand{\mS}{\bm{S}}               % Sample covariance
\newcommand{\mSigma}{\bm{\Sigma}}      % Covariance matrix
\newcommand{\mX}{\bm{X}}               % Design matrix
\newcommand{\mY}{\bm{Y}}

% --- Calculus operators ---
\newcommand{\grad}{\nabla}                              % Gradient ∇
\newcommand{\Grad}[1]{\nabla #1}                        % Gradient of #1
\newcommand{\GradP}[2]{\nabla_{#1} #2}                 % Gradient of #2 w.r.t. #1
\newcommand{\Div}[1]{\nabla \cdot #1}                  % Divergence
\newcommand{\Curl}[1]{\nabla \times #1}                % Curl
\newcommand{\Lap}[1]{\nabla^2 #1}                      % Laplacian
\newcommand{\Hess}[1]{\bm{H}_{#1}}                     % Hessian of #1
\newcommand{\Jac}[1]{\bm{J}_{#1}}                      % Jacobian of #1

% Partial derivative shorthand
\newcommand{\pd}[2]{\frac{\partial #1}{\partial #2}}           % ∂f/∂x
\newcommand{\pdd}[2]{\frac{\partial^2 #1}{\partial #2^2}}      % ∂²f/∂x²
\newcommand{\pdm}[3]{\frac{\partial^2 #1}{\partial #2\,\partial #3}}  % ∂²f/∂x∂y

% Ordinary derivative shorthand
\newcommand{\od}[2]{\frac{d #1}{d #2}}                         % df/dx
\newcommand{\odd}[2]{\frac{d^2 #1}{d #2^2}}                    % d²f/dx²

% Integrals
\newcommand{\dx}{\,dx}
\newcommand{\dy}{\,dy}
\newcommand{\dz}{\,dz}
\newcommand{\dt}{\,dt}
\newcommand{\dA}{\,dA}
\newcommand{\dV}{\,dV}
\newcommand{\ds}{\,ds}
\newcommand{\dvx}{\,d\vx}              % Vector integral element

% --- Statistical notation ---
\newcommand{\E}[1]{\mathbb{E}\!\left[#1\right]}          % Expectation
\newcommand{\Var}[1]{\operatorname{Var}\!\left(#1\right)} % Variance
\newcommand{\Cov}[2]{\operatorname{Cov}\!\left(#1,\,#2\right)} % Covariance
\newcommand{\Cor}[2]{\operatorname{Cor}\!\left(#1,\,#2\right)} % Correlation

\newcommand{\given}{\,\vert\,}                           % Conditional bar
\newcommand{\iid}{\overset{\text{iid}}{\sim}}           % iid ~
\newcommand{\ind}{\overset{\text{ind}}{\sim}}           % ind ~

% Distributions
\newcommand{\Normal}{\mathcal{N}}                        % Normal distribution
\newcommand{\MVN}[2]{\mathcal{N}\!\left(#1,\,#2\right)} % Multivariate normal
\newcommand{\Bern}{\operatorname{Bernoulli}}
\newcommand{\Bin}{\operatorname{Binomial}}
\newcommand{\Pois}{\operatorname{Poisson}}
\newcommand{\Unif}{\operatorname{Uniform}}
\newcommand{\Gam}{\operatorname{Gamma}}
\newcommand{\Beta}{\operatorname{Beta}}

% Likelihood & inference
\newcommand{\Lik}{\mathcal{L}}                          % Likelihood
\newcommand{\loglik}{\ell}                              % Log-likelihood
\newcommand{\Score}[1]{s(#1)}                           % Score function
\newcommand{\FI}{\mathcal{I}}                           % Fisher information
\newcommand{\MLE}[1]{\hat{#1}_{\text{MLE}}}            % MLE estimator
\newcommand{\MAP}[1]{\hat{#1}_{\text{MAP}}}            % MAP estimator
\newcommand{\posterior}{p(\vtheta \given \vx)}          % Posterior shorthand
\newcommand{\prior}{p(\vtheta)}                         % Prior shorthand
\newcommand{\likelihood}{p(\vx \given \vtheta)}        % Likelihood shorthand
\newcommand{\KL}[2]{D_{\text{KL}}\!\left(#1\,\|\,#2\right)} % KL divergence

% --- Linear algebra ---
\newcommand{\tr}[1]{\operatorname{tr}\!\left(#1\right)}        % Trace
\newcommand{\rank}[1]{\operatorname{rank}\!\left(#1\right)}    % Rank
\newcommand{\diag}[1]{\operatorname{diag}\!\left(#1\right)}    % Diagonal matrix
\newcommand{\spn}[1]{\operatorname{span}\!\left\{#1\right\}}   % Span
\newcommand{\norm}[1]{\left\|#1\right\|}                       % Norm
\newcommand{\abs}[1]{\left|#1\right|}                          % Absolute value
\newcommand{\inner}[2]{\langle #1,\, #2 \rangle}               % Inner product
\newcommand{\T}{^{\top}}                                        % Transpose superscript

% Positive (semi-)definiteness shorthand
\newcommand{\psd}{\succeq 0}    % Positive semi-definite
\newcommand{\pd}{\succ 0}       % Positive definite (Note: redefines \pd — see below)
% Note: \pd is overloaded: partial derivative above, positive definite here.
% In ambiguous contexts, use \succ 0 directly for positive definite.

% --- Miscellaneous ---
\newcommand{\defeq}{\coloneqq}                          % Definition equality :=
\newcommand{\st}{\text{ s.t. }}                         % "such that"
\newcommand{\approxeq}{\approx}                         % Approximate equality
\newcommand{\BigO}[1]{\mathcal{O}\!\left(#1\right)}    % Big-O notation
\newcommand{\sigmoid}[1]{\sigma\!\left(#1\right)}      % Sigmoid function
\newcommand{\softmax}[1]{\operatorname{softmax}\!\left(#1\right)}


% ------------------------------------------------------------
% DOCUMENT METADATA COMMANDS
% (Set these in each individual exercise file)
% ------------------------------------------------------------
%
%   \newcommand{\coursemodule}{Module 02, Unit 03}
%   \newcommand{\coursetitle}{The Multivariable Chain Rule}
%   \newcommand{\coursedate}{February 2026}
%
% These are used in the header/footer and title block.
% Define them BEFORE \input{...latex-preamble.tex} or immediately after.
%
% Default definitions (overridden per-file):
\providecommand{\coursemodule}{[Module TBD]}
\providecommand{\coursetitle}{[Unit Title TBD]}
\providecommand{\coursedate}{\today}


% ------------------------------------------------------------
% TITLE BLOCK MACRO
% ------------------------------------------------------------
%
% Usage in exercise files:
%   \maketitleblock
%
\newcommand{\maketitleblock}{%
  \begin{center}
    {\LARGE\bfseries\color{ts-blue} Threat Surfaces}\\[4pt]
    {\large Multivariable Calculus for AI Security}\\[2pt]
    {\small\color{ts-gray} fischer\textsuperscript{3} Education}\\[12pt]
    \rule{\linewidth}{0.4pt}\\[6pt]
    {\large\bfseries \coursemodule: \coursetitle}\\[4pt]
    {\small\color{ts-gray} Exercises \textbar\ \coursedate}\\[4pt]
    \rule{\linewidth}{0.4pt}
  \end{center}
  \vspace{12pt}
}


% ============================================================
% END OF PREAMBLE
% ============================================================

%
%   Adjust the relative path as needed for the file's location
%   within the repository.
% ============================================================


% ------------------------------------------------------------
% PACKAGES
% ------------------------------------------------------------

% Page layout
\usepackage[
  margin=1.25in,
  top=1.0in,
  bottom=1.0in,
  headheight=14pt
]{geometry}

% Typography & encoding
\usepackage[T1]{fontenc}
\usepackage[utf8]{inputenc}
\usepackage{microtype}                % Improved justification
\usepackage{lmodern}                  % Scalable CM fonts

% Mathematics
\usepackage{amsmath}                  % Core math environments
\usepackage{amssymb}                  % Extended symbol set
\usepackage{amsthm}                   % Theorem environments
\usepackage{mathtools}                % Extensions to amsmath
\usepackage{bm}                       % Bold math (\bm)
\usepackage{cancel}                   % Strikethrough in math

% Graphics & color
\usepackage{graphicx}
\usepackage{xcolor}

% Course color palette
\definecolor{ts-blue}{RGB}{30, 100, 180}      % Primary — headings, links
\definecolor{ts-amber}{RGB}{200, 120, 20}     % Accent — warnings, highlights
\definecolor{ts-green}{RGB}{30, 140, 80}      % Statistical bridges
\definecolor{ts-gray}{RGB}{100, 100, 110}     % Secondary text
\definecolor{ts-light}{RGB}{245, 247, 250}    % Background tints

% Hyperlinks
\usepackage{hyperref}
\hypersetup{
  colorlinks=true,
  linkcolor=ts-blue,
  citecolor=ts-blue,
  urlcolor=ts-blue,
  pdfauthor={fischer\textsuperscript{3} Education},
  pdfcreator={LaTeX}
}

% Headers & footers
\usepackage{fancyhdr}
\pagestyle{fancy}
\fancyhf{}
\renewcommand{\headrulewidth}{0.4pt}
\fancyhead[L]{\small\color{ts-gray}\textit{Threat Surfaces}}
\fancyhead[R]{\small\color{ts-gray}\textit{\coursemodule}}   % Set \coursemodule in each file
\fancyfoot[C]{\small\color{ts-gray}\thepage}

% Tables
\usepackage{booktabs}
\usepackage{array}

% Lists
\usepackage{enumitem}
\setlist[enumerate,1]{label=\textbf{\arabic*.}}
\setlist[enumerate,2]{label=(\alph*)}

% Boxes & frames
\usepackage{tcolorbox}
\tcbuselibrary{breakable, skins, theorems}

% Code listings (for Python lab cross-references)
\usepackage{listings}
\lstset{
  basicstyle=\ttfamily\small,
  keywordstyle=\color{ts-blue},
  commentstyle=\color{ts-gray},
  stringstyle=\color{ts-green},
  numbers=left,
  numberstyle=\tiny\color{ts-gray},
  frame=single,
  breaklines=true,
  backgroundcolor=\color{ts-light}
}


% ------------------------------------------------------------
% THEOREM ENVIRONMENTS
% ------------------------------------------------------------

\theoremstyle{definition}
\newtheorem{definition}{Definition}[section]
\newtheorem{example}{Example}[section]
\newtheorem{exercise}{Exercise}[section]

\theoremstyle{plain}
\newtheorem{theorem}{Theorem}[section]
\newtheorem{lemma}[theorem]{Lemma}
\newtheorem{proposition}[theorem]{Proposition}
\newtheorem{corollary}[theorem]{Corollary}

\theoremstyle{remark}
\newtheorem*{remark}{Remark}
\newtheorem*{note}{Note}


% ------------------------------------------------------------
% TCOLORBOX ENVIRONMENTS
% ------------------------------------------------------------

% Problem statement box
\newtcolorbox{problembox}[1][]{
  enhanced,
  breakable,
  colback=ts-light,
  colframe=ts-blue,
  title={Problem},
  fonttitle=\bfseries,
  #1
}

% Statistical bridge box (green accent)
\newtcolorbox{bridgebox}[1][]{
  enhanced,
  breakable,
  colback=ts-light,
  colframe=ts-green,
  title={Statistical Bridge},
  fonttitle=\bfseries\color{ts-green},
  #1
}

% Warning / watch out box (amber accent)
\newtcolorbox{warningbox}[1][]{
  enhanced,
  breakable,
  colback=ts-light,
  colframe=ts-amber,
  title={Watch Out},
  fonttitle=\bfseries\color{ts-amber},
  #1
}

% Hint box (for exercise files — only appears in solution sets)
\newtcolorbox{hintbox}[1][]{
  enhanced,
  breakable,
  colback=ts-light,
  colframe=ts-gray,
  title={Hint},
  fonttitle=\bfseries\color{ts-gray},
  #1
}


% ------------------------------------------------------------
% NOTATION MACROS
% ------------------------------------------------------------

% --- Number systems ---
\newcommand{\R}{\mathbb{R}}             % Real numbers
\newcommand{\N}{\mathbb{N}}             % Natural numbers
\newcommand{\Z}{\mathbb{Z}}             % Integers
\newcommand{\Q}{\mathbb{Q}}             % Rationals
\newcommand{\C}{\mathbb{C}}             % Complex numbers

% --- Vectors & matrices ---
% Convention: bold lowercase for vectors, bold uppercase for matrices
\newcommand{\va}{\bm{a}}
\newcommand{\vb}{\bm{b}}
\newcommand{\vc}{\bm{c}}
\newcommand{\vd}{\bm{d}}
\newcommand{\ve}{\bm{e}}
\newcommand{\vf}{\bm{f}}
\newcommand{\vg}{\bm{g}}
\newcommand{\vh}{\bm{h}}
\newcommand{\vi}{\bm{i}}               % Unit vector i
\newcommand{\vj}{\bm{j}}               % Unit vector j
\newcommand{\vk}{\bm{k}}               % Unit vector k
\newcommand{\vn}{\bm{n}}               % Normal vector
\newcommand{\vr}{\bm{r}}               % Position vector
\newcommand{\vu}{\bm{u}}               % Unit direction vector
\newcommand{\vv}{\bm{v}}
\newcommand{\vw}{\bm{w}}
\newcommand{\vx}{\bm{x}}               % Input vector
\newcommand{\vy}{\bm{y}}               % Output / observation vector
\newcommand{\vz}{\bm{z}}               % Latent vector
\newcommand{\vzero}{\bm{0}}            % Zero vector

% Statistical vectors
\newcommand{\vmu}{\bm{\mu}}            % Mean vector
\newcommand{\vbeta}{\bm{\beta}}        % Regression coefficients
\newcommand{\vtheta}{\bm{\theta}}      % Parameter vector
\newcommand{\veps}{\bm{\varepsilon}}   % Error vector

% Matrices
\newcommand{\mA}{\bm{A}}
\newcommand{\mB}{\bm{B}}
\newcommand{\mC}{\bm{C}}
\newcommand{\mH}{\bm{H}}               % Hessian (also \Hess below)
\newcommand{\mI}{\bm{I}}               % Identity matrix
\newcommand{\mJ}{\bm{J}}               % Jacobian (also \Jac below)
\newcommand{\mK}{\bm{K}}               % Kernel matrix
\newcommand{\mL}{\bm{L}}               % Lower triangular / Cholesky
\newcommand{\mQ}{\bm{Q}}               % Orthogonal matrix
\newcommand{\mR}{\bm{R}}               % Upper triangular
\newcommand{\mS}{\bm{S}}               % Sample covariance
\newcommand{\mSigma}{\bm{\Sigma}}      % Covariance matrix
\newcommand{\mX}{\bm{X}}               % Design matrix
\newcommand{\mY}{\bm{Y}}

% --- Calculus operators ---
\newcommand{\grad}{\nabla}                              % Gradient ∇
\newcommand{\Grad}[1]{\nabla #1}                        % Gradient of #1
\newcommand{\GradP}[2]{\nabla_{#1} #2}                 % Gradient of #2 w.r.t. #1
\newcommand{\Div}[1]{\nabla \cdot #1}                  % Divergence
\newcommand{\Curl}[1]{\nabla \times #1}                % Curl
\newcommand{\Lap}[1]{\nabla^2 #1}                      % Laplacian
\newcommand{\Hess}[1]{\bm{H}_{#1}}                     % Hessian of #1
\newcommand{\Jac}[1]{\bm{J}_{#1}}                      % Jacobian of #1

% Partial derivative shorthand
\newcommand{\pd}[2]{\frac{\partial #1}{\partial #2}}           % ∂f/∂x
\newcommand{\pdd}[2]{\frac{\partial^2 #1}{\partial #2^2}}      % ∂²f/∂x²
\newcommand{\pdm}[3]{\frac{\partial^2 #1}{\partial #2\,\partial #3}}  % ∂²f/∂x∂y

% Ordinary derivative shorthand
\newcommand{\od}[2]{\frac{d #1}{d #2}}                         % df/dx
\newcommand{\odd}[2]{\frac{d^2 #1}{d #2^2}}                    % d²f/dx²

% Integrals
\newcommand{\dx}{\,dx}
\newcommand{\dy}{\,dy}
\newcommand{\dz}{\,dz}
\newcommand{\dt}{\,dt}
\newcommand{\dA}{\,dA}
\newcommand{\dV}{\,dV}
\newcommand{\ds}{\,ds}
\newcommand{\dvx}{\,d\vx}              % Vector integral element

% --- Statistical notation ---
\newcommand{\E}[1]{\mathbb{E}\!\left[#1\right]}          % Expectation
\newcommand{\Var}[1]{\operatorname{Var}\!\left(#1\right)} % Variance
\newcommand{\Cov}[2]{\operatorname{Cov}\!\left(#1,\,#2\right)} % Covariance
\newcommand{\Cor}[2]{\operatorname{Cor}\!\left(#1,\,#2\right)} % Correlation

\newcommand{\given}{\,\vert\,}                           % Conditional bar
\newcommand{\iid}{\overset{\text{iid}}{\sim}}           % iid ~
\newcommand{\ind}{\overset{\text{ind}}{\sim}}           % ind ~

% Distributions
\newcommand{\Normal}{\mathcal{N}}                        % Normal distribution
\newcommand{\MVN}[2]{\mathcal{N}\!\left(#1,\,#2\right)} % Multivariate normal
\newcommand{\Bern}{\operatorname{Bernoulli}}
\newcommand{\Bin}{\operatorname{Binomial}}
\newcommand{\Pois}{\operatorname{Poisson}}
\newcommand{\Unif}{\operatorname{Uniform}}
\newcommand{\Gam}{\operatorname{Gamma}}
\newcommand{\Beta}{\operatorname{Beta}}

% Likelihood & inference
\newcommand{\Lik}{\mathcal{L}}                          % Likelihood
\newcommand{\loglik}{\ell}                              % Log-likelihood
\newcommand{\Score}[1]{s(#1)}                           % Score function
\newcommand{\FI}{\mathcal{I}}                           % Fisher information
\newcommand{\MLE}[1]{\hat{#1}_{\text{MLE}}}            % MLE estimator
\newcommand{\MAP}[1]{\hat{#1}_{\text{MAP}}}            % MAP estimator
\newcommand{\posterior}{p(\vtheta \given \vx)}          % Posterior shorthand
\newcommand{\prior}{p(\vtheta)}                         % Prior shorthand
\newcommand{\likelihood}{p(\vx \given \vtheta)}        % Likelihood shorthand
\newcommand{\KL}[2]{D_{\text{KL}}\!\left(#1\,\|\,#2\right)} % KL divergence

% --- Linear algebra ---
\newcommand{\tr}[1]{\operatorname{tr}\!\left(#1\right)}        % Trace
\newcommand{\rank}[1]{\operatorname{rank}\!\left(#1\right)}    % Rank
\newcommand{\diag}[1]{\operatorname{diag}\!\left(#1\right)}    % Diagonal matrix
\newcommand{\spn}[1]{\operatorname{span}\!\left\{#1\right\}}   % Span
\newcommand{\norm}[1]{\left\|#1\right\|}                       % Norm
\newcommand{\abs}[1]{\left|#1\right|}                          % Absolute value
\newcommand{\inner}[2]{\langle #1,\, #2 \rangle}               % Inner product
\newcommand{\T}{^{\top}}                                        % Transpose superscript

% Positive (semi-)definiteness shorthand
\newcommand{\psd}{\succeq 0}    % Positive semi-definite
\newcommand{\pd}{\succ 0}       % Positive definite (Note: redefines \pd — see below)
% Note: \pd is overloaded: partial derivative above, positive definite here.
% In ambiguous contexts, use \succ 0 directly for positive definite.

% --- Miscellaneous ---
\newcommand{\defeq}{\coloneqq}                          % Definition equality :=
\newcommand{\st}{\text{ s.t. }}                         % "such that"
\newcommand{\approxeq}{\approx}                         % Approximate equality
\newcommand{\BigO}[1]{\mathcal{O}\!\left(#1\right)}    % Big-O notation
\newcommand{\sigmoid}[1]{\sigma\!\left(#1\right)}      % Sigmoid function
\newcommand{\softmax}[1]{\operatorname{softmax}\!\left(#1\right)}


% ------------------------------------------------------------
% DOCUMENT METADATA COMMANDS
% (Set these in each individual exercise file)
% ------------------------------------------------------------
%
%   \newcommand{\coursemodule}{Module 02, Unit 03}
%   \newcommand{\coursetitle}{The Multivariable Chain Rule}
%   \newcommand{\coursedate}{February 2026}
%
% These are used in the header/footer and title block.
% Define them BEFORE % ============================================================
% latex-preamble.tex — Threat Surfaces: Multivariable Calculus for AI Security
% fischer³ Education
%
% PURPOSE:
%   Shared preamble included by all exercise and solution files.
%   Defines packages, page layout, theorem environments, and
%   course-specific notation macros.
%
% USAGE:
%   In any exercise or solution .tex file, place the following
%   at the top of the document before \begin{document}:
%
%     \input{../../assets/latex-preamble.tex}
%
%   Adjust the relative path as needed for the file's location
%   within the repository.
% ============================================================


% ------------------------------------------------------------
% PACKAGES
% ------------------------------------------------------------

% Page layout
\usepackage[
  margin=1.25in,
  top=1.0in,
  bottom=1.0in,
  headheight=14pt
]{geometry}

% Typography & encoding
\usepackage[T1]{fontenc}
\usepackage[utf8]{inputenc}
\usepackage{microtype}                % Improved justification
\usepackage{lmodern}                  % Scalable CM fonts

% Mathematics
\usepackage{amsmath}                  % Core math environments
\usepackage{amssymb}                  % Extended symbol set
\usepackage{amsthm}                   % Theorem environments
\usepackage{mathtools}                % Extensions to amsmath
\usepackage{bm}                       % Bold math (\bm)
\usepackage{cancel}                   % Strikethrough in math

% Graphics & color
\usepackage{graphicx}
\usepackage{xcolor}

% Course color palette
\definecolor{ts-blue}{RGB}{30, 100, 180}      % Primary — headings, links
\definecolor{ts-amber}{RGB}{200, 120, 20}     % Accent — warnings, highlights
\definecolor{ts-green}{RGB}{30, 140, 80}      % Statistical bridges
\definecolor{ts-gray}{RGB}{100, 100, 110}     % Secondary text
\definecolor{ts-light}{RGB}{245, 247, 250}    % Background tints

% Hyperlinks
\usepackage{hyperref}
\hypersetup{
  colorlinks=true,
  linkcolor=ts-blue,
  citecolor=ts-blue,
  urlcolor=ts-blue,
  pdfauthor={fischer\textsuperscript{3} Education},
  pdfcreator={LaTeX}
}

% Headers & footers
\usepackage{fancyhdr}
\pagestyle{fancy}
\fancyhf{}
\renewcommand{\headrulewidth}{0.4pt}
\fancyhead[L]{\small\color{ts-gray}\textit{Threat Surfaces}}
\fancyhead[R]{\small\color{ts-gray}\textit{\coursemodule}}   % Set \coursemodule in each file
\fancyfoot[C]{\small\color{ts-gray}\thepage}

% Tables
\usepackage{booktabs}
\usepackage{array}

% Lists
\usepackage{enumitem}
\setlist[enumerate,1]{label=\textbf{\arabic*.}}
\setlist[enumerate,2]{label=(\alph*)}

% Boxes & frames
\usepackage{tcolorbox}
\tcbuselibrary{breakable, skins, theorems}

% Code listings (for Python lab cross-references)
\usepackage{listings}
\lstset{
  basicstyle=\ttfamily\small,
  keywordstyle=\color{ts-blue},
  commentstyle=\color{ts-gray},
  stringstyle=\color{ts-green},
  numbers=left,
  numberstyle=\tiny\color{ts-gray},
  frame=single,
  breaklines=true,
  backgroundcolor=\color{ts-light}
}


% ------------------------------------------------------------
% THEOREM ENVIRONMENTS
% ------------------------------------------------------------

\theoremstyle{definition}
\newtheorem{definition}{Definition}[section]
\newtheorem{example}{Example}[section]
\newtheorem{exercise}{Exercise}[section]

\theoremstyle{plain}
\newtheorem{theorem}{Theorem}[section]
\newtheorem{lemma}[theorem]{Lemma}
\newtheorem{proposition}[theorem]{Proposition}
\newtheorem{corollary}[theorem]{Corollary}

\theoremstyle{remark}
\newtheorem*{remark}{Remark}
\newtheorem*{note}{Note}


% ------------------------------------------------------------
% TCOLORBOX ENVIRONMENTS
% ------------------------------------------------------------

% Problem statement box
\newtcolorbox{problembox}[1][]{
  enhanced,
  breakable,
  colback=ts-light,
  colframe=ts-blue,
  title={Problem},
  fonttitle=\bfseries,
  #1
}

% Statistical bridge box (green accent)
\newtcolorbox{bridgebox}[1][]{
  enhanced,
  breakable,
  colback=ts-light,
  colframe=ts-green,
  title={Statistical Bridge},
  fonttitle=\bfseries\color{ts-green},
  #1
}

% Warning / watch out box (amber accent)
\newtcolorbox{warningbox}[1][]{
  enhanced,
  breakable,
  colback=ts-light,
  colframe=ts-amber,
  title={Watch Out},
  fonttitle=\bfseries\color{ts-amber},
  #1
}

% Hint box (for exercise files — only appears in solution sets)
\newtcolorbox{hintbox}[1][]{
  enhanced,
  breakable,
  colback=ts-light,
  colframe=ts-gray,
  title={Hint},
  fonttitle=\bfseries\color{ts-gray},
  #1
}


% ------------------------------------------------------------
% NOTATION MACROS
% ------------------------------------------------------------

% --- Number systems ---
\newcommand{\R}{\mathbb{R}}             % Real numbers
\newcommand{\N}{\mathbb{N}}             % Natural numbers
\newcommand{\Z}{\mathbb{Z}}             % Integers
\newcommand{\Q}{\mathbb{Q}}             % Rationals
\newcommand{\C}{\mathbb{C}}             % Complex numbers

% --- Vectors & matrices ---
% Convention: bold lowercase for vectors, bold uppercase for matrices
\newcommand{\va}{\bm{a}}
\newcommand{\vb}{\bm{b}}
\newcommand{\vc}{\bm{c}}
\newcommand{\vd}{\bm{d}}
\newcommand{\ve}{\bm{e}}
\newcommand{\vf}{\bm{f}}
\newcommand{\vg}{\bm{g}}
\newcommand{\vh}{\bm{h}}
\newcommand{\vi}{\bm{i}}               % Unit vector i
\newcommand{\vj}{\bm{j}}               % Unit vector j
\newcommand{\vk}{\bm{k}}               % Unit vector k
\newcommand{\vn}{\bm{n}}               % Normal vector
\newcommand{\vr}{\bm{r}}               % Position vector
\newcommand{\vu}{\bm{u}}               % Unit direction vector
\newcommand{\vv}{\bm{v}}
\newcommand{\vw}{\bm{w}}
\newcommand{\vx}{\bm{x}}               % Input vector
\newcommand{\vy}{\bm{y}}               % Output / observation vector
\newcommand{\vz}{\bm{z}}               % Latent vector
\newcommand{\vzero}{\bm{0}}            % Zero vector

% Statistical vectors
\newcommand{\vmu}{\bm{\mu}}            % Mean vector
\newcommand{\vbeta}{\bm{\beta}}        % Regression coefficients
\newcommand{\vtheta}{\bm{\theta}}      % Parameter vector
\newcommand{\veps}{\bm{\varepsilon}}   % Error vector

% Matrices
\newcommand{\mA}{\bm{A}}
\newcommand{\mB}{\bm{B}}
\newcommand{\mC}{\bm{C}}
\newcommand{\mH}{\bm{H}}               % Hessian (also \Hess below)
\newcommand{\mI}{\bm{I}}               % Identity matrix
\newcommand{\mJ}{\bm{J}}               % Jacobian (also \Jac below)
\newcommand{\mK}{\bm{K}}               % Kernel matrix
\newcommand{\mL}{\bm{L}}               % Lower triangular / Cholesky
\newcommand{\mQ}{\bm{Q}}               % Orthogonal matrix
\newcommand{\mR}{\bm{R}}               % Upper triangular
\newcommand{\mS}{\bm{S}}               % Sample covariance
\newcommand{\mSigma}{\bm{\Sigma}}      % Covariance matrix
\newcommand{\mX}{\bm{X}}               % Design matrix
\newcommand{\mY}{\bm{Y}}

% --- Calculus operators ---
\newcommand{\grad}{\nabla}                              % Gradient ∇
\newcommand{\Grad}[1]{\nabla #1}                        % Gradient of #1
\newcommand{\GradP}[2]{\nabla_{#1} #2}                 % Gradient of #2 w.r.t. #1
\newcommand{\Div}[1]{\nabla \cdot #1}                  % Divergence
\newcommand{\Curl}[1]{\nabla \times #1}                % Curl
\newcommand{\Lap}[1]{\nabla^2 #1}                      % Laplacian
\newcommand{\Hess}[1]{\bm{H}_{#1}}                     % Hessian of #1
\newcommand{\Jac}[1]{\bm{J}_{#1}}                      % Jacobian of #1

% Partial derivative shorthand
\newcommand{\pd}[2]{\frac{\partial #1}{\partial #2}}           % ∂f/∂x
\newcommand{\pdd}[2]{\frac{\partial^2 #1}{\partial #2^2}}      % ∂²f/∂x²
\newcommand{\pdm}[3]{\frac{\partial^2 #1}{\partial #2\,\partial #3}}  % ∂²f/∂x∂y

% Ordinary derivative shorthand
\newcommand{\od}[2]{\frac{d #1}{d #2}}                         % df/dx
\newcommand{\odd}[2]{\frac{d^2 #1}{d #2^2}}                    % d²f/dx²

% Integrals
\newcommand{\dx}{\,dx}
\newcommand{\dy}{\,dy}
\newcommand{\dz}{\,dz}
\newcommand{\dt}{\,dt}
\newcommand{\dA}{\,dA}
\newcommand{\dV}{\,dV}
\newcommand{\ds}{\,ds}
\newcommand{\dvx}{\,d\vx}              % Vector integral element

% --- Statistical notation ---
\newcommand{\E}[1]{\mathbb{E}\!\left[#1\right]}          % Expectation
\newcommand{\Var}[1]{\operatorname{Var}\!\left(#1\right)} % Variance
\newcommand{\Cov}[2]{\operatorname{Cov}\!\left(#1,\,#2\right)} % Covariance
\newcommand{\Cor}[2]{\operatorname{Cor}\!\left(#1,\,#2\right)} % Correlation

\newcommand{\given}{\,\vert\,}                           % Conditional bar
\newcommand{\iid}{\overset{\text{iid}}{\sim}}           % iid ~
\newcommand{\ind}{\overset{\text{ind}}{\sim}}           % ind ~

% Distributions
\newcommand{\Normal}{\mathcal{N}}                        % Normal distribution
\newcommand{\MVN}[2]{\mathcal{N}\!\left(#1,\,#2\right)} % Multivariate normal
\newcommand{\Bern}{\operatorname{Bernoulli}}
\newcommand{\Bin}{\operatorname{Binomial}}
\newcommand{\Pois}{\operatorname{Poisson}}
\newcommand{\Unif}{\operatorname{Uniform}}
\newcommand{\Gam}{\operatorname{Gamma}}
\newcommand{\Beta}{\operatorname{Beta}}

% Likelihood & inference
\newcommand{\Lik}{\mathcal{L}}                          % Likelihood
\newcommand{\loglik}{\ell}                              % Log-likelihood
\newcommand{\Score}[1]{s(#1)}                           % Score function
\newcommand{\FI}{\mathcal{I}}                           % Fisher information
\newcommand{\MLE}[1]{\hat{#1}_{\text{MLE}}}            % MLE estimator
\newcommand{\MAP}[1]{\hat{#1}_{\text{MAP}}}            % MAP estimator
\newcommand{\posterior}{p(\vtheta \given \vx)}          % Posterior shorthand
\newcommand{\prior}{p(\vtheta)}                         % Prior shorthand
\newcommand{\likelihood}{p(\vx \given \vtheta)}        % Likelihood shorthand
\newcommand{\KL}[2]{D_{\text{KL}}\!\left(#1\,\|\,#2\right)} % KL divergence

% --- Linear algebra ---
\newcommand{\tr}[1]{\operatorname{tr}\!\left(#1\right)}        % Trace
\newcommand{\rank}[1]{\operatorname{rank}\!\left(#1\right)}    % Rank
\newcommand{\diag}[1]{\operatorname{diag}\!\left(#1\right)}    % Diagonal matrix
\newcommand{\spn}[1]{\operatorname{span}\!\left\{#1\right\}}   % Span
\newcommand{\norm}[1]{\left\|#1\right\|}                       % Norm
\newcommand{\abs}[1]{\left|#1\right|}                          % Absolute value
\newcommand{\inner}[2]{\langle #1,\, #2 \rangle}               % Inner product
\newcommand{\T}{^{\top}}                                        % Transpose superscript

% Positive (semi-)definiteness shorthand
\newcommand{\psd}{\succeq 0}    % Positive semi-definite
\newcommand{\pd}{\succ 0}       % Positive definite (Note: redefines \pd — see below)
% Note: \pd is overloaded: partial derivative above, positive definite here.
% In ambiguous contexts, use \succ 0 directly for positive definite.

% --- Miscellaneous ---
\newcommand{\defeq}{\coloneqq}                          % Definition equality :=
\newcommand{\st}{\text{ s.t. }}                         % "such that"
\newcommand{\approxeq}{\approx}                         % Approximate equality
\newcommand{\BigO}[1]{\mathcal{O}\!\left(#1\right)}    % Big-O notation
\newcommand{\sigmoid}[1]{\sigma\!\left(#1\right)}      % Sigmoid function
\newcommand{\softmax}[1]{\operatorname{softmax}\!\left(#1\right)}


% ------------------------------------------------------------
% DOCUMENT METADATA COMMANDS
% (Set these in each individual exercise file)
% ------------------------------------------------------------
%
%   \newcommand{\coursemodule}{Module 02, Unit 03}
%   \newcommand{\coursetitle}{The Multivariable Chain Rule}
%   \newcommand{\coursedate}{February 2026}
%
% These are used in the header/footer and title block.
% Define them BEFORE \input{...latex-preamble.tex} or immediately after.
%
% Default definitions (overridden per-file):
\providecommand{\coursemodule}{[Module TBD]}
\providecommand{\coursetitle}{[Unit Title TBD]}
\providecommand{\coursedate}{\today}


% ------------------------------------------------------------
% TITLE BLOCK MACRO
% ------------------------------------------------------------
%
% Usage in exercise files:
%   \maketitleblock
%
\newcommand{\maketitleblock}{%
  \begin{center}
    {\LARGE\bfseries\color{ts-blue} Threat Surfaces}\\[4pt]
    {\large Multivariable Calculus for AI Security}\\[2pt]
    {\small\color{ts-gray} fischer\textsuperscript{3} Education}\\[12pt]
    \rule{\linewidth}{0.4pt}\\[6pt]
    {\large\bfseries \coursemodule: \coursetitle}\\[4pt]
    {\small\color{ts-gray} Exercises \textbar\ \coursedate}\\[4pt]
    \rule{\linewidth}{0.4pt}
  \end{center}
  \vspace{12pt}
}


% ============================================================
% END OF PREAMBLE
% ============================================================
 or immediately after.
%
% Default definitions (overridden per-file):
\providecommand{\coursemodule}{[Module TBD]}
\providecommand{\coursetitle}{[Unit Title TBD]}
\providecommand{\coursedate}{\today}


% ------------------------------------------------------------
% TITLE BLOCK MACRO
% ------------------------------------------------------------
%
% Usage in exercise files:
%   \maketitleblock
%
\newcommand{\maketitleblock}{%
  \begin{center}
    {\LARGE\bfseries\color{ts-blue} Threat Surfaces}\\[4pt]
    {\large Multivariable Calculus for AI Security}\\[2pt]
    {\small\color{ts-gray} fischer\textsuperscript{3} Education}\\[12pt]
    \rule{\linewidth}{0.4pt}\\[6pt]
    {\large\bfseries \coursemodule: \coursetitle}\\[4pt]
    {\small\color{ts-gray} Exercises \textbar\ \coursedate}\\[4pt]
    \rule{\linewidth}{0.4pt}
  \end{center}
  \vspace{12pt}
}


% ============================================================
% END OF PREAMBLE
% ============================================================
 or immediately after.
%
% Default definitions (overridden per-file):
\providecommand{\coursemodule}{[Module TBD]}
\providecommand{\coursetitle}{[Unit Title TBD]}
\providecommand{\coursedate}{\today}


% ------------------------------------------------------------
% TITLE BLOCK MACRO
% ------------------------------------------------------------
%
% Usage in exercise files:
%   \maketitleblock
%
\newcommand{\maketitleblock}{%
  \begin{center}
    {\LARGE\bfseries\color{ts-blue} Threat Surfaces}\\[4pt]
    {\large Multivariable Calculus for AI Security}\\[2pt]
    {\small\color{ts-gray} fischer\textsuperscript{3} Education}\\[12pt]
    \rule{\linewidth}{0.4pt}\\[6pt]
    {\large\bfseries \coursemodule: \coursetitle}\\[4pt]
    {\small\color{ts-gray} Exercises \textbar\ \coursedate}\\[4pt]
    \rule{\linewidth}{0.4pt}
  \end{center}
  \vspace{12pt}
}


% ============================================================
% END OF PREAMBLE
% ============================================================


% ============================================================
\begin{document}
% ============================================================

\maketitleblock

\begin{tcolorbox}[
  colback=ts-light,
  colframe=ts-gray,
  title=Before You Begin,
  fonttitle=\bfseries
]
Complete \texttt{notes.md} for this unit before working these exercises.
If trigonometry feels rusty, visit the refresher before starting Problem~2:
\texttt{refreshers/trig-essentials/notes.md}.

\medskip
\textbf{Estimated time:} 25--35 minutes \quad
\textbf{Problems:} 4 \quad
\textbf{Difficulty:}
$\star$ Foundational \enspace
$\star\star$ Applied \enspace
$\star\star\star$ Extension
\end{tcolorbox}

\vspace{16pt}


% ============================================================
% PROBLEM 1 — Dot Product: Algebra & Orthogonality
% ============================================================
\begin{problembox}[title={Problem 1 \hfill $\star$}]
Let $\vu = (3,\, -2,\, 1)\T$, $\vv = (1,\, 4,\, 2)\T$, and $\vw = (2,\, 3,\, -8)\T$,
all in $\R^3$.
\end{problembox}

\begin{enumerate}
  \item Compute each dot product. Show your work.
    \begin{enumerate}
      \item $\vu \cdot \vv$
      \item $\vu \cdot \vw$
      \item $\vv \cdot \vw$
    \end{enumerate}

  \item Based on your answers to part (a), identify any pair of orthogonal
  vectors. State the orthogonality condition you are applying.

  \item Verify that $\norm{\vu}^2 = \vu \cdot \vu$ by computing both sides
  independently and confirming they are equal.

  \item Is it possible for $\vx \cdot \vy > 0$ and $\vx \cdot \vz > 0$
  but $\vy \cdot \vz < 0$? Either provide an example in $\R^2$,
  or prove it is impossible.
\end{enumerate}

\vspace{20pt}


% ============================================================
% PROBLEM 2 — Angles & Projections
% ============================================================
\begin{problembox}[title={Problem 2 \hfill $\star\star$}]
Let $\vp = (4,\, 1)\T$ and $\vq = (1,\, 3)\T$ in $\R^2$.
\end{problembox}

\begin{enumerate}
  \item Find the angle $\theta$ between $\vp$ and $\vq$.
  Express your answer both as an exact expression involving $\arccos$
  and as a decimal approximation in degrees.

  \item Compute the \textbf{vector projection} of $\vp$ onto $\vq$.
  Label your answer $\text{proj}_{\vq}\,\vp$.

  \item Compute the \textbf{orthogonal component} $\vp_\perp = \vp - \text{proj}_{\vq}\,\vp$.

  \item Verify that $\vp_\perp \perp \vq$ by computing their dot product.

  \item Verify the Pythagorean identity:
  $$
  \norm{\vp}^2 = \norm{\text{proj}_{\vq}\,\vp}^2 + \norm{\vp_\perp}^2
  $$
  \emph{This identity always holds when a vector is decomposed into projection
  and orthogonal components. What theorem does it generalize?}
\end{enumerate}

\vspace{20pt}


% ============================================================
% PROBLEM 3 — Statistical Bridge: OLS Geometry
% ============================================================
\begin{problembox}[title={Problem 3 \hfill $\star\star$ — Statistical Bridge}]
In ordinary least squares regression with a single predictor,
we have response values $\vy = (2,\, 4,\, 5,\, 3)\T \in \R^4$
and predictor values $\vx = (1,\, 2,\, 3,\, 2)\T \in \R^4$.

The OLS fitted values (the projection of $\vy$ onto the span of $\vx$) are:
$$
\hat{\vy} = \text{proj}_{\vx}\,\vy = \frac{\vy \cdot \vx}{\vx \cdot \vx}\,\vx
$$

\emph{Note:} This is the vector projection formula from Unit 02 applied
directly in $\R^4$ — one observation per component.
\end{problembox}

\begin{enumerate}
  \item Compute $\vy \cdot \vx$ and $\vx \cdot \vx$. Then compute the
  scalar coefficient $\hat{\beta} = \dfrac{\vy \cdot \vx}{\vx \cdot \vx}$.

  \item Compute the fitted value vector $\hat{\vy} = \hat{\beta}\,\vx$.

  \item Compute the residual vector $\ve = \vy - \hat{\vy}$.

  \item Verify the orthogonality condition: confirm that $\ve \cdot \vx = 0$.
  In one sentence, explain what this means geometrically.

  \item Verify the Pythagorean decomposition:
  $$
  \norm{\vy}^2 = \norm{\hat{\vy}}^2 + \norm{\ve}^2
  $$
  In the context of regression, $\norm{\hat{\vy}}^2$ is related to
  the \emph{explained} variation and $\norm{\ve}^2$ is the
  \emph{unexplained} variation (residual sum of squares).
  Does this decomposition remind you of anything from basic statistics?
\end{enumerate}

\vspace{20pt}


% ============================================================
% PROBLEM 4 — Cross Product & Extension
% ============================================================
\begin{problembox}[title={Problem 4 \hfill $\star\star\star$ — Extension}]
Let $\va = (2,\, 1,\, -1)\T$ and $\vb = (-1,\, 3,\, 2)\T$ in $\R^3$.
\end{problembox}

\begin{enumerate}
  \item Compute $\va \times \vb$ using the determinant formula.

  \item Verify that $\va \times \vb$ is orthogonal to both $\va$ and $\vb$
  by computing the relevant dot products.

  \item Compute $\norm{\va \times \vb}$ and the area of the parallelogram
  spanned by $\va$ and $\vb$.

  \item Compute $\va \times \va$. State the general rule this illustrates
  and explain it geometrically.

  \item \textbf{Challenge.} Use the relationship
  $\norm{\vx \times \vy}^2 + (\vx \cdot \vy)^2 = \norm{\vx}^2\norm{\vy}^2$
  to verify your answer from part (c) without computing the cross product
  components explicitly.
  \emph{Hint:} You already computed $\norm{\va}$, $\norm{\vb}$, and
  $\va \cdot \vb$ in the process of solving the earlier parts.
  What trigonometric identity does this formula encode?
\end{enumerate}

\vspace{20pt}


% ============================================================
% REFLECTION
% ============================================================
\begin{tcolorbox}[
  colback=ts-light,
  colframe=ts-green,
  title=Reflection,
  fonttitle=\bfseries\color{ts-green}
]
Before moving to the Python lab, take two minutes to answer these in your notes:

\begin{enumerate}
  \item The statistical bridge showed that OLS residuals are orthogonal
  to the predictor. In your own words, what does this mean geometrically?
  Draw a sketch if it helps.

  \item Problem 2 had you decompose $\vp$ into a component along $\vq$
  and a component perpendicular to $\vq$. Where exactly does this
  decomposition appear in the OLS formula from Problem 3?

  \item What is one question this unit raised that you want to
  carry forward into Module 06 (the MVN and regression capstone)?
\end{enumerate}

\medskip
\textbf{Next:} Python Lab (\texttt{python-lab.ipynb}) $\;\to\;$
Solutions (\texttt{threat-surfaces-solutions} repository)
\end{tcolorbox}


% ============================================================
\end{document}
% ============================================================
