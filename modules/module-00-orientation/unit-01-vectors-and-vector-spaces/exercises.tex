% ============================================================
% Module 00, Unit 01 — Vectors & Vector Spaces
% Exercises
% Threat Surfaces: Multivariable Calculus for AI Security
% fischer³ Education
% ============================================================

\newcommand{\coursemodule}{Module 00, Unit 01}
\newcommand{\coursetitle}{Vectors \& Vector Spaces}
\newcommand{\coursedate}{February 2026}

\documentclass[11pt, letterpaper]{article}
% ============================================================
% latex-preamble.tex — Threat Surfaces: Multivariable Calculus for AI Security
% fischer³ Education
%
% PURPOSE:
%   Shared preamble included by all exercise and solution files.
%   Defines packages, page layout, theorem environments, and
%   course-specific notation macros.
%
% USAGE:
%   In any exercise or solution .tex file, place the following
%   at the top of the document before \begin{document}:
%
%     % ============================================================
% latex-preamble.tex — Threat Surfaces: Multivariable Calculus for AI Security
% fischer³ Education
%
% PURPOSE:
%   Shared preamble included by all exercise and solution files.
%   Defines packages, page layout, theorem environments, and
%   course-specific notation macros.
%
% USAGE:
%   In any exercise or solution .tex file, place the following
%   at the top of the document before \begin{document}:
%
%     % ============================================================
% latex-preamble.tex — Threat Surfaces: Multivariable Calculus for AI Security
% fischer³ Education
%
% PURPOSE:
%   Shared preamble included by all exercise and solution files.
%   Defines packages, page layout, theorem environments, and
%   course-specific notation macros.
%
% USAGE:
%   In any exercise or solution .tex file, place the following
%   at the top of the document before \begin{document}:
%
%     \input{../../assets/latex-preamble.tex}
%
%   Adjust the relative path as needed for the file's location
%   within the repository.
% ============================================================


% ------------------------------------------------------------
% PACKAGES
% ------------------------------------------------------------

% Page layout
\usepackage[
  margin=1.25in,
  top=1.0in,
  bottom=1.0in,
  headheight=14pt
]{geometry}

% Typography & encoding
\usepackage[T1]{fontenc}
\usepackage[utf8]{inputenc}
\usepackage{microtype}                % Improved justification
\usepackage{lmodern}                  % Scalable CM fonts

% Mathematics
\usepackage{amsmath}                  % Core math environments
\usepackage{amssymb}                  % Extended symbol set
\usepackage{amsthm}                   % Theorem environments
\usepackage{mathtools}                % Extensions to amsmath
\usepackage{bm}                       % Bold math (\bm)
\usepackage{cancel}                   % Strikethrough in math

% Graphics & color
\usepackage{graphicx}
\usepackage{xcolor}

% Course color palette
\definecolor{ts-blue}{RGB}{30, 100, 180}      % Primary — headings, links
\definecolor{ts-amber}{RGB}{200, 120, 20}     % Accent — warnings, highlights
\definecolor{ts-green}{RGB}{30, 140, 80}      % Statistical bridges
\definecolor{ts-gray}{RGB}{100, 100, 110}     % Secondary text
\definecolor{ts-light}{RGB}{245, 247, 250}    % Background tints

% Hyperlinks
\usepackage{hyperref}
\hypersetup{
  colorlinks=true,
  linkcolor=ts-blue,
  citecolor=ts-blue,
  urlcolor=ts-blue,
  pdfauthor={fischer\textsuperscript{3} Education},
  pdfcreator={LaTeX}
}

% Headers & footers
\usepackage{fancyhdr}
\pagestyle{fancy}
\fancyhf{}
\renewcommand{\headrulewidth}{0.4pt}
\fancyhead[L]{\small\color{ts-gray}\textit{Threat Surfaces}}
\fancyhead[R]{\small\color{ts-gray}\textit{\coursemodule}}   % Set \coursemodule in each file
\fancyfoot[C]{\small\color{ts-gray}\thepage}

% Tables
\usepackage{booktabs}
\usepackage{array}

% Lists
\usepackage{enumitem}
\setlist[enumerate,1]{label=\textbf{\arabic*.}}
\setlist[enumerate,2]{label=(\alph*)}

% Boxes & frames
\usepackage{tcolorbox}
\tcbuselibrary{breakable, skins, theorems}

% Code listings (for Python lab cross-references)
\usepackage{listings}
\lstset{
  basicstyle=\ttfamily\small,
  keywordstyle=\color{ts-blue},
  commentstyle=\color{ts-gray},
  stringstyle=\color{ts-green},
  numbers=left,
  numberstyle=\tiny\color{ts-gray},
  frame=single,
  breaklines=true,
  backgroundcolor=\color{ts-light}
}


% ------------------------------------------------------------
% THEOREM ENVIRONMENTS
% ------------------------------------------------------------

\theoremstyle{definition}
\newtheorem{definition}{Definition}[section]
\newtheorem{example}{Example}[section]
\newtheorem{exercise}{Exercise}[section]

\theoremstyle{plain}
\newtheorem{theorem}{Theorem}[section]
\newtheorem{lemma}[theorem]{Lemma}
\newtheorem{proposition}[theorem]{Proposition}
\newtheorem{corollary}[theorem]{Corollary}

\theoremstyle{remark}
\newtheorem*{remark}{Remark}
\newtheorem*{note}{Note}


% ------------------------------------------------------------
% TCOLORBOX ENVIRONMENTS
% ------------------------------------------------------------

% Problem statement box
\newtcolorbox{problembox}[1][]{
  enhanced,
  breakable,
  colback=ts-light,
  colframe=ts-blue,
  title={Problem},
  fonttitle=\bfseries,
  #1
}

% Statistical bridge box (green accent)
\newtcolorbox{bridgebox}[1][]{
  enhanced,
  breakable,
  colback=ts-light,
  colframe=ts-green,
  title={Statistical Bridge},
  fonttitle=\bfseries\color{ts-green},
  #1
}

% Warning / watch out box (amber accent)
\newtcolorbox{warningbox}[1][]{
  enhanced,
  breakable,
  colback=ts-light,
  colframe=ts-amber,
  title={Watch Out},
  fonttitle=\bfseries\color{ts-amber},
  #1
}

% Hint box (for exercise files — only appears in solution sets)
\newtcolorbox{hintbox}[1][]{
  enhanced,
  breakable,
  colback=ts-light,
  colframe=ts-gray,
  title={Hint},
  fonttitle=\bfseries\color{ts-gray},
  #1
}


% ------------------------------------------------------------
% NOTATION MACROS
% ------------------------------------------------------------

% --- Number systems ---
\newcommand{\R}{\mathbb{R}}             % Real numbers
\newcommand{\N}{\mathbb{N}}             % Natural numbers
\newcommand{\Z}{\mathbb{Z}}             % Integers
\newcommand{\Q}{\mathbb{Q}}             % Rationals
\newcommand{\C}{\mathbb{C}}             % Complex numbers

% --- Vectors & matrices ---
% Convention: bold lowercase for vectors, bold uppercase for matrices
\newcommand{\va}{\bm{a}}
\newcommand{\vb}{\bm{b}}
\newcommand{\vc}{\bm{c}}
\newcommand{\vd}{\bm{d}}
\newcommand{\ve}{\bm{e}}
\newcommand{\vf}{\bm{f}}
\newcommand{\vg}{\bm{g}}
\newcommand{\vh}{\bm{h}}
\newcommand{\vi}{\bm{i}}               % Unit vector i
\newcommand{\vj}{\bm{j}}               % Unit vector j
\newcommand{\vk}{\bm{k}}               % Unit vector k
\newcommand{\vn}{\bm{n}}               % Normal vector
\newcommand{\vr}{\bm{r}}               % Position vector
\newcommand{\vu}{\bm{u}}               % Unit direction vector
\newcommand{\vv}{\bm{v}}
\newcommand{\vw}{\bm{w}}
\newcommand{\vx}{\bm{x}}               % Input vector
\newcommand{\vy}{\bm{y}}               % Output / observation vector
\newcommand{\vz}{\bm{z}}               % Latent vector
\newcommand{\vzero}{\bm{0}}            % Zero vector

% Statistical vectors
\newcommand{\vmu}{\bm{\mu}}            % Mean vector
\newcommand{\vbeta}{\bm{\beta}}        % Regression coefficients
\newcommand{\vtheta}{\bm{\theta}}      % Parameter vector
\newcommand{\veps}{\bm{\varepsilon}}   % Error vector

% Matrices
\newcommand{\mA}{\bm{A}}
\newcommand{\mB}{\bm{B}}
\newcommand{\mC}{\bm{C}}
\newcommand{\mH}{\bm{H}}               % Hessian (also \Hess below)
\newcommand{\mI}{\bm{I}}               % Identity matrix
\newcommand{\mJ}{\bm{J}}               % Jacobian (also \Jac below)
\newcommand{\mK}{\bm{K}}               % Kernel matrix
\newcommand{\mL}{\bm{L}}               % Lower triangular / Cholesky
\newcommand{\mQ}{\bm{Q}}               % Orthogonal matrix
\newcommand{\mR}{\bm{R}}               % Upper triangular
\newcommand{\mS}{\bm{S}}               % Sample covariance
\newcommand{\mSigma}{\bm{\Sigma}}      % Covariance matrix
\newcommand{\mX}{\bm{X}}               % Design matrix
\newcommand{\mY}{\bm{Y}}

% --- Calculus operators ---
\newcommand{\grad}{\nabla}                              % Gradient ∇
\newcommand{\Grad}[1]{\nabla #1}                        % Gradient of #1
\newcommand{\GradP}[2]{\nabla_{#1} #2}                 % Gradient of #2 w.r.t. #1
\newcommand{\Div}[1]{\nabla \cdot #1}                  % Divergence
\newcommand{\Curl}[1]{\nabla \times #1}                % Curl
\newcommand{\Lap}[1]{\nabla^2 #1}                      % Laplacian
\newcommand{\Hess}[1]{\bm{H}_{#1}}                     % Hessian of #1
\newcommand{\Jac}[1]{\bm{J}_{#1}}                      % Jacobian of #1

% Partial derivative shorthand
\newcommand{\pd}[2]{\frac{\partial #1}{\partial #2}}           % ∂f/∂x
\newcommand{\pdd}[2]{\frac{\partial^2 #1}{\partial #2^2}}      % ∂²f/∂x²
\newcommand{\pdm}[3]{\frac{\partial^2 #1}{\partial #2\,\partial #3}}  % ∂²f/∂x∂y

% Ordinary derivative shorthand
\newcommand{\od}[2]{\frac{d #1}{d #2}}                         % df/dx
\newcommand{\odd}[2]{\frac{d^2 #1}{d #2^2}}                    % d²f/dx²

% Integrals
\newcommand{\dx}{\,dx}
\newcommand{\dy}{\,dy}
\newcommand{\dz}{\,dz}
\newcommand{\dt}{\,dt}
\newcommand{\dA}{\,dA}
\newcommand{\dV}{\,dV}
\newcommand{\ds}{\,ds}
\newcommand{\dvx}{\,d\vx}              % Vector integral element

% --- Statistical notation ---
\newcommand{\E}[1]{\mathbb{E}\!\left[#1\right]}          % Expectation
\newcommand{\Var}[1]{\operatorname{Var}\!\left(#1\right)} % Variance
\newcommand{\Cov}[2]{\operatorname{Cov}\!\left(#1,\,#2\right)} % Covariance
\newcommand{\Cor}[2]{\operatorname{Cor}\!\left(#1,\,#2\right)} % Correlation

\newcommand{\given}{\,\vert\,}                           % Conditional bar
\newcommand{\iid}{\overset{\text{iid}}{\sim}}           % iid ~
\newcommand{\ind}{\overset{\text{ind}}{\sim}}           % ind ~

% Distributions
\newcommand{\Normal}{\mathcal{N}}                        % Normal distribution
\newcommand{\MVN}[2]{\mathcal{N}\!\left(#1,\,#2\right)} % Multivariate normal
\newcommand{\Bern}{\operatorname{Bernoulli}}
\newcommand{\Bin}{\operatorname{Binomial}}
\newcommand{\Pois}{\operatorname{Poisson}}
\newcommand{\Unif}{\operatorname{Uniform}}
\newcommand{\Gam}{\operatorname{Gamma}}
\newcommand{\Beta}{\operatorname{Beta}}

% Likelihood & inference
\newcommand{\Lik}{\mathcal{L}}                          % Likelihood
\newcommand{\loglik}{\ell}                              % Log-likelihood
\newcommand{\Score}[1]{s(#1)}                           % Score function
\newcommand{\FI}{\mathcal{I}}                           % Fisher information
\newcommand{\MLE}[1]{\hat{#1}_{\text{MLE}}}            % MLE estimator
\newcommand{\MAP}[1]{\hat{#1}_{\text{MAP}}}            % MAP estimator
\newcommand{\posterior}{p(\vtheta \given \vx)}          % Posterior shorthand
\newcommand{\prior}{p(\vtheta)}                         % Prior shorthand
\newcommand{\likelihood}{p(\vx \given \vtheta)}        % Likelihood shorthand
\newcommand{\KL}[2]{D_{\text{KL}}\!\left(#1\,\|\,#2\right)} % KL divergence

% --- Linear algebra ---
\newcommand{\tr}[1]{\operatorname{tr}\!\left(#1\right)}        % Trace
\newcommand{\rank}[1]{\operatorname{rank}\!\left(#1\right)}    % Rank
\newcommand{\diag}[1]{\operatorname{diag}\!\left(#1\right)}    % Diagonal matrix
\newcommand{\spn}[1]{\operatorname{span}\!\left\{#1\right\}}   % Span
\newcommand{\norm}[1]{\left\|#1\right\|}                       % Norm
\newcommand{\abs}[1]{\left|#1\right|}                          % Absolute value
\newcommand{\inner}[2]{\langle #1,\, #2 \rangle}               % Inner product
\newcommand{\T}{^{\top}}                                        % Transpose superscript

% Positive (semi-)definiteness shorthand
\newcommand{\psd}{\succeq 0}    % Positive semi-definite
\newcommand{\pd}{\succ 0}       % Positive definite (Note: redefines \pd — see below)
% Note: \pd is overloaded: partial derivative above, positive definite here.
% In ambiguous contexts, use \succ 0 directly for positive definite.

% --- Miscellaneous ---
\newcommand{\defeq}{\coloneqq}                          % Definition equality :=
\newcommand{\st}{\text{ s.t. }}                         % "such that"
\newcommand{\approxeq}{\approx}                         % Approximate equality
\newcommand{\BigO}[1]{\mathcal{O}\!\left(#1\right)}    % Big-O notation
\newcommand{\sigmoid}[1]{\sigma\!\left(#1\right)}      % Sigmoid function
\newcommand{\softmax}[1]{\operatorname{softmax}\!\left(#1\right)}


% ------------------------------------------------------------
% DOCUMENT METADATA COMMANDS
% (Set these in each individual exercise file)
% ------------------------------------------------------------
%
%   \newcommand{\coursemodule}{Module 02, Unit 03}
%   \newcommand{\coursetitle}{The Multivariable Chain Rule}
%   \newcommand{\coursedate}{February 2026}
%
% These are used in the header/footer and title block.
% Define them BEFORE \input{...latex-preamble.tex} or immediately after.
%
% Default definitions (overridden per-file):
\providecommand{\coursemodule}{[Module TBD]}
\providecommand{\coursetitle}{[Unit Title TBD]}
\providecommand{\coursedate}{\today}


% ------------------------------------------------------------
% TITLE BLOCK MACRO
% ------------------------------------------------------------
%
% Usage in exercise files:
%   \maketitleblock
%
\newcommand{\maketitleblock}{%
  \begin{center}
    {\LARGE\bfseries\color{ts-blue} Threat Surfaces}\\[4pt]
    {\large Multivariable Calculus for AI Security}\\[2pt]
    {\small\color{ts-gray} fischer\textsuperscript{3} Education}\\[12pt]
    \rule{\linewidth}{0.4pt}\\[6pt]
    {\large\bfseries \coursemodule: \coursetitle}\\[4pt]
    {\small\color{ts-gray} Exercises \textbar\ \coursedate}\\[4pt]
    \rule{\linewidth}{0.4pt}
  \end{center}
  \vspace{12pt}
}


% ============================================================
% END OF PREAMBLE
% ============================================================

%
%   Adjust the relative path as needed for the file's location
%   within the repository.
% ============================================================


% ------------------------------------------------------------
% PACKAGES
% ------------------------------------------------------------

% Page layout
\usepackage[
  margin=1.25in,
  top=1.0in,
  bottom=1.0in,
  headheight=14pt
]{geometry}

% Typography & encoding
\usepackage[T1]{fontenc}
\usepackage[utf8]{inputenc}
\usepackage{microtype}                % Improved justification
\usepackage{lmodern}                  % Scalable CM fonts

% Mathematics
\usepackage{amsmath}                  % Core math environments
\usepackage{amssymb}                  % Extended symbol set
\usepackage{amsthm}                   % Theorem environments
\usepackage{mathtools}                % Extensions to amsmath
\usepackage{bm}                       % Bold math (\bm)
\usepackage{cancel}                   % Strikethrough in math

% Graphics & color
\usepackage{graphicx}
\usepackage{xcolor}

% Course color palette
\definecolor{ts-blue}{RGB}{30, 100, 180}      % Primary — headings, links
\definecolor{ts-amber}{RGB}{200, 120, 20}     % Accent — warnings, highlights
\definecolor{ts-green}{RGB}{30, 140, 80}      % Statistical bridges
\definecolor{ts-gray}{RGB}{100, 100, 110}     % Secondary text
\definecolor{ts-light}{RGB}{245, 247, 250}    % Background tints

% Hyperlinks
\usepackage{hyperref}
\hypersetup{
  colorlinks=true,
  linkcolor=ts-blue,
  citecolor=ts-blue,
  urlcolor=ts-blue,
  pdfauthor={fischer\textsuperscript{3} Education},
  pdfcreator={LaTeX}
}

% Headers & footers
\usepackage{fancyhdr}
\pagestyle{fancy}
\fancyhf{}
\renewcommand{\headrulewidth}{0.4pt}
\fancyhead[L]{\small\color{ts-gray}\textit{Threat Surfaces}}
\fancyhead[R]{\small\color{ts-gray}\textit{\coursemodule}}   % Set \coursemodule in each file
\fancyfoot[C]{\small\color{ts-gray}\thepage}

% Tables
\usepackage{booktabs}
\usepackage{array}

% Lists
\usepackage{enumitem}
\setlist[enumerate,1]{label=\textbf{\arabic*.}}
\setlist[enumerate,2]{label=(\alph*)}

% Boxes & frames
\usepackage{tcolorbox}
\tcbuselibrary{breakable, skins, theorems}

% Code listings (for Python lab cross-references)
\usepackage{listings}
\lstset{
  basicstyle=\ttfamily\small,
  keywordstyle=\color{ts-blue},
  commentstyle=\color{ts-gray},
  stringstyle=\color{ts-green},
  numbers=left,
  numberstyle=\tiny\color{ts-gray},
  frame=single,
  breaklines=true,
  backgroundcolor=\color{ts-light}
}


% ------------------------------------------------------------
% THEOREM ENVIRONMENTS
% ------------------------------------------------------------

\theoremstyle{definition}
\newtheorem{definition}{Definition}[section]
\newtheorem{example}{Example}[section]
\newtheorem{exercise}{Exercise}[section]

\theoremstyle{plain}
\newtheorem{theorem}{Theorem}[section]
\newtheorem{lemma}[theorem]{Lemma}
\newtheorem{proposition}[theorem]{Proposition}
\newtheorem{corollary}[theorem]{Corollary}

\theoremstyle{remark}
\newtheorem*{remark}{Remark}
\newtheorem*{note}{Note}


% ------------------------------------------------------------
% TCOLORBOX ENVIRONMENTS
% ------------------------------------------------------------

% Problem statement box
\newtcolorbox{problembox}[1][]{
  enhanced,
  breakable,
  colback=ts-light,
  colframe=ts-blue,
  title={Problem},
  fonttitle=\bfseries,
  #1
}

% Statistical bridge box (green accent)
\newtcolorbox{bridgebox}[1][]{
  enhanced,
  breakable,
  colback=ts-light,
  colframe=ts-green,
  title={Statistical Bridge},
  fonttitle=\bfseries\color{ts-green},
  #1
}

% Warning / watch out box (amber accent)
\newtcolorbox{warningbox}[1][]{
  enhanced,
  breakable,
  colback=ts-light,
  colframe=ts-amber,
  title={Watch Out},
  fonttitle=\bfseries\color{ts-amber},
  #1
}

% Hint box (for exercise files — only appears in solution sets)
\newtcolorbox{hintbox}[1][]{
  enhanced,
  breakable,
  colback=ts-light,
  colframe=ts-gray,
  title={Hint},
  fonttitle=\bfseries\color{ts-gray},
  #1
}


% ------------------------------------------------------------
% NOTATION MACROS
% ------------------------------------------------------------

% --- Number systems ---
\newcommand{\R}{\mathbb{R}}             % Real numbers
\newcommand{\N}{\mathbb{N}}             % Natural numbers
\newcommand{\Z}{\mathbb{Z}}             % Integers
\newcommand{\Q}{\mathbb{Q}}             % Rationals
\newcommand{\C}{\mathbb{C}}             % Complex numbers

% --- Vectors & matrices ---
% Convention: bold lowercase for vectors, bold uppercase for matrices
\newcommand{\va}{\bm{a}}
\newcommand{\vb}{\bm{b}}
\newcommand{\vc}{\bm{c}}
\newcommand{\vd}{\bm{d}}
\newcommand{\ve}{\bm{e}}
\newcommand{\vf}{\bm{f}}
\newcommand{\vg}{\bm{g}}
\newcommand{\vh}{\bm{h}}
\newcommand{\vi}{\bm{i}}               % Unit vector i
\newcommand{\vj}{\bm{j}}               % Unit vector j
\newcommand{\vk}{\bm{k}}               % Unit vector k
\newcommand{\vn}{\bm{n}}               % Normal vector
\newcommand{\vr}{\bm{r}}               % Position vector
\newcommand{\vu}{\bm{u}}               % Unit direction vector
\newcommand{\vv}{\bm{v}}
\newcommand{\vw}{\bm{w}}
\newcommand{\vx}{\bm{x}}               % Input vector
\newcommand{\vy}{\bm{y}}               % Output / observation vector
\newcommand{\vz}{\bm{z}}               % Latent vector
\newcommand{\vzero}{\bm{0}}            % Zero vector

% Statistical vectors
\newcommand{\vmu}{\bm{\mu}}            % Mean vector
\newcommand{\vbeta}{\bm{\beta}}        % Regression coefficients
\newcommand{\vtheta}{\bm{\theta}}      % Parameter vector
\newcommand{\veps}{\bm{\varepsilon}}   % Error vector

% Matrices
\newcommand{\mA}{\bm{A}}
\newcommand{\mB}{\bm{B}}
\newcommand{\mC}{\bm{C}}
\newcommand{\mH}{\bm{H}}               % Hessian (also \Hess below)
\newcommand{\mI}{\bm{I}}               % Identity matrix
\newcommand{\mJ}{\bm{J}}               % Jacobian (also \Jac below)
\newcommand{\mK}{\bm{K}}               % Kernel matrix
\newcommand{\mL}{\bm{L}}               % Lower triangular / Cholesky
\newcommand{\mQ}{\bm{Q}}               % Orthogonal matrix
\newcommand{\mR}{\bm{R}}               % Upper triangular
\newcommand{\mS}{\bm{S}}               % Sample covariance
\newcommand{\mSigma}{\bm{\Sigma}}      % Covariance matrix
\newcommand{\mX}{\bm{X}}               % Design matrix
\newcommand{\mY}{\bm{Y}}

% --- Calculus operators ---
\newcommand{\grad}{\nabla}                              % Gradient ∇
\newcommand{\Grad}[1]{\nabla #1}                        % Gradient of #1
\newcommand{\GradP}[2]{\nabla_{#1} #2}                 % Gradient of #2 w.r.t. #1
\newcommand{\Div}[1]{\nabla \cdot #1}                  % Divergence
\newcommand{\Curl}[1]{\nabla \times #1}                % Curl
\newcommand{\Lap}[1]{\nabla^2 #1}                      % Laplacian
\newcommand{\Hess}[1]{\bm{H}_{#1}}                     % Hessian of #1
\newcommand{\Jac}[1]{\bm{J}_{#1}}                      % Jacobian of #1

% Partial derivative shorthand
\newcommand{\pd}[2]{\frac{\partial #1}{\partial #2}}           % ∂f/∂x
\newcommand{\pdd}[2]{\frac{\partial^2 #1}{\partial #2^2}}      % ∂²f/∂x²
\newcommand{\pdm}[3]{\frac{\partial^2 #1}{\partial #2\,\partial #3}}  % ∂²f/∂x∂y

% Ordinary derivative shorthand
\newcommand{\od}[2]{\frac{d #1}{d #2}}                         % df/dx
\newcommand{\odd}[2]{\frac{d^2 #1}{d #2^2}}                    % d²f/dx²

% Integrals
\newcommand{\dx}{\,dx}
\newcommand{\dy}{\,dy}
\newcommand{\dz}{\,dz}
\newcommand{\dt}{\,dt}
\newcommand{\dA}{\,dA}
\newcommand{\dV}{\,dV}
\newcommand{\ds}{\,ds}
\newcommand{\dvx}{\,d\vx}              % Vector integral element

% --- Statistical notation ---
\newcommand{\E}[1]{\mathbb{E}\!\left[#1\right]}          % Expectation
\newcommand{\Var}[1]{\operatorname{Var}\!\left(#1\right)} % Variance
\newcommand{\Cov}[2]{\operatorname{Cov}\!\left(#1,\,#2\right)} % Covariance
\newcommand{\Cor}[2]{\operatorname{Cor}\!\left(#1,\,#2\right)} % Correlation

\newcommand{\given}{\,\vert\,}                           % Conditional bar
\newcommand{\iid}{\overset{\text{iid}}{\sim}}           % iid ~
\newcommand{\ind}{\overset{\text{ind}}{\sim}}           % ind ~

% Distributions
\newcommand{\Normal}{\mathcal{N}}                        % Normal distribution
\newcommand{\MVN}[2]{\mathcal{N}\!\left(#1,\,#2\right)} % Multivariate normal
\newcommand{\Bern}{\operatorname{Bernoulli}}
\newcommand{\Bin}{\operatorname{Binomial}}
\newcommand{\Pois}{\operatorname{Poisson}}
\newcommand{\Unif}{\operatorname{Uniform}}
\newcommand{\Gam}{\operatorname{Gamma}}
\newcommand{\Beta}{\operatorname{Beta}}

% Likelihood & inference
\newcommand{\Lik}{\mathcal{L}}                          % Likelihood
\newcommand{\loglik}{\ell}                              % Log-likelihood
\newcommand{\Score}[1]{s(#1)}                           % Score function
\newcommand{\FI}{\mathcal{I}}                           % Fisher information
\newcommand{\MLE}[1]{\hat{#1}_{\text{MLE}}}            % MLE estimator
\newcommand{\MAP}[1]{\hat{#1}_{\text{MAP}}}            % MAP estimator
\newcommand{\posterior}{p(\vtheta \given \vx)}          % Posterior shorthand
\newcommand{\prior}{p(\vtheta)}                         % Prior shorthand
\newcommand{\likelihood}{p(\vx \given \vtheta)}        % Likelihood shorthand
\newcommand{\KL}[2]{D_{\text{KL}}\!\left(#1\,\|\,#2\right)} % KL divergence

% --- Linear algebra ---
\newcommand{\tr}[1]{\operatorname{tr}\!\left(#1\right)}        % Trace
\newcommand{\rank}[1]{\operatorname{rank}\!\left(#1\right)}    % Rank
\newcommand{\diag}[1]{\operatorname{diag}\!\left(#1\right)}    % Diagonal matrix
\newcommand{\spn}[1]{\operatorname{span}\!\left\{#1\right\}}   % Span
\newcommand{\norm}[1]{\left\|#1\right\|}                       % Norm
\newcommand{\abs}[1]{\left|#1\right|}                          % Absolute value
\newcommand{\inner}[2]{\langle #1,\, #2 \rangle}               % Inner product
\newcommand{\T}{^{\top}}                                        % Transpose superscript

% Positive (semi-)definiteness shorthand
\newcommand{\psd}{\succeq 0}    % Positive semi-definite
\newcommand{\pd}{\succ 0}       % Positive definite (Note: redefines \pd — see below)
% Note: \pd is overloaded: partial derivative above, positive definite here.
% In ambiguous contexts, use \succ 0 directly for positive definite.

% --- Miscellaneous ---
\newcommand{\defeq}{\coloneqq}                          % Definition equality :=
\newcommand{\st}{\text{ s.t. }}                         % "such that"
\newcommand{\approxeq}{\approx}                         % Approximate equality
\newcommand{\BigO}[1]{\mathcal{O}\!\left(#1\right)}    % Big-O notation
\newcommand{\sigmoid}[1]{\sigma\!\left(#1\right)}      % Sigmoid function
\newcommand{\softmax}[1]{\operatorname{softmax}\!\left(#1\right)}


% ------------------------------------------------------------
% DOCUMENT METADATA COMMANDS
% (Set these in each individual exercise file)
% ------------------------------------------------------------
%
%   \newcommand{\coursemodule}{Module 02, Unit 03}
%   \newcommand{\coursetitle}{The Multivariable Chain Rule}
%   \newcommand{\coursedate}{February 2026}
%
% These are used in the header/footer and title block.
% Define them BEFORE % ============================================================
% latex-preamble.tex — Threat Surfaces: Multivariable Calculus for AI Security
% fischer³ Education
%
% PURPOSE:
%   Shared preamble included by all exercise and solution files.
%   Defines packages, page layout, theorem environments, and
%   course-specific notation macros.
%
% USAGE:
%   In any exercise or solution .tex file, place the following
%   at the top of the document before \begin{document}:
%
%     \input{../../assets/latex-preamble.tex}
%
%   Adjust the relative path as needed for the file's location
%   within the repository.
% ============================================================


% ------------------------------------------------------------
% PACKAGES
% ------------------------------------------------------------

% Page layout
\usepackage[
  margin=1.25in,
  top=1.0in,
  bottom=1.0in,
  headheight=14pt
]{geometry}

% Typography & encoding
\usepackage[T1]{fontenc}
\usepackage[utf8]{inputenc}
\usepackage{microtype}                % Improved justification
\usepackage{lmodern}                  % Scalable CM fonts

% Mathematics
\usepackage{amsmath}                  % Core math environments
\usepackage{amssymb}                  % Extended symbol set
\usepackage{amsthm}                   % Theorem environments
\usepackage{mathtools}                % Extensions to amsmath
\usepackage{bm}                       % Bold math (\bm)
\usepackage{cancel}                   % Strikethrough in math

% Graphics & color
\usepackage{graphicx}
\usepackage{xcolor}

% Course color palette
\definecolor{ts-blue}{RGB}{30, 100, 180}      % Primary — headings, links
\definecolor{ts-amber}{RGB}{200, 120, 20}     % Accent — warnings, highlights
\definecolor{ts-green}{RGB}{30, 140, 80}      % Statistical bridges
\definecolor{ts-gray}{RGB}{100, 100, 110}     % Secondary text
\definecolor{ts-light}{RGB}{245, 247, 250}    % Background tints

% Hyperlinks
\usepackage{hyperref}
\hypersetup{
  colorlinks=true,
  linkcolor=ts-blue,
  citecolor=ts-blue,
  urlcolor=ts-blue,
  pdfauthor={fischer\textsuperscript{3} Education},
  pdfcreator={LaTeX}
}

% Headers & footers
\usepackage{fancyhdr}
\pagestyle{fancy}
\fancyhf{}
\renewcommand{\headrulewidth}{0.4pt}
\fancyhead[L]{\small\color{ts-gray}\textit{Threat Surfaces}}
\fancyhead[R]{\small\color{ts-gray}\textit{\coursemodule}}   % Set \coursemodule in each file
\fancyfoot[C]{\small\color{ts-gray}\thepage}

% Tables
\usepackage{booktabs}
\usepackage{array}

% Lists
\usepackage{enumitem}
\setlist[enumerate,1]{label=\textbf{\arabic*.}}
\setlist[enumerate,2]{label=(\alph*)}

% Boxes & frames
\usepackage{tcolorbox}
\tcbuselibrary{breakable, skins, theorems}

% Code listings (for Python lab cross-references)
\usepackage{listings}
\lstset{
  basicstyle=\ttfamily\small,
  keywordstyle=\color{ts-blue},
  commentstyle=\color{ts-gray},
  stringstyle=\color{ts-green},
  numbers=left,
  numberstyle=\tiny\color{ts-gray},
  frame=single,
  breaklines=true,
  backgroundcolor=\color{ts-light}
}


% ------------------------------------------------------------
% THEOREM ENVIRONMENTS
% ------------------------------------------------------------

\theoremstyle{definition}
\newtheorem{definition}{Definition}[section]
\newtheorem{example}{Example}[section]
\newtheorem{exercise}{Exercise}[section]

\theoremstyle{plain}
\newtheorem{theorem}{Theorem}[section]
\newtheorem{lemma}[theorem]{Lemma}
\newtheorem{proposition}[theorem]{Proposition}
\newtheorem{corollary}[theorem]{Corollary}

\theoremstyle{remark}
\newtheorem*{remark}{Remark}
\newtheorem*{note}{Note}


% ------------------------------------------------------------
% TCOLORBOX ENVIRONMENTS
% ------------------------------------------------------------

% Problem statement box
\newtcolorbox{problembox}[1][]{
  enhanced,
  breakable,
  colback=ts-light,
  colframe=ts-blue,
  title={Problem},
  fonttitle=\bfseries,
  #1
}

% Statistical bridge box (green accent)
\newtcolorbox{bridgebox}[1][]{
  enhanced,
  breakable,
  colback=ts-light,
  colframe=ts-green,
  title={Statistical Bridge},
  fonttitle=\bfseries\color{ts-green},
  #1
}

% Warning / watch out box (amber accent)
\newtcolorbox{warningbox}[1][]{
  enhanced,
  breakable,
  colback=ts-light,
  colframe=ts-amber,
  title={Watch Out},
  fonttitle=\bfseries\color{ts-amber},
  #1
}

% Hint box (for exercise files — only appears in solution sets)
\newtcolorbox{hintbox}[1][]{
  enhanced,
  breakable,
  colback=ts-light,
  colframe=ts-gray,
  title={Hint},
  fonttitle=\bfseries\color{ts-gray},
  #1
}


% ------------------------------------------------------------
% NOTATION MACROS
% ------------------------------------------------------------

% --- Number systems ---
\newcommand{\R}{\mathbb{R}}             % Real numbers
\newcommand{\N}{\mathbb{N}}             % Natural numbers
\newcommand{\Z}{\mathbb{Z}}             % Integers
\newcommand{\Q}{\mathbb{Q}}             % Rationals
\newcommand{\C}{\mathbb{C}}             % Complex numbers

% --- Vectors & matrices ---
% Convention: bold lowercase for vectors, bold uppercase for matrices
\newcommand{\va}{\bm{a}}
\newcommand{\vb}{\bm{b}}
\newcommand{\vc}{\bm{c}}
\newcommand{\vd}{\bm{d}}
\newcommand{\ve}{\bm{e}}
\newcommand{\vf}{\bm{f}}
\newcommand{\vg}{\bm{g}}
\newcommand{\vh}{\bm{h}}
\newcommand{\vi}{\bm{i}}               % Unit vector i
\newcommand{\vj}{\bm{j}}               % Unit vector j
\newcommand{\vk}{\bm{k}}               % Unit vector k
\newcommand{\vn}{\bm{n}}               % Normal vector
\newcommand{\vr}{\bm{r}}               % Position vector
\newcommand{\vu}{\bm{u}}               % Unit direction vector
\newcommand{\vv}{\bm{v}}
\newcommand{\vw}{\bm{w}}
\newcommand{\vx}{\bm{x}}               % Input vector
\newcommand{\vy}{\bm{y}}               % Output / observation vector
\newcommand{\vz}{\bm{z}}               % Latent vector
\newcommand{\vzero}{\bm{0}}            % Zero vector

% Statistical vectors
\newcommand{\vmu}{\bm{\mu}}            % Mean vector
\newcommand{\vbeta}{\bm{\beta}}        % Regression coefficients
\newcommand{\vtheta}{\bm{\theta}}      % Parameter vector
\newcommand{\veps}{\bm{\varepsilon}}   % Error vector

% Matrices
\newcommand{\mA}{\bm{A}}
\newcommand{\mB}{\bm{B}}
\newcommand{\mC}{\bm{C}}
\newcommand{\mH}{\bm{H}}               % Hessian (also \Hess below)
\newcommand{\mI}{\bm{I}}               % Identity matrix
\newcommand{\mJ}{\bm{J}}               % Jacobian (also \Jac below)
\newcommand{\mK}{\bm{K}}               % Kernel matrix
\newcommand{\mL}{\bm{L}}               % Lower triangular / Cholesky
\newcommand{\mQ}{\bm{Q}}               % Orthogonal matrix
\newcommand{\mR}{\bm{R}}               % Upper triangular
\newcommand{\mS}{\bm{S}}               % Sample covariance
\newcommand{\mSigma}{\bm{\Sigma}}      % Covariance matrix
\newcommand{\mX}{\bm{X}}               % Design matrix
\newcommand{\mY}{\bm{Y}}

% --- Calculus operators ---
\newcommand{\grad}{\nabla}                              % Gradient ∇
\newcommand{\Grad}[1]{\nabla #1}                        % Gradient of #1
\newcommand{\GradP}[2]{\nabla_{#1} #2}                 % Gradient of #2 w.r.t. #1
\newcommand{\Div}[1]{\nabla \cdot #1}                  % Divergence
\newcommand{\Curl}[1]{\nabla \times #1}                % Curl
\newcommand{\Lap}[1]{\nabla^2 #1}                      % Laplacian
\newcommand{\Hess}[1]{\bm{H}_{#1}}                     % Hessian of #1
\newcommand{\Jac}[1]{\bm{J}_{#1}}                      % Jacobian of #1

% Partial derivative shorthand
\newcommand{\pd}[2]{\frac{\partial #1}{\partial #2}}           % ∂f/∂x
\newcommand{\pdd}[2]{\frac{\partial^2 #1}{\partial #2^2}}      % ∂²f/∂x²
\newcommand{\pdm}[3]{\frac{\partial^2 #1}{\partial #2\,\partial #3}}  % ∂²f/∂x∂y

% Ordinary derivative shorthand
\newcommand{\od}[2]{\frac{d #1}{d #2}}                         % df/dx
\newcommand{\odd}[2]{\frac{d^2 #1}{d #2^2}}                    % d²f/dx²

% Integrals
\newcommand{\dx}{\,dx}
\newcommand{\dy}{\,dy}
\newcommand{\dz}{\,dz}
\newcommand{\dt}{\,dt}
\newcommand{\dA}{\,dA}
\newcommand{\dV}{\,dV}
\newcommand{\ds}{\,ds}
\newcommand{\dvx}{\,d\vx}              % Vector integral element

% --- Statistical notation ---
\newcommand{\E}[1]{\mathbb{E}\!\left[#1\right]}          % Expectation
\newcommand{\Var}[1]{\operatorname{Var}\!\left(#1\right)} % Variance
\newcommand{\Cov}[2]{\operatorname{Cov}\!\left(#1,\,#2\right)} % Covariance
\newcommand{\Cor}[2]{\operatorname{Cor}\!\left(#1,\,#2\right)} % Correlation

\newcommand{\given}{\,\vert\,}                           % Conditional bar
\newcommand{\iid}{\overset{\text{iid}}{\sim}}           % iid ~
\newcommand{\ind}{\overset{\text{ind}}{\sim}}           % ind ~

% Distributions
\newcommand{\Normal}{\mathcal{N}}                        % Normal distribution
\newcommand{\MVN}[2]{\mathcal{N}\!\left(#1,\,#2\right)} % Multivariate normal
\newcommand{\Bern}{\operatorname{Bernoulli}}
\newcommand{\Bin}{\operatorname{Binomial}}
\newcommand{\Pois}{\operatorname{Poisson}}
\newcommand{\Unif}{\operatorname{Uniform}}
\newcommand{\Gam}{\operatorname{Gamma}}
\newcommand{\Beta}{\operatorname{Beta}}

% Likelihood & inference
\newcommand{\Lik}{\mathcal{L}}                          % Likelihood
\newcommand{\loglik}{\ell}                              % Log-likelihood
\newcommand{\Score}[1]{s(#1)}                           % Score function
\newcommand{\FI}{\mathcal{I}}                           % Fisher information
\newcommand{\MLE}[1]{\hat{#1}_{\text{MLE}}}            % MLE estimator
\newcommand{\MAP}[1]{\hat{#1}_{\text{MAP}}}            % MAP estimator
\newcommand{\posterior}{p(\vtheta \given \vx)}          % Posterior shorthand
\newcommand{\prior}{p(\vtheta)}                         % Prior shorthand
\newcommand{\likelihood}{p(\vx \given \vtheta)}        % Likelihood shorthand
\newcommand{\KL}[2]{D_{\text{KL}}\!\left(#1\,\|\,#2\right)} % KL divergence

% --- Linear algebra ---
\newcommand{\tr}[1]{\operatorname{tr}\!\left(#1\right)}        % Trace
\newcommand{\rank}[1]{\operatorname{rank}\!\left(#1\right)}    % Rank
\newcommand{\diag}[1]{\operatorname{diag}\!\left(#1\right)}    % Diagonal matrix
\newcommand{\spn}[1]{\operatorname{span}\!\left\{#1\right\}}   % Span
\newcommand{\norm}[1]{\left\|#1\right\|}                       % Norm
\newcommand{\abs}[1]{\left|#1\right|}                          % Absolute value
\newcommand{\inner}[2]{\langle #1,\, #2 \rangle}               % Inner product
\newcommand{\T}{^{\top}}                                        % Transpose superscript

% Positive (semi-)definiteness shorthand
\newcommand{\psd}{\succeq 0}    % Positive semi-definite
\newcommand{\pd}{\succ 0}       % Positive definite (Note: redefines \pd — see below)
% Note: \pd is overloaded: partial derivative above, positive definite here.
% In ambiguous contexts, use \succ 0 directly for positive definite.

% --- Miscellaneous ---
\newcommand{\defeq}{\coloneqq}                          % Definition equality :=
\newcommand{\st}{\text{ s.t. }}                         % "such that"
\newcommand{\approxeq}{\approx}                         % Approximate equality
\newcommand{\BigO}[1]{\mathcal{O}\!\left(#1\right)}    % Big-O notation
\newcommand{\sigmoid}[1]{\sigma\!\left(#1\right)}      % Sigmoid function
\newcommand{\softmax}[1]{\operatorname{softmax}\!\left(#1\right)}


% ------------------------------------------------------------
% DOCUMENT METADATA COMMANDS
% (Set these in each individual exercise file)
% ------------------------------------------------------------
%
%   \newcommand{\coursemodule}{Module 02, Unit 03}
%   \newcommand{\coursetitle}{The Multivariable Chain Rule}
%   \newcommand{\coursedate}{February 2026}
%
% These are used in the header/footer and title block.
% Define them BEFORE \input{...latex-preamble.tex} or immediately after.
%
% Default definitions (overridden per-file):
\providecommand{\coursemodule}{[Module TBD]}
\providecommand{\coursetitle}{[Unit Title TBD]}
\providecommand{\coursedate}{\today}


% ------------------------------------------------------------
% TITLE BLOCK MACRO
% ------------------------------------------------------------
%
% Usage in exercise files:
%   \maketitleblock
%
\newcommand{\maketitleblock}{%
  \begin{center}
    {\LARGE\bfseries\color{ts-blue} Threat Surfaces}\\[4pt]
    {\large Multivariable Calculus for AI Security}\\[2pt]
    {\small\color{ts-gray} fischer\textsuperscript{3} Education}\\[12pt]
    \rule{\linewidth}{0.4pt}\\[6pt]
    {\large\bfseries \coursemodule: \coursetitle}\\[4pt]
    {\small\color{ts-gray} Exercises \textbar\ \coursedate}\\[4pt]
    \rule{\linewidth}{0.4pt}
  \end{center}
  \vspace{12pt}
}


% ============================================================
% END OF PREAMBLE
% ============================================================
 or immediately after.
%
% Default definitions (overridden per-file):
\providecommand{\coursemodule}{[Module TBD]}
\providecommand{\coursetitle}{[Unit Title TBD]}
\providecommand{\coursedate}{\today}


% ------------------------------------------------------------
% TITLE BLOCK MACRO
% ------------------------------------------------------------
%
% Usage in exercise files:
%   \maketitleblock
%
\newcommand{\maketitleblock}{%
  \begin{center}
    {\LARGE\bfseries\color{ts-blue} Threat Surfaces}\\[4pt]
    {\large Multivariable Calculus for AI Security}\\[2pt]
    {\small\color{ts-gray} fischer\textsuperscript{3} Education}\\[12pt]
    \rule{\linewidth}{0.4pt}\\[6pt]
    {\large\bfseries \coursemodule: \coursetitle}\\[4pt]
    {\small\color{ts-gray} Exercises \textbar\ \coursedate}\\[4pt]
    \rule{\linewidth}{0.4pt}
  \end{center}
  \vspace{12pt}
}


% ============================================================
% END OF PREAMBLE
% ============================================================

%
%   Adjust the relative path as needed for the file's location
%   within the repository.
% ============================================================


% ------------------------------------------------------------
% PACKAGES
% ------------------------------------------------------------

% Page layout
\usepackage[
  margin=1.25in,
  top=1.0in,
  bottom=1.0in,
  headheight=14pt
]{geometry}

% Typography & encoding
\usepackage[T1]{fontenc}
\usepackage[utf8]{inputenc}
\usepackage{microtype}                % Improved justification
\usepackage{lmodern}                  % Scalable CM fonts

% Mathematics
\usepackage{amsmath}                  % Core math environments
\usepackage{amssymb}                  % Extended symbol set
\usepackage{amsthm}                   % Theorem environments
\usepackage{mathtools}                % Extensions to amsmath
\usepackage{bm}                       % Bold math (\bm)
\usepackage{cancel}                   % Strikethrough in math

% Graphics & color
\usepackage{graphicx}
\usepackage{xcolor}

% Course color palette
\definecolor{ts-blue}{RGB}{30, 100, 180}      % Primary — headings, links
\definecolor{ts-amber}{RGB}{200, 120, 20}     % Accent — warnings, highlights
\definecolor{ts-green}{RGB}{30, 140, 80}      % Statistical bridges
\definecolor{ts-gray}{RGB}{100, 100, 110}     % Secondary text
\definecolor{ts-light}{RGB}{245, 247, 250}    % Background tints

% Hyperlinks
\usepackage{hyperref}
\hypersetup{
  colorlinks=true,
  linkcolor=ts-blue,
  citecolor=ts-blue,
  urlcolor=ts-blue,
  pdfauthor={fischer\textsuperscript{3} Education},
  pdfcreator={LaTeX}
}

% Headers & footers
\usepackage{fancyhdr}
\pagestyle{fancy}
\fancyhf{}
\renewcommand{\headrulewidth}{0.4pt}
\fancyhead[L]{\small\color{ts-gray}\textit{Threat Surfaces}}
\fancyhead[R]{\small\color{ts-gray}\textit{\coursemodule}}   % Set \coursemodule in each file
\fancyfoot[C]{\small\color{ts-gray}\thepage}

% Tables
\usepackage{booktabs}
\usepackage{array}

% Lists
\usepackage{enumitem}
\setlist[enumerate,1]{label=\textbf{\arabic*.}}
\setlist[enumerate,2]{label=(\alph*)}

% Boxes & frames
\usepackage{tcolorbox}
\tcbuselibrary{breakable, skins, theorems}

% Code listings (for Python lab cross-references)
\usepackage{listings}
\lstset{
  basicstyle=\ttfamily\small,
  keywordstyle=\color{ts-blue},
  commentstyle=\color{ts-gray},
  stringstyle=\color{ts-green},
  numbers=left,
  numberstyle=\tiny\color{ts-gray},
  frame=single,
  breaklines=true,
  backgroundcolor=\color{ts-light}
}


% ------------------------------------------------------------
% THEOREM ENVIRONMENTS
% ------------------------------------------------------------

\theoremstyle{definition}
\newtheorem{definition}{Definition}[section]
\newtheorem{example}{Example}[section]
\newtheorem{exercise}{Exercise}[section]

\theoremstyle{plain}
\newtheorem{theorem}{Theorem}[section]
\newtheorem{lemma}[theorem]{Lemma}
\newtheorem{proposition}[theorem]{Proposition}
\newtheorem{corollary}[theorem]{Corollary}

\theoremstyle{remark}
\newtheorem*{remark}{Remark}
\newtheorem*{note}{Note}


% ------------------------------------------------------------
% TCOLORBOX ENVIRONMENTS
% ------------------------------------------------------------

% Problem statement box
\newtcolorbox{problembox}[1][]{
  enhanced,
  breakable,
  colback=ts-light,
  colframe=ts-blue,
  title={Problem},
  fonttitle=\bfseries,
  #1
}

% Statistical bridge box (green accent)
\newtcolorbox{bridgebox}[1][]{
  enhanced,
  breakable,
  colback=ts-light,
  colframe=ts-green,
  title={Statistical Bridge},
  fonttitle=\bfseries\color{ts-green},
  #1
}

% Warning / watch out box (amber accent)
\newtcolorbox{warningbox}[1][]{
  enhanced,
  breakable,
  colback=ts-light,
  colframe=ts-amber,
  title={Watch Out},
  fonttitle=\bfseries\color{ts-amber},
  #1
}

% Hint box (for exercise files — only appears in solution sets)
\newtcolorbox{hintbox}[1][]{
  enhanced,
  breakable,
  colback=ts-light,
  colframe=ts-gray,
  title={Hint},
  fonttitle=\bfseries\color{ts-gray},
  #1
}


% ------------------------------------------------------------
% NOTATION MACROS
% ------------------------------------------------------------

% --- Number systems ---
\newcommand{\R}{\mathbb{R}}             % Real numbers
\newcommand{\N}{\mathbb{N}}             % Natural numbers
\newcommand{\Z}{\mathbb{Z}}             % Integers
\newcommand{\Q}{\mathbb{Q}}             % Rationals
\newcommand{\C}{\mathbb{C}}             % Complex numbers

% --- Vectors & matrices ---
% Convention: bold lowercase for vectors, bold uppercase for matrices
\newcommand{\va}{\bm{a}}
\newcommand{\vb}{\bm{b}}
\newcommand{\vc}{\bm{c}}
\newcommand{\vd}{\bm{d}}
\newcommand{\ve}{\bm{e}}
\newcommand{\vf}{\bm{f}}
\newcommand{\vg}{\bm{g}}
\newcommand{\vh}{\bm{h}}
\newcommand{\vi}{\bm{i}}               % Unit vector i
\newcommand{\vj}{\bm{j}}               % Unit vector j
\newcommand{\vk}{\bm{k}}               % Unit vector k
\newcommand{\vn}{\bm{n}}               % Normal vector
\newcommand{\vr}{\bm{r}}               % Position vector
\newcommand{\vu}{\bm{u}}               % Unit direction vector
\newcommand{\vv}{\bm{v}}
\newcommand{\vw}{\bm{w}}
\newcommand{\vx}{\bm{x}}               % Input vector
\newcommand{\vy}{\bm{y}}               % Output / observation vector
\newcommand{\vz}{\bm{z}}               % Latent vector
\newcommand{\vzero}{\bm{0}}            % Zero vector

% Statistical vectors
\newcommand{\vmu}{\bm{\mu}}            % Mean vector
\newcommand{\vbeta}{\bm{\beta}}        % Regression coefficients
\newcommand{\vtheta}{\bm{\theta}}      % Parameter vector
\newcommand{\veps}{\bm{\varepsilon}}   % Error vector

% Matrices
\newcommand{\mA}{\bm{A}}
\newcommand{\mB}{\bm{B}}
\newcommand{\mC}{\bm{C}}
\newcommand{\mH}{\bm{H}}               % Hessian (also \Hess below)
\newcommand{\mI}{\bm{I}}               % Identity matrix
\newcommand{\mJ}{\bm{J}}               % Jacobian (also \Jac below)
\newcommand{\mK}{\bm{K}}               % Kernel matrix
\newcommand{\mL}{\bm{L}}               % Lower triangular / Cholesky
\newcommand{\mQ}{\bm{Q}}               % Orthogonal matrix
\newcommand{\mR}{\bm{R}}               % Upper triangular
\newcommand{\mS}{\bm{S}}               % Sample covariance
\newcommand{\mSigma}{\bm{\Sigma}}      % Covariance matrix
\newcommand{\mX}{\bm{X}}               % Design matrix
\newcommand{\mY}{\bm{Y}}

% --- Calculus operators ---
\newcommand{\grad}{\nabla}                              % Gradient ∇
\newcommand{\Grad}[1]{\nabla #1}                        % Gradient of #1
\newcommand{\GradP}[2]{\nabla_{#1} #2}                 % Gradient of #2 w.r.t. #1
\newcommand{\Div}[1]{\nabla \cdot #1}                  % Divergence
\newcommand{\Curl}[1]{\nabla \times #1}                % Curl
\newcommand{\Lap}[1]{\nabla^2 #1}                      % Laplacian
\newcommand{\Hess}[1]{\bm{H}_{#1}}                     % Hessian of #1
\newcommand{\Jac}[1]{\bm{J}_{#1}}                      % Jacobian of #1

% Partial derivative shorthand
\newcommand{\pd}[2]{\frac{\partial #1}{\partial #2}}           % ∂f/∂x
\newcommand{\pdd}[2]{\frac{\partial^2 #1}{\partial #2^2}}      % ∂²f/∂x²
\newcommand{\pdm}[3]{\frac{\partial^2 #1}{\partial #2\,\partial #3}}  % ∂²f/∂x∂y

% Ordinary derivative shorthand
\newcommand{\od}[2]{\frac{d #1}{d #2}}                         % df/dx
\newcommand{\odd}[2]{\frac{d^2 #1}{d #2^2}}                    % d²f/dx²

% Integrals
\newcommand{\dx}{\,dx}
\newcommand{\dy}{\,dy}
\newcommand{\dz}{\,dz}
\newcommand{\dt}{\,dt}
\newcommand{\dA}{\,dA}
\newcommand{\dV}{\,dV}
\newcommand{\ds}{\,ds}
\newcommand{\dvx}{\,d\vx}              % Vector integral element

% --- Statistical notation ---
\newcommand{\E}[1]{\mathbb{E}\!\left[#1\right]}          % Expectation
\newcommand{\Var}[1]{\operatorname{Var}\!\left(#1\right)} % Variance
\newcommand{\Cov}[2]{\operatorname{Cov}\!\left(#1,\,#2\right)} % Covariance
\newcommand{\Cor}[2]{\operatorname{Cor}\!\left(#1,\,#2\right)} % Correlation

\newcommand{\given}{\,\vert\,}                           % Conditional bar
\newcommand{\iid}{\overset{\text{iid}}{\sim}}           % iid ~
\newcommand{\ind}{\overset{\text{ind}}{\sim}}           % ind ~

% Distributions
\newcommand{\Normal}{\mathcal{N}}                        % Normal distribution
\newcommand{\MVN}[2]{\mathcal{N}\!\left(#1,\,#2\right)} % Multivariate normal
\newcommand{\Bern}{\operatorname{Bernoulli}}
\newcommand{\Bin}{\operatorname{Binomial}}
\newcommand{\Pois}{\operatorname{Poisson}}
\newcommand{\Unif}{\operatorname{Uniform}}
\newcommand{\Gam}{\operatorname{Gamma}}
\newcommand{\Beta}{\operatorname{Beta}}

% Likelihood & inference
\newcommand{\Lik}{\mathcal{L}}                          % Likelihood
\newcommand{\loglik}{\ell}                              % Log-likelihood
\newcommand{\Score}[1]{s(#1)}                           % Score function
\newcommand{\FI}{\mathcal{I}}                           % Fisher information
\newcommand{\MLE}[1]{\hat{#1}_{\text{MLE}}}            % MLE estimator
\newcommand{\MAP}[1]{\hat{#1}_{\text{MAP}}}            % MAP estimator
\newcommand{\posterior}{p(\vtheta \given \vx)}          % Posterior shorthand
\newcommand{\prior}{p(\vtheta)}                         % Prior shorthand
\newcommand{\likelihood}{p(\vx \given \vtheta)}        % Likelihood shorthand
\newcommand{\KL}[2]{D_{\text{KL}}\!\left(#1\,\|\,#2\right)} % KL divergence

% --- Linear algebra ---
\newcommand{\tr}[1]{\operatorname{tr}\!\left(#1\right)}        % Trace
\newcommand{\rank}[1]{\operatorname{rank}\!\left(#1\right)}    % Rank
\newcommand{\diag}[1]{\operatorname{diag}\!\left(#1\right)}    % Diagonal matrix
\newcommand{\spn}[1]{\operatorname{span}\!\left\{#1\right\}}   % Span
\newcommand{\norm}[1]{\left\|#1\right\|}                       % Norm
\newcommand{\abs}[1]{\left|#1\right|}                          % Absolute value
\newcommand{\inner}[2]{\langle #1,\, #2 \rangle}               % Inner product
\newcommand{\T}{^{\top}}                                        % Transpose superscript

% Positive (semi-)definiteness shorthand
\newcommand{\psd}{\succeq 0}    % Positive semi-definite
\newcommand{\pd}{\succ 0}       % Positive definite (Note: redefines \pd — see below)
% Note: \pd is overloaded: partial derivative above, positive definite here.
% In ambiguous contexts, use \succ 0 directly for positive definite.

% --- Miscellaneous ---
\newcommand{\defeq}{\coloneqq}                          % Definition equality :=
\newcommand{\st}{\text{ s.t. }}                         % "such that"
\newcommand{\approxeq}{\approx}                         % Approximate equality
\newcommand{\BigO}[1]{\mathcal{O}\!\left(#1\right)}    % Big-O notation
\newcommand{\sigmoid}[1]{\sigma\!\left(#1\right)}      % Sigmoid function
\newcommand{\softmax}[1]{\operatorname{softmax}\!\left(#1\right)}


% ------------------------------------------------------------
% DOCUMENT METADATA COMMANDS
% (Set these in each individual exercise file)
% ------------------------------------------------------------
%
%   \newcommand{\coursemodule}{Module 02, Unit 03}
%   \newcommand{\coursetitle}{The Multivariable Chain Rule}
%   \newcommand{\coursedate}{February 2026}
%
% These are used in the header/footer and title block.
% Define them BEFORE % ============================================================
% latex-preamble.tex — Threat Surfaces: Multivariable Calculus for AI Security
% fischer³ Education
%
% PURPOSE:
%   Shared preamble included by all exercise and solution files.
%   Defines packages, page layout, theorem environments, and
%   course-specific notation macros.
%
% USAGE:
%   In any exercise or solution .tex file, place the following
%   at the top of the document before \begin{document}:
%
%     % ============================================================
% latex-preamble.tex — Threat Surfaces: Multivariable Calculus for AI Security
% fischer³ Education
%
% PURPOSE:
%   Shared preamble included by all exercise and solution files.
%   Defines packages, page layout, theorem environments, and
%   course-specific notation macros.
%
% USAGE:
%   In any exercise or solution .tex file, place the following
%   at the top of the document before \begin{document}:
%
%     \input{../../assets/latex-preamble.tex}
%
%   Adjust the relative path as needed for the file's location
%   within the repository.
% ============================================================


% ------------------------------------------------------------
% PACKAGES
% ------------------------------------------------------------

% Page layout
\usepackage[
  margin=1.25in,
  top=1.0in,
  bottom=1.0in,
  headheight=14pt
]{geometry}

% Typography & encoding
\usepackage[T1]{fontenc}
\usepackage[utf8]{inputenc}
\usepackage{microtype}                % Improved justification
\usepackage{lmodern}                  % Scalable CM fonts

% Mathematics
\usepackage{amsmath}                  % Core math environments
\usepackage{amssymb}                  % Extended symbol set
\usepackage{amsthm}                   % Theorem environments
\usepackage{mathtools}                % Extensions to amsmath
\usepackage{bm}                       % Bold math (\bm)
\usepackage{cancel}                   % Strikethrough in math

% Graphics & color
\usepackage{graphicx}
\usepackage{xcolor}

% Course color palette
\definecolor{ts-blue}{RGB}{30, 100, 180}      % Primary — headings, links
\definecolor{ts-amber}{RGB}{200, 120, 20}     % Accent — warnings, highlights
\definecolor{ts-green}{RGB}{30, 140, 80}      % Statistical bridges
\definecolor{ts-gray}{RGB}{100, 100, 110}     % Secondary text
\definecolor{ts-light}{RGB}{245, 247, 250}    % Background tints

% Hyperlinks
\usepackage{hyperref}
\hypersetup{
  colorlinks=true,
  linkcolor=ts-blue,
  citecolor=ts-blue,
  urlcolor=ts-blue,
  pdfauthor={fischer\textsuperscript{3} Education},
  pdfcreator={LaTeX}
}

% Headers & footers
\usepackage{fancyhdr}
\pagestyle{fancy}
\fancyhf{}
\renewcommand{\headrulewidth}{0.4pt}
\fancyhead[L]{\small\color{ts-gray}\textit{Threat Surfaces}}
\fancyhead[R]{\small\color{ts-gray}\textit{\coursemodule}}   % Set \coursemodule in each file
\fancyfoot[C]{\small\color{ts-gray}\thepage}

% Tables
\usepackage{booktabs}
\usepackage{array}

% Lists
\usepackage{enumitem}
\setlist[enumerate,1]{label=\textbf{\arabic*.}}
\setlist[enumerate,2]{label=(\alph*)}

% Boxes & frames
\usepackage{tcolorbox}
\tcbuselibrary{breakable, skins, theorems}

% Code listings (for Python lab cross-references)
\usepackage{listings}
\lstset{
  basicstyle=\ttfamily\small,
  keywordstyle=\color{ts-blue},
  commentstyle=\color{ts-gray},
  stringstyle=\color{ts-green},
  numbers=left,
  numberstyle=\tiny\color{ts-gray},
  frame=single,
  breaklines=true,
  backgroundcolor=\color{ts-light}
}


% ------------------------------------------------------------
% THEOREM ENVIRONMENTS
% ------------------------------------------------------------

\theoremstyle{definition}
\newtheorem{definition}{Definition}[section]
\newtheorem{example}{Example}[section]
\newtheorem{exercise}{Exercise}[section]

\theoremstyle{plain}
\newtheorem{theorem}{Theorem}[section]
\newtheorem{lemma}[theorem]{Lemma}
\newtheorem{proposition}[theorem]{Proposition}
\newtheorem{corollary}[theorem]{Corollary}

\theoremstyle{remark}
\newtheorem*{remark}{Remark}
\newtheorem*{note}{Note}


% ------------------------------------------------------------
% TCOLORBOX ENVIRONMENTS
% ------------------------------------------------------------

% Problem statement box
\newtcolorbox{problembox}[1][]{
  enhanced,
  breakable,
  colback=ts-light,
  colframe=ts-blue,
  title={Problem},
  fonttitle=\bfseries,
  #1
}

% Statistical bridge box (green accent)
\newtcolorbox{bridgebox}[1][]{
  enhanced,
  breakable,
  colback=ts-light,
  colframe=ts-green,
  title={Statistical Bridge},
  fonttitle=\bfseries\color{ts-green},
  #1
}

% Warning / watch out box (amber accent)
\newtcolorbox{warningbox}[1][]{
  enhanced,
  breakable,
  colback=ts-light,
  colframe=ts-amber,
  title={Watch Out},
  fonttitle=\bfseries\color{ts-amber},
  #1
}

% Hint box (for exercise files — only appears in solution sets)
\newtcolorbox{hintbox}[1][]{
  enhanced,
  breakable,
  colback=ts-light,
  colframe=ts-gray,
  title={Hint},
  fonttitle=\bfseries\color{ts-gray},
  #1
}


% ------------------------------------------------------------
% NOTATION MACROS
% ------------------------------------------------------------

% --- Number systems ---
\newcommand{\R}{\mathbb{R}}             % Real numbers
\newcommand{\N}{\mathbb{N}}             % Natural numbers
\newcommand{\Z}{\mathbb{Z}}             % Integers
\newcommand{\Q}{\mathbb{Q}}             % Rationals
\newcommand{\C}{\mathbb{C}}             % Complex numbers

% --- Vectors & matrices ---
% Convention: bold lowercase for vectors, bold uppercase for matrices
\newcommand{\va}{\bm{a}}
\newcommand{\vb}{\bm{b}}
\newcommand{\vc}{\bm{c}}
\newcommand{\vd}{\bm{d}}
\newcommand{\ve}{\bm{e}}
\newcommand{\vf}{\bm{f}}
\newcommand{\vg}{\bm{g}}
\newcommand{\vh}{\bm{h}}
\newcommand{\vi}{\bm{i}}               % Unit vector i
\newcommand{\vj}{\bm{j}}               % Unit vector j
\newcommand{\vk}{\bm{k}}               % Unit vector k
\newcommand{\vn}{\bm{n}}               % Normal vector
\newcommand{\vr}{\bm{r}}               % Position vector
\newcommand{\vu}{\bm{u}}               % Unit direction vector
\newcommand{\vv}{\bm{v}}
\newcommand{\vw}{\bm{w}}
\newcommand{\vx}{\bm{x}}               % Input vector
\newcommand{\vy}{\bm{y}}               % Output / observation vector
\newcommand{\vz}{\bm{z}}               % Latent vector
\newcommand{\vzero}{\bm{0}}            % Zero vector

% Statistical vectors
\newcommand{\vmu}{\bm{\mu}}            % Mean vector
\newcommand{\vbeta}{\bm{\beta}}        % Regression coefficients
\newcommand{\vtheta}{\bm{\theta}}      % Parameter vector
\newcommand{\veps}{\bm{\varepsilon}}   % Error vector

% Matrices
\newcommand{\mA}{\bm{A}}
\newcommand{\mB}{\bm{B}}
\newcommand{\mC}{\bm{C}}
\newcommand{\mH}{\bm{H}}               % Hessian (also \Hess below)
\newcommand{\mI}{\bm{I}}               % Identity matrix
\newcommand{\mJ}{\bm{J}}               % Jacobian (also \Jac below)
\newcommand{\mK}{\bm{K}}               % Kernel matrix
\newcommand{\mL}{\bm{L}}               % Lower triangular / Cholesky
\newcommand{\mQ}{\bm{Q}}               % Orthogonal matrix
\newcommand{\mR}{\bm{R}}               % Upper triangular
\newcommand{\mS}{\bm{S}}               % Sample covariance
\newcommand{\mSigma}{\bm{\Sigma}}      % Covariance matrix
\newcommand{\mX}{\bm{X}}               % Design matrix
\newcommand{\mY}{\bm{Y}}

% --- Calculus operators ---
\newcommand{\grad}{\nabla}                              % Gradient ∇
\newcommand{\Grad}[1]{\nabla #1}                        % Gradient of #1
\newcommand{\GradP}[2]{\nabla_{#1} #2}                 % Gradient of #2 w.r.t. #1
\newcommand{\Div}[1]{\nabla \cdot #1}                  % Divergence
\newcommand{\Curl}[1]{\nabla \times #1}                % Curl
\newcommand{\Lap}[1]{\nabla^2 #1}                      % Laplacian
\newcommand{\Hess}[1]{\bm{H}_{#1}}                     % Hessian of #1
\newcommand{\Jac}[1]{\bm{J}_{#1}}                      % Jacobian of #1

% Partial derivative shorthand
\newcommand{\pd}[2]{\frac{\partial #1}{\partial #2}}           % ∂f/∂x
\newcommand{\pdd}[2]{\frac{\partial^2 #1}{\partial #2^2}}      % ∂²f/∂x²
\newcommand{\pdm}[3]{\frac{\partial^2 #1}{\partial #2\,\partial #3}}  % ∂²f/∂x∂y

% Ordinary derivative shorthand
\newcommand{\od}[2]{\frac{d #1}{d #2}}                         % df/dx
\newcommand{\odd}[2]{\frac{d^2 #1}{d #2^2}}                    % d²f/dx²

% Integrals
\newcommand{\dx}{\,dx}
\newcommand{\dy}{\,dy}
\newcommand{\dz}{\,dz}
\newcommand{\dt}{\,dt}
\newcommand{\dA}{\,dA}
\newcommand{\dV}{\,dV}
\newcommand{\ds}{\,ds}
\newcommand{\dvx}{\,d\vx}              % Vector integral element

% --- Statistical notation ---
\newcommand{\E}[1]{\mathbb{E}\!\left[#1\right]}          % Expectation
\newcommand{\Var}[1]{\operatorname{Var}\!\left(#1\right)} % Variance
\newcommand{\Cov}[2]{\operatorname{Cov}\!\left(#1,\,#2\right)} % Covariance
\newcommand{\Cor}[2]{\operatorname{Cor}\!\left(#1,\,#2\right)} % Correlation

\newcommand{\given}{\,\vert\,}                           % Conditional bar
\newcommand{\iid}{\overset{\text{iid}}{\sim}}           % iid ~
\newcommand{\ind}{\overset{\text{ind}}{\sim}}           % ind ~

% Distributions
\newcommand{\Normal}{\mathcal{N}}                        % Normal distribution
\newcommand{\MVN}[2]{\mathcal{N}\!\left(#1,\,#2\right)} % Multivariate normal
\newcommand{\Bern}{\operatorname{Bernoulli}}
\newcommand{\Bin}{\operatorname{Binomial}}
\newcommand{\Pois}{\operatorname{Poisson}}
\newcommand{\Unif}{\operatorname{Uniform}}
\newcommand{\Gam}{\operatorname{Gamma}}
\newcommand{\Beta}{\operatorname{Beta}}

% Likelihood & inference
\newcommand{\Lik}{\mathcal{L}}                          % Likelihood
\newcommand{\loglik}{\ell}                              % Log-likelihood
\newcommand{\Score}[1]{s(#1)}                           % Score function
\newcommand{\FI}{\mathcal{I}}                           % Fisher information
\newcommand{\MLE}[1]{\hat{#1}_{\text{MLE}}}            % MLE estimator
\newcommand{\MAP}[1]{\hat{#1}_{\text{MAP}}}            % MAP estimator
\newcommand{\posterior}{p(\vtheta \given \vx)}          % Posterior shorthand
\newcommand{\prior}{p(\vtheta)}                         % Prior shorthand
\newcommand{\likelihood}{p(\vx \given \vtheta)}        % Likelihood shorthand
\newcommand{\KL}[2]{D_{\text{KL}}\!\left(#1\,\|\,#2\right)} % KL divergence

% --- Linear algebra ---
\newcommand{\tr}[1]{\operatorname{tr}\!\left(#1\right)}        % Trace
\newcommand{\rank}[1]{\operatorname{rank}\!\left(#1\right)}    % Rank
\newcommand{\diag}[1]{\operatorname{diag}\!\left(#1\right)}    % Diagonal matrix
\newcommand{\spn}[1]{\operatorname{span}\!\left\{#1\right\}}   % Span
\newcommand{\norm}[1]{\left\|#1\right\|}                       % Norm
\newcommand{\abs}[1]{\left|#1\right|}                          % Absolute value
\newcommand{\inner}[2]{\langle #1,\, #2 \rangle}               % Inner product
\newcommand{\T}{^{\top}}                                        % Transpose superscript

% Positive (semi-)definiteness shorthand
\newcommand{\psd}{\succeq 0}    % Positive semi-definite
\newcommand{\pd}{\succ 0}       % Positive definite (Note: redefines \pd — see below)
% Note: \pd is overloaded: partial derivative above, positive definite here.
% In ambiguous contexts, use \succ 0 directly for positive definite.

% --- Miscellaneous ---
\newcommand{\defeq}{\coloneqq}                          % Definition equality :=
\newcommand{\st}{\text{ s.t. }}                         % "such that"
\newcommand{\approxeq}{\approx}                         % Approximate equality
\newcommand{\BigO}[1]{\mathcal{O}\!\left(#1\right)}    % Big-O notation
\newcommand{\sigmoid}[1]{\sigma\!\left(#1\right)}      % Sigmoid function
\newcommand{\softmax}[1]{\operatorname{softmax}\!\left(#1\right)}


% ------------------------------------------------------------
% DOCUMENT METADATA COMMANDS
% (Set these in each individual exercise file)
% ------------------------------------------------------------
%
%   \newcommand{\coursemodule}{Module 02, Unit 03}
%   \newcommand{\coursetitle}{The Multivariable Chain Rule}
%   \newcommand{\coursedate}{February 2026}
%
% These are used in the header/footer and title block.
% Define them BEFORE \input{...latex-preamble.tex} or immediately after.
%
% Default definitions (overridden per-file):
\providecommand{\coursemodule}{[Module TBD]}
\providecommand{\coursetitle}{[Unit Title TBD]}
\providecommand{\coursedate}{\today}


% ------------------------------------------------------------
% TITLE BLOCK MACRO
% ------------------------------------------------------------
%
% Usage in exercise files:
%   \maketitleblock
%
\newcommand{\maketitleblock}{%
  \begin{center}
    {\LARGE\bfseries\color{ts-blue} Threat Surfaces}\\[4pt]
    {\large Multivariable Calculus for AI Security}\\[2pt]
    {\small\color{ts-gray} fischer\textsuperscript{3} Education}\\[12pt]
    \rule{\linewidth}{0.4pt}\\[6pt]
    {\large\bfseries \coursemodule: \coursetitle}\\[4pt]
    {\small\color{ts-gray} Exercises \textbar\ \coursedate}\\[4pt]
    \rule{\linewidth}{0.4pt}
  \end{center}
  \vspace{12pt}
}


% ============================================================
% END OF PREAMBLE
% ============================================================

%
%   Adjust the relative path as needed for the file's location
%   within the repository.
% ============================================================


% ------------------------------------------------------------
% PACKAGES
% ------------------------------------------------------------

% Page layout
\usepackage[
  margin=1.25in,
  top=1.0in,
  bottom=1.0in,
  headheight=14pt
]{geometry}

% Typography & encoding
\usepackage[T1]{fontenc}
\usepackage[utf8]{inputenc}
\usepackage{microtype}                % Improved justification
\usepackage{lmodern}                  % Scalable CM fonts

% Mathematics
\usepackage{amsmath}                  % Core math environments
\usepackage{amssymb}                  % Extended symbol set
\usepackage{amsthm}                   % Theorem environments
\usepackage{mathtools}                % Extensions to amsmath
\usepackage{bm}                       % Bold math (\bm)
\usepackage{cancel}                   % Strikethrough in math

% Graphics & color
\usepackage{graphicx}
\usepackage{xcolor}

% Course color palette
\definecolor{ts-blue}{RGB}{30, 100, 180}      % Primary — headings, links
\definecolor{ts-amber}{RGB}{200, 120, 20}     % Accent — warnings, highlights
\definecolor{ts-green}{RGB}{30, 140, 80}      % Statistical bridges
\definecolor{ts-gray}{RGB}{100, 100, 110}     % Secondary text
\definecolor{ts-light}{RGB}{245, 247, 250}    % Background tints

% Hyperlinks
\usepackage{hyperref}
\hypersetup{
  colorlinks=true,
  linkcolor=ts-blue,
  citecolor=ts-blue,
  urlcolor=ts-blue,
  pdfauthor={fischer\textsuperscript{3} Education},
  pdfcreator={LaTeX}
}

% Headers & footers
\usepackage{fancyhdr}
\pagestyle{fancy}
\fancyhf{}
\renewcommand{\headrulewidth}{0.4pt}
\fancyhead[L]{\small\color{ts-gray}\textit{Threat Surfaces}}
\fancyhead[R]{\small\color{ts-gray}\textit{\coursemodule}}   % Set \coursemodule in each file
\fancyfoot[C]{\small\color{ts-gray}\thepage}

% Tables
\usepackage{booktabs}
\usepackage{array}

% Lists
\usepackage{enumitem}
\setlist[enumerate,1]{label=\textbf{\arabic*.}}
\setlist[enumerate,2]{label=(\alph*)}

% Boxes & frames
\usepackage{tcolorbox}
\tcbuselibrary{breakable, skins, theorems}

% Code listings (for Python lab cross-references)
\usepackage{listings}
\lstset{
  basicstyle=\ttfamily\small,
  keywordstyle=\color{ts-blue},
  commentstyle=\color{ts-gray},
  stringstyle=\color{ts-green},
  numbers=left,
  numberstyle=\tiny\color{ts-gray},
  frame=single,
  breaklines=true,
  backgroundcolor=\color{ts-light}
}


% ------------------------------------------------------------
% THEOREM ENVIRONMENTS
% ------------------------------------------------------------

\theoremstyle{definition}
\newtheorem{definition}{Definition}[section]
\newtheorem{example}{Example}[section]
\newtheorem{exercise}{Exercise}[section]

\theoremstyle{plain}
\newtheorem{theorem}{Theorem}[section]
\newtheorem{lemma}[theorem]{Lemma}
\newtheorem{proposition}[theorem]{Proposition}
\newtheorem{corollary}[theorem]{Corollary}

\theoremstyle{remark}
\newtheorem*{remark}{Remark}
\newtheorem*{note}{Note}


% ------------------------------------------------------------
% TCOLORBOX ENVIRONMENTS
% ------------------------------------------------------------

% Problem statement box
\newtcolorbox{problembox}[1][]{
  enhanced,
  breakable,
  colback=ts-light,
  colframe=ts-blue,
  title={Problem},
  fonttitle=\bfseries,
  #1
}

% Statistical bridge box (green accent)
\newtcolorbox{bridgebox}[1][]{
  enhanced,
  breakable,
  colback=ts-light,
  colframe=ts-green,
  title={Statistical Bridge},
  fonttitle=\bfseries\color{ts-green},
  #1
}

% Warning / watch out box (amber accent)
\newtcolorbox{warningbox}[1][]{
  enhanced,
  breakable,
  colback=ts-light,
  colframe=ts-amber,
  title={Watch Out},
  fonttitle=\bfseries\color{ts-amber},
  #1
}

% Hint box (for exercise files — only appears in solution sets)
\newtcolorbox{hintbox}[1][]{
  enhanced,
  breakable,
  colback=ts-light,
  colframe=ts-gray,
  title={Hint},
  fonttitle=\bfseries\color{ts-gray},
  #1
}


% ------------------------------------------------------------
% NOTATION MACROS
% ------------------------------------------------------------

% --- Number systems ---
\newcommand{\R}{\mathbb{R}}             % Real numbers
\newcommand{\N}{\mathbb{N}}             % Natural numbers
\newcommand{\Z}{\mathbb{Z}}             % Integers
\newcommand{\Q}{\mathbb{Q}}             % Rationals
\newcommand{\C}{\mathbb{C}}             % Complex numbers

% --- Vectors & matrices ---
% Convention: bold lowercase for vectors, bold uppercase for matrices
\newcommand{\va}{\bm{a}}
\newcommand{\vb}{\bm{b}}
\newcommand{\vc}{\bm{c}}
\newcommand{\vd}{\bm{d}}
\newcommand{\ve}{\bm{e}}
\newcommand{\vf}{\bm{f}}
\newcommand{\vg}{\bm{g}}
\newcommand{\vh}{\bm{h}}
\newcommand{\vi}{\bm{i}}               % Unit vector i
\newcommand{\vj}{\bm{j}}               % Unit vector j
\newcommand{\vk}{\bm{k}}               % Unit vector k
\newcommand{\vn}{\bm{n}}               % Normal vector
\newcommand{\vr}{\bm{r}}               % Position vector
\newcommand{\vu}{\bm{u}}               % Unit direction vector
\newcommand{\vv}{\bm{v}}
\newcommand{\vw}{\bm{w}}
\newcommand{\vx}{\bm{x}}               % Input vector
\newcommand{\vy}{\bm{y}}               % Output / observation vector
\newcommand{\vz}{\bm{z}}               % Latent vector
\newcommand{\vzero}{\bm{0}}            % Zero vector

% Statistical vectors
\newcommand{\vmu}{\bm{\mu}}            % Mean vector
\newcommand{\vbeta}{\bm{\beta}}        % Regression coefficients
\newcommand{\vtheta}{\bm{\theta}}      % Parameter vector
\newcommand{\veps}{\bm{\varepsilon}}   % Error vector

% Matrices
\newcommand{\mA}{\bm{A}}
\newcommand{\mB}{\bm{B}}
\newcommand{\mC}{\bm{C}}
\newcommand{\mH}{\bm{H}}               % Hessian (also \Hess below)
\newcommand{\mI}{\bm{I}}               % Identity matrix
\newcommand{\mJ}{\bm{J}}               % Jacobian (also \Jac below)
\newcommand{\mK}{\bm{K}}               % Kernel matrix
\newcommand{\mL}{\bm{L}}               % Lower triangular / Cholesky
\newcommand{\mQ}{\bm{Q}}               % Orthogonal matrix
\newcommand{\mR}{\bm{R}}               % Upper triangular
\newcommand{\mS}{\bm{S}}               % Sample covariance
\newcommand{\mSigma}{\bm{\Sigma}}      % Covariance matrix
\newcommand{\mX}{\bm{X}}               % Design matrix
\newcommand{\mY}{\bm{Y}}

% --- Calculus operators ---
\newcommand{\grad}{\nabla}                              % Gradient ∇
\newcommand{\Grad}[1]{\nabla #1}                        % Gradient of #1
\newcommand{\GradP}[2]{\nabla_{#1} #2}                 % Gradient of #2 w.r.t. #1
\newcommand{\Div}[1]{\nabla \cdot #1}                  % Divergence
\newcommand{\Curl}[1]{\nabla \times #1}                % Curl
\newcommand{\Lap}[1]{\nabla^2 #1}                      % Laplacian
\newcommand{\Hess}[1]{\bm{H}_{#1}}                     % Hessian of #1
\newcommand{\Jac}[1]{\bm{J}_{#1}}                      % Jacobian of #1

% Partial derivative shorthand
\newcommand{\pd}[2]{\frac{\partial #1}{\partial #2}}           % ∂f/∂x
\newcommand{\pdd}[2]{\frac{\partial^2 #1}{\partial #2^2}}      % ∂²f/∂x²
\newcommand{\pdm}[3]{\frac{\partial^2 #1}{\partial #2\,\partial #3}}  % ∂²f/∂x∂y

% Ordinary derivative shorthand
\newcommand{\od}[2]{\frac{d #1}{d #2}}                         % df/dx
\newcommand{\odd}[2]{\frac{d^2 #1}{d #2^2}}                    % d²f/dx²

% Integrals
\newcommand{\dx}{\,dx}
\newcommand{\dy}{\,dy}
\newcommand{\dz}{\,dz}
\newcommand{\dt}{\,dt}
\newcommand{\dA}{\,dA}
\newcommand{\dV}{\,dV}
\newcommand{\ds}{\,ds}
\newcommand{\dvx}{\,d\vx}              % Vector integral element

% --- Statistical notation ---
\newcommand{\E}[1]{\mathbb{E}\!\left[#1\right]}          % Expectation
\newcommand{\Var}[1]{\operatorname{Var}\!\left(#1\right)} % Variance
\newcommand{\Cov}[2]{\operatorname{Cov}\!\left(#1,\,#2\right)} % Covariance
\newcommand{\Cor}[2]{\operatorname{Cor}\!\left(#1,\,#2\right)} % Correlation

\newcommand{\given}{\,\vert\,}                           % Conditional bar
\newcommand{\iid}{\overset{\text{iid}}{\sim}}           % iid ~
\newcommand{\ind}{\overset{\text{ind}}{\sim}}           % ind ~

% Distributions
\newcommand{\Normal}{\mathcal{N}}                        % Normal distribution
\newcommand{\MVN}[2]{\mathcal{N}\!\left(#1,\,#2\right)} % Multivariate normal
\newcommand{\Bern}{\operatorname{Bernoulli}}
\newcommand{\Bin}{\operatorname{Binomial}}
\newcommand{\Pois}{\operatorname{Poisson}}
\newcommand{\Unif}{\operatorname{Uniform}}
\newcommand{\Gam}{\operatorname{Gamma}}
\newcommand{\Beta}{\operatorname{Beta}}

% Likelihood & inference
\newcommand{\Lik}{\mathcal{L}}                          % Likelihood
\newcommand{\loglik}{\ell}                              % Log-likelihood
\newcommand{\Score}[1]{s(#1)}                           % Score function
\newcommand{\FI}{\mathcal{I}}                           % Fisher information
\newcommand{\MLE}[1]{\hat{#1}_{\text{MLE}}}            % MLE estimator
\newcommand{\MAP}[1]{\hat{#1}_{\text{MAP}}}            % MAP estimator
\newcommand{\posterior}{p(\vtheta \given \vx)}          % Posterior shorthand
\newcommand{\prior}{p(\vtheta)}                         % Prior shorthand
\newcommand{\likelihood}{p(\vx \given \vtheta)}        % Likelihood shorthand
\newcommand{\KL}[2]{D_{\text{KL}}\!\left(#1\,\|\,#2\right)} % KL divergence

% --- Linear algebra ---
\newcommand{\tr}[1]{\operatorname{tr}\!\left(#1\right)}        % Trace
\newcommand{\rank}[1]{\operatorname{rank}\!\left(#1\right)}    % Rank
\newcommand{\diag}[1]{\operatorname{diag}\!\left(#1\right)}    % Diagonal matrix
\newcommand{\spn}[1]{\operatorname{span}\!\left\{#1\right\}}   % Span
\newcommand{\norm}[1]{\left\|#1\right\|}                       % Norm
\newcommand{\abs}[1]{\left|#1\right|}                          % Absolute value
\newcommand{\inner}[2]{\langle #1,\, #2 \rangle}               % Inner product
\newcommand{\T}{^{\top}}                                        % Transpose superscript

% Positive (semi-)definiteness shorthand
\newcommand{\psd}{\succeq 0}    % Positive semi-definite
\newcommand{\pd}{\succ 0}       % Positive definite (Note: redefines \pd — see below)
% Note: \pd is overloaded: partial derivative above, positive definite here.
% In ambiguous contexts, use \succ 0 directly for positive definite.

% --- Miscellaneous ---
\newcommand{\defeq}{\coloneqq}                          % Definition equality :=
\newcommand{\st}{\text{ s.t. }}                         % "such that"
\newcommand{\approxeq}{\approx}                         % Approximate equality
\newcommand{\BigO}[1]{\mathcal{O}\!\left(#1\right)}    % Big-O notation
\newcommand{\sigmoid}[1]{\sigma\!\left(#1\right)}      % Sigmoid function
\newcommand{\softmax}[1]{\operatorname{softmax}\!\left(#1\right)}


% ------------------------------------------------------------
% DOCUMENT METADATA COMMANDS
% (Set these in each individual exercise file)
% ------------------------------------------------------------
%
%   \newcommand{\coursemodule}{Module 02, Unit 03}
%   \newcommand{\coursetitle}{The Multivariable Chain Rule}
%   \newcommand{\coursedate}{February 2026}
%
% These are used in the header/footer and title block.
% Define them BEFORE % ============================================================
% latex-preamble.tex — Threat Surfaces: Multivariable Calculus for AI Security
% fischer³ Education
%
% PURPOSE:
%   Shared preamble included by all exercise and solution files.
%   Defines packages, page layout, theorem environments, and
%   course-specific notation macros.
%
% USAGE:
%   In any exercise or solution .tex file, place the following
%   at the top of the document before \begin{document}:
%
%     \input{../../assets/latex-preamble.tex}
%
%   Adjust the relative path as needed for the file's location
%   within the repository.
% ============================================================


% ------------------------------------------------------------
% PACKAGES
% ------------------------------------------------------------

% Page layout
\usepackage[
  margin=1.25in,
  top=1.0in,
  bottom=1.0in,
  headheight=14pt
]{geometry}

% Typography & encoding
\usepackage[T1]{fontenc}
\usepackage[utf8]{inputenc}
\usepackage{microtype}                % Improved justification
\usepackage{lmodern}                  % Scalable CM fonts

% Mathematics
\usepackage{amsmath}                  % Core math environments
\usepackage{amssymb}                  % Extended symbol set
\usepackage{amsthm}                   % Theorem environments
\usepackage{mathtools}                % Extensions to amsmath
\usepackage{bm}                       % Bold math (\bm)
\usepackage{cancel}                   % Strikethrough in math

% Graphics & color
\usepackage{graphicx}
\usepackage{xcolor}

% Course color palette
\definecolor{ts-blue}{RGB}{30, 100, 180}      % Primary — headings, links
\definecolor{ts-amber}{RGB}{200, 120, 20}     % Accent — warnings, highlights
\definecolor{ts-green}{RGB}{30, 140, 80}      % Statistical bridges
\definecolor{ts-gray}{RGB}{100, 100, 110}     % Secondary text
\definecolor{ts-light}{RGB}{245, 247, 250}    % Background tints

% Hyperlinks
\usepackage{hyperref}
\hypersetup{
  colorlinks=true,
  linkcolor=ts-blue,
  citecolor=ts-blue,
  urlcolor=ts-blue,
  pdfauthor={fischer\textsuperscript{3} Education},
  pdfcreator={LaTeX}
}

% Headers & footers
\usepackage{fancyhdr}
\pagestyle{fancy}
\fancyhf{}
\renewcommand{\headrulewidth}{0.4pt}
\fancyhead[L]{\small\color{ts-gray}\textit{Threat Surfaces}}
\fancyhead[R]{\small\color{ts-gray}\textit{\coursemodule}}   % Set \coursemodule in each file
\fancyfoot[C]{\small\color{ts-gray}\thepage}

% Tables
\usepackage{booktabs}
\usepackage{array}

% Lists
\usepackage{enumitem}
\setlist[enumerate,1]{label=\textbf{\arabic*.}}
\setlist[enumerate,2]{label=(\alph*)}

% Boxes & frames
\usepackage{tcolorbox}
\tcbuselibrary{breakable, skins, theorems}

% Code listings (for Python lab cross-references)
\usepackage{listings}
\lstset{
  basicstyle=\ttfamily\small,
  keywordstyle=\color{ts-blue},
  commentstyle=\color{ts-gray},
  stringstyle=\color{ts-green},
  numbers=left,
  numberstyle=\tiny\color{ts-gray},
  frame=single,
  breaklines=true,
  backgroundcolor=\color{ts-light}
}


% ------------------------------------------------------------
% THEOREM ENVIRONMENTS
% ------------------------------------------------------------

\theoremstyle{definition}
\newtheorem{definition}{Definition}[section]
\newtheorem{example}{Example}[section]
\newtheorem{exercise}{Exercise}[section]

\theoremstyle{plain}
\newtheorem{theorem}{Theorem}[section]
\newtheorem{lemma}[theorem]{Lemma}
\newtheorem{proposition}[theorem]{Proposition}
\newtheorem{corollary}[theorem]{Corollary}

\theoremstyle{remark}
\newtheorem*{remark}{Remark}
\newtheorem*{note}{Note}


% ------------------------------------------------------------
% TCOLORBOX ENVIRONMENTS
% ------------------------------------------------------------

% Problem statement box
\newtcolorbox{problembox}[1][]{
  enhanced,
  breakable,
  colback=ts-light,
  colframe=ts-blue,
  title={Problem},
  fonttitle=\bfseries,
  #1
}

% Statistical bridge box (green accent)
\newtcolorbox{bridgebox}[1][]{
  enhanced,
  breakable,
  colback=ts-light,
  colframe=ts-green,
  title={Statistical Bridge},
  fonttitle=\bfseries\color{ts-green},
  #1
}

% Warning / watch out box (amber accent)
\newtcolorbox{warningbox}[1][]{
  enhanced,
  breakable,
  colback=ts-light,
  colframe=ts-amber,
  title={Watch Out},
  fonttitle=\bfseries\color{ts-amber},
  #1
}

% Hint box (for exercise files — only appears in solution sets)
\newtcolorbox{hintbox}[1][]{
  enhanced,
  breakable,
  colback=ts-light,
  colframe=ts-gray,
  title={Hint},
  fonttitle=\bfseries\color{ts-gray},
  #1
}


% ------------------------------------------------------------
% NOTATION MACROS
% ------------------------------------------------------------

% --- Number systems ---
\newcommand{\R}{\mathbb{R}}             % Real numbers
\newcommand{\N}{\mathbb{N}}             % Natural numbers
\newcommand{\Z}{\mathbb{Z}}             % Integers
\newcommand{\Q}{\mathbb{Q}}             % Rationals
\newcommand{\C}{\mathbb{C}}             % Complex numbers

% --- Vectors & matrices ---
% Convention: bold lowercase for vectors, bold uppercase for matrices
\newcommand{\va}{\bm{a}}
\newcommand{\vb}{\bm{b}}
\newcommand{\vc}{\bm{c}}
\newcommand{\vd}{\bm{d}}
\newcommand{\ve}{\bm{e}}
\newcommand{\vf}{\bm{f}}
\newcommand{\vg}{\bm{g}}
\newcommand{\vh}{\bm{h}}
\newcommand{\vi}{\bm{i}}               % Unit vector i
\newcommand{\vj}{\bm{j}}               % Unit vector j
\newcommand{\vk}{\bm{k}}               % Unit vector k
\newcommand{\vn}{\bm{n}}               % Normal vector
\newcommand{\vr}{\bm{r}}               % Position vector
\newcommand{\vu}{\bm{u}}               % Unit direction vector
\newcommand{\vv}{\bm{v}}
\newcommand{\vw}{\bm{w}}
\newcommand{\vx}{\bm{x}}               % Input vector
\newcommand{\vy}{\bm{y}}               % Output / observation vector
\newcommand{\vz}{\bm{z}}               % Latent vector
\newcommand{\vzero}{\bm{0}}            % Zero vector

% Statistical vectors
\newcommand{\vmu}{\bm{\mu}}            % Mean vector
\newcommand{\vbeta}{\bm{\beta}}        % Regression coefficients
\newcommand{\vtheta}{\bm{\theta}}      % Parameter vector
\newcommand{\veps}{\bm{\varepsilon}}   % Error vector

% Matrices
\newcommand{\mA}{\bm{A}}
\newcommand{\mB}{\bm{B}}
\newcommand{\mC}{\bm{C}}
\newcommand{\mH}{\bm{H}}               % Hessian (also \Hess below)
\newcommand{\mI}{\bm{I}}               % Identity matrix
\newcommand{\mJ}{\bm{J}}               % Jacobian (also \Jac below)
\newcommand{\mK}{\bm{K}}               % Kernel matrix
\newcommand{\mL}{\bm{L}}               % Lower triangular / Cholesky
\newcommand{\mQ}{\bm{Q}}               % Orthogonal matrix
\newcommand{\mR}{\bm{R}}               % Upper triangular
\newcommand{\mS}{\bm{S}}               % Sample covariance
\newcommand{\mSigma}{\bm{\Sigma}}      % Covariance matrix
\newcommand{\mX}{\bm{X}}               % Design matrix
\newcommand{\mY}{\bm{Y}}

% --- Calculus operators ---
\newcommand{\grad}{\nabla}                              % Gradient ∇
\newcommand{\Grad}[1]{\nabla #1}                        % Gradient of #1
\newcommand{\GradP}[2]{\nabla_{#1} #2}                 % Gradient of #2 w.r.t. #1
\newcommand{\Div}[1]{\nabla \cdot #1}                  % Divergence
\newcommand{\Curl}[1]{\nabla \times #1}                % Curl
\newcommand{\Lap}[1]{\nabla^2 #1}                      % Laplacian
\newcommand{\Hess}[1]{\bm{H}_{#1}}                     % Hessian of #1
\newcommand{\Jac}[1]{\bm{J}_{#1}}                      % Jacobian of #1

% Partial derivative shorthand
\newcommand{\pd}[2]{\frac{\partial #1}{\partial #2}}           % ∂f/∂x
\newcommand{\pdd}[2]{\frac{\partial^2 #1}{\partial #2^2}}      % ∂²f/∂x²
\newcommand{\pdm}[3]{\frac{\partial^2 #1}{\partial #2\,\partial #3}}  % ∂²f/∂x∂y

% Ordinary derivative shorthand
\newcommand{\od}[2]{\frac{d #1}{d #2}}                         % df/dx
\newcommand{\odd}[2]{\frac{d^2 #1}{d #2^2}}                    % d²f/dx²

% Integrals
\newcommand{\dx}{\,dx}
\newcommand{\dy}{\,dy}
\newcommand{\dz}{\,dz}
\newcommand{\dt}{\,dt}
\newcommand{\dA}{\,dA}
\newcommand{\dV}{\,dV}
\newcommand{\ds}{\,ds}
\newcommand{\dvx}{\,d\vx}              % Vector integral element

% --- Statistical notation ---
\newcommand{\E}[1]{\mathbb{E}\!\left[#1\right]}          % Expectation
\newcommand{\Var}[1]{\operatorname{Var}\!\left(#1\right)} % Variance
\newcommand{\Cov}[2]{\operatorname{Cov}\!\left(#1,\,#2\right)} % Covariance
\newcommand{\Cor}[2]{\operatorname{Cor}\!\left(#1,\,#2\right)} % Correlation

\newcommand{\given}{\,\vert\,}                           % Conditional bar
\newcommand{\iid}{\overset{\text{iid}}{\sim}}           % iid ~
\newcommand{\ind}{\overset{\text{ind}}{\sim}}           % ind ~

% Distributions
\newcommand{\Normal}{\mathcal{N}}                        % Normal distribution
\newcommand{\MVN}[2]{\mathcal{N}\!\left(#1,\,#2\right)} % Multivariate normal
\newcommand{\Bern}{\operatorname{Bernoulli}}
\newcommand{\Bin}{\operatorname{Binomial}}
\newcommand{\Pois}{\operatorname{Poisson}}
\newcommand{\Unif}{\operatorname{Uniform}}
\newcommand{\Gam}{\operatorname{Gamma}}
\newcommand{\Beta}{\operatorname{Beta}}

% Likelihood & inference
\newcommand{\Lik}{\mathcal{L}}                          % Likelihood
\newcommand{\loglik}{\ell}                              % Log-likelihood
\newcommand{\Score}[1]{s(#1)}                           % Score function
\newcommand{\FI}{\mathcal{I}}                           % Fisher information
\newcommand{\MLE}[1]{\hat{#1}_{\text{MLE}}}            % MLE estimator
\newcommand{\MAP}[1]{\hat{#1}_{\text{MAP}}}            % MAP estimator
\newcommand{\posterior}{p(\vtheta \given \vx)}          % Posterior shorthand
\newcommand{\prior}{p(\vtheta)}                         % Prior shorthand
\newcommand{\likelihood}{p(\vx \given \vtheta)}        % Likelihood shorthand
\newcommand{\KL}[2]{D_{\text{KL}}\!\left(#1\,\|\,#2\right)} % KL divergence

% --- Linear algebra ---
\newcommand{\tr}[1]{\operatorname{tr}\!\left(#1\right)}        % Trace
\newcommand{\rank}[1]{\operatorname{rank}\!\left(#1\right)}    % Rank
\newcommand{\diag}[1]{\operatorname{diag}\!\left(#1\right)}    % Diagonal matrix
\newcommand{\spn}[1]{\operatorname{span}\!\left\{#1\right\}}   % Span
\newcommand{\norm}[1]{\left\|#1\right\|}                       % Norm
\newcommand{\abs}[1]{\left|#1\right|}                          % Absolute value
\newcommand{\inner}[2]{\langle #1,\, #2 \rangle}               % Inner product
\newcommand{\T}{^{\top}}                                        % Transpose superscript

% Positive (semi-)definiteness shorthand
\newcommand{\psd}{\succeq 0}    % Positive semi-definite
\newcommand{\pd}{\succ 0}       % Positive definite (Note: redefines \pd — see below)
% Note: \pd is overloaded: partial derivative above, positive definite here.
% In ambiguous contexts, use \succ 0 directly for positive definite.

% --- Miscellaneous ---
\newcommand{\defeq}{\coloneqq}                          % Definition equality :=
\newcommand{\st}{\text{ s.t. }}                         % "such that"
\newcommand{\approxeq}{\approx}                         % Approximate equality
\newcommand{\BigO}[1]{\mathcal{O}\!\left(#1\right)}    % Big-O notation
\newcommand{\sigmoid}[1]{\sigma\!\left(#1\right)}      % Sigmoid function
\newcommand{\softmax}[1]{\operatorname{softmax}\!\left(#1\right)}


% ------------------------------------------------------------
% DOCUMENT METADATA COMMANDS
% (Set these in each individual exercise file)
% ------------------------------------------------------------
%
%   \newcommand{\coursemodule}{Module 02, Unit 03}
%   \newcommand{\coursetitle}{The Multivariable Chain Rule}
%   \newcommand{\coursedate}{February 2026}
%
% These are used in the header/footer and title block.
% Define them BEFORE \input{...latex-preamble.tex} or immediately after.
%
% Default definitions (overridden per-file):
\providecommand{\coursemodule}{[Module TBD]}
\providecommand{\coursetitle}{[Unit Title TBD]}
\providecommand{\coursedate}{\today}


% ------------------------------------------------------------
% TITLE BLOCK MACRO
% ------------------------------------------------------------
%
% Usage in exercise files:
%   \maketitleblock
%
\newcommand{\maketitleblock}{%
  \begin{center}
    {\LARGE\bfseries\color{ts-blue} Threat Surfaces}\\[4pt]
    {\large Multivariable Calculus for AI Security}\\[2pt]
    {\small\color{ts-gray} fischer\textsuperscript{3} Education}\\[12pt]
    \rule{\linewidth}{0.4pt}\\[6pt]
    {\large\bfseries \coursemodule: \coursetitle}\\[4pt]
    {\small\color{ts-gray} Exercises \textbar\ \coursedate}\\[4pt]
    \rule{\linewidth}{0.4pt}
  \end{center}
  \vspace{12pt}
}


% ============================================================
% END OF PREAMBLE
% ============================================================
 or immediately after.
%
% Default definitions (overridden per-file):
\providecommand{\coursemodule}{[Module TBD]}
\providecommand{\coursetitle}{[Unit Title TBD]}
\providecommand{\coursedate}{\today}


% ------------------------------------------------------------
% TITLE BLOCK MACRO
% ------------------------------------------------------------
%
% Usage in exercise files:
%   \maketitleblock
%
\newcommand{\maketitleblock}{%
  \begin{center}
    {\LARGE\bfseries\color{ts-blue} Threat Surfaces}\\[4pt]
    {\large Multivariable Calculus for AI Security}\\[2pt]
    {\small\color{ts-gray} fischer\textsuperscript{3} Education}\\[12pt]
    \rule{\linewidth}{0.4pt}\\[6pt]
    {\large\bfseries \coursemodule: \coursetitle}\\[4pt]
    {\small\color{ts-gray} Exercises \textbar\ \coursedate}\\[4pt]
    \rule{\linewidth}{0.4pt}
  \end{center}
  \vspace{12pt}
}


% ============================================================
% END OF PREAMBLE
% ============================================================
 or immediately after.
%
% Default definitions (overridden per-file):
\providecommand{\coursemodule}{[Module TBD]}
\providecommand{\coursetitle}{[Unit Title TBD]}
\providecommand{\coursedate}{\today}


% ------------------------------------------------------------
% TITLE BLOCK MACRO
% ------------------------------------------------------------
%
% Usage in exercise files:
%   \maketitleblock
%
\newcommand{\maketitleblock}{%
  \begin{center}
    {\LARGE\bfseries\color{ts-blue} Threat Surfaces}\\[4pt]
    {\large Multivariable Calculus for AI Security}\\[2pt]
    {\small\color{ts-gray} fischer\textsuperscript{3} Education}\\[12pt]
    \rule{\linewidth}{0.4pt}\\[6pt]
    {\large\bfseries \coursemodule: \coursetitle}\\[4pt]
    {\small\color{ts-gray} Exercises \textbar\ \coursedate}\\[4pt]
    \rule{\linewidth}{0.4pt}
  \end{center}
  \vspace{12pt}
}


% ============================================================
% END OF PREAMBLE
% ============================================================


% ============================================================
\begin{document}
% ============================================================

\maketitleblock

% ------------------------------------------------------------
\begin{tcolorbox}[
  colback=ts-light,
  colframe=ts-gray,
  title=Before You Begin,
  fonttitle=\bfseries
]
Complete \texttt{notes.md} for this unit before working these exercises.
Work all problems by hand. Use the Python lab (\texttt{python-lab.ipynb}) 
to verify your answers computationally \emph{after} you have committed to 
a hand solution.

\medskip
\textbf{Estimated time:} 20--30 minutes \quad
\textbf{Problems:} 4 \quad
\textbf{Difficulty:}
$\star$ Foundational \enspace
$\star\star$ Applied \enspace
$\star\star\star$ Extension
\end{tcolorbox}

\vspace{16pt}


% ============================================================
% PROBLEM 1 — Vector Arithmetic
% ============================================================
\begin{problembox}[title={Problem 1 \hfill $\star$}]
Let $\va = (2,\, -3,\, 1)\T$ and $\vb = (-1,\, 4,\, 2)\T$ in $\R^3$.
\end{problembox}

\begin{enumerate}
  \item Compute each of the following. Show your work componentwise.
    \begin{enumerate}
      \item $\va + \vb$
      \item $3\va - 2\vb$
      \item $\frac{1}{2}\va + \frac{3}{2}\vb$
    \end{enumerate}

  \item Compute the Euclidean norm $\norm{\va}$ and $\norm{\vb}$. Leave your answers in exact form (no decimal approximations).

  \item Normalize $\va$ to produce the unit vector $\hat{\va}$. Verify that $\norm{\hat{\va}} = 1$ by direct computation.

  \item Without computing, determine whether $\norm{\va + \vb} \leq \norm{\va} + \norm{\vb}$. 
  State the name of the inequality you are applying. Then verify by computing both sides.
\end{enumerate}

\vspace{20pt}


% ============================================================
% PROBLEM 2 — Linear Combinations & Span
% ============================================================
\begin{problembox}[title={Problem 2 \hfill $\star$}]
Let $\ve_1, \ve_2, \ve_3$ denote the standard basis vectors in $\R^3$.
\end{problembox}

\begin{enumerate}
  \item Express the vector $\vc = (5,\, 0,\, -7)\T$ as a linear combination of the standard basis vectors. Write out the full expression.

  \item Let $\vv_1 = (1,\, 1)\T$ and $\vv_2 = (1,\, -1)\T$ in $\R^2$. 
  Find scalars $c_1, c_2$ such that:
  $$
  \vu = \begin{pmatrix} 4 \\ 2 \end{pmatrix} = c_1 \vv_1 + c_2 \vv_2
  $$
  \emph{Hint}: Set up a $2 \times 2$ linear system and solve.

  \item Using the scalars found in part (b), verify your answer by computing 
  $c_1 \vv_1 + c_2 \vv_2$ directly and confirming it equals $\vu$.

  \item The vectors $\vv_1$ and $\vv_2$ from part (b) form a basis for $\R^2$.
  In one or two sentences, explain geometrically why any vector $\vu \in \R^2$ 
  can be expressed as a linear combination of these two vectors.
\end{enumerate}

\vspace{20pt}


% ============================================================
% PROBLEM 3 — Statistical Bridge
% ============================================================
\begin{problembox}[title={Problem 3 \hfill $\star\star$ — Statistical Bridge}]
In Ridge regression (also called $\ell_2$-regularized linear regression), 
we fit a model by minimizing the following penalized objective:

$$
J(\vbeta) = \norm{\vy - \mX\vbeta}^2 + \lambda\norm{\vbeta}^2
$$

where:
\begin{itemize}
  \item $\vy \in \R^n$ is the vector of observed responses
  \item $\mX \in \R^{n \times p}$ is the design matrix (each row is one observation)
  \item $\vbeta \in \R^p$ is the parameter vector we are estimating
  \item $\lambda \geq 0$ is the regularization strength
\end{itemize}

For this problem, work in a simplified setting with $p = 2$ parameters:
$\vbeta = (\beta_1,\, \beta_2)\T$.
\end{problembox}

\begin{enumerate}
  \item Write out $\norm{\vbeta}^2$ explicitly in terms of $\beta_1$ and $\beta_2$.
  This is the \textbf{regularization penalty}.

  \item Suppose the current parameter estimate is $\vbeta = (3,\, -4)\T$. Compute:
    \begin{enumerate}
      \item The regularization penalty $\norm{\vbeta}^2$
      \item The Euclidean norm $\norm{\vbeta}$
      \item The unit vector $\hat{\vbeta}$ pointing in the same direction as $\vbeta$
    \end{enumerate}

  \item Now suppose $\lambda = 2$. The scalar $\lambda$ controls how strongly 
  we penalize large values of $\norm{\vbeta}$. In one or two sentences, 
  describe geometrically what happens to the set of 
  ``acceptable'' $\vbeta$ vectors as $\lambda \to \infty$.

  \item Consider two competing parameter estimates:
  $$
  \vbeta^{(1)} = \begin{pmatrix}5\\0\end{pmatrix}, \qquad \vbeta^{(2)} = \begin{pmatrix}3\\4\end{pmatrix}
  $$
  Both have the same $\ell_2$ norm. Verify this. What does this tell you about 
  the regularization penalty — does it distinguish between these two estimates?
\end{enumerate}

\vspace{20pt}


% ============================================================
% PROBLEM 4 — Extension
% ============================================================
\begin{problembox}[title={Problem 4 \hfill $\star\star\star$ — Extension (Optional)}]
This problem explores why the choice of norm matters in regularization, 
and gives you a preview of geometry in higher dimensions.
\end{problembox}

\begin{enumerate}
  \item In $\R^2$, the \textbf{unit ball} for a norm $\norm{\cdot}$ is the set of all 
  vectors with norm at most 1:
  $$
  B = \left\{ \vx \in \R^2 \;\middle|\; \norm{\vx} \leq 1 \right\}
  $$
  Describe the shape of $B$ for:
    \begin{enumerate}
      \item The $\ell_2$ norm: $\norm{\vx}_2 = \sqrt{x_1^2 + x_2^2}$
      \item The $\ell_1$ norm: $\norm{\vx}_1 = |x_1| + |x_2|$
      \item The $\ell_\infty$ norm: $\norm{\vx}_\infty = \max(|x_1|, |x_2|)$
    \end{enumerate}
  For each, describe the shape in words (e.g., ``a disk'', ``a square'') and 
  identify the four points where the boundary intersects the coordinate axes.

  \item Lasso regularization uses the $\ell_1$ penalty $\norm{\vbeta}_1$ rather 
  than the $\ell_2$ penalty $\norm{\vbeta}_2^2$. The shape of the $\ell_1$ unit 
  ball (from part a) explains geometrically why Lasso tends to produce 
  \textbf{sparse} solutions (many components of $\vbeta$ exactly zero) 
  while Ridge does not. In two or three sentences, explain this geometric 
  intuition using your answers from part (a).
  
  \emph{Note}: You do not need to prove this formally. A geometric argument 
  using the shapes of the unit balls is sufficient.
\end{enumerate}

\vspace{20pt}


% ============================================================
% REFLECTION
% ============================================================
\begin{tcolorbox}[
  colback=ts-light,
  colframe=ts-green,
  title=Reflection,
  fonttitle=\bfseries\color{ts-green}
]
Before moving to the Python lab, take two minutes to answer these in your notes:

\begin{enumerate}
  \item Where in your solutions did you rely on geometric intuition rather than 
  pure computation? Was that intuition correct?

  \item The statistical bridge problem introduced vectors in the context of 
  Ridge regression. Can you identify two other places in machine learning or 
  statistics where a vector norm appears, even implicitly?

  \item What question do you still have that these exercises did not resolve?
\end{enumerate}

\medskip
\textbf{Next:} Python Lab (\texttt{python-lab.ipynb}) $\;\to\;$ 
Solutions (\texttt{threat-surfaces-solutions} repository)
\end{tcolorbox}


% ============================================================
\end{document}
% ============================================================
